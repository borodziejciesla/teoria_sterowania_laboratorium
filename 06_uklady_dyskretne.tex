\documentclass{article}

\usepackage{polski}
\usepackage[utf8]{inputenc}

\usepackage[margin=0.75in]{geometry}

\usepackage{graphicx} % Required for inserting images
\usepackage{amsmath}
\usepackage{amssymb}
\usepackage{float}
\usepackage[shortlabels]{enumitem}
\usepackage{matlab-prettifier}

\title{
    Teoria Sterowania\\
    \Large Układy Dyskretne
}
\author{Maciej Różewicz}
\date{2025}

\makeatletter         
\def\@maketitle{
\raggedright
\begin{center}
    \includegraphics[width=0.5\linewidth]{fig/agh_logo.PNG}\\[8ex]
\end{center}
\begin{center}
    {\huge \bfseries \sffamily \@title }\\[4ex] 
    {\large  \@author}\\[4ex] 
    \@date\\[8ex]
\end{center}}
\makeatother

\begin{document}
\maketitle
\newpage

%%%%%%%%%%%%%%%%%%%%%%%%%%%%%%%%%%%%%%%%%%%%%%%%%%%%%%%%%%%
\section{Cel ćwiczenia}
Celem ćwiczenia jest zapoznanie się z własnościami systemów dynamicznych dyskretnych w czasie.
Systemy takie opisuje się zwykle za pomocą równań rekurencyjnych zamiast równań różniczkowych, oraz transmitancji opartej na transformacie $\mathcal{Z}$ zamiast transformacie Laplace'a.
Z modelami takimi można spotkać się w praktyce bardzo często, praktycznie zawsze przy sterowaniu komputerowym i numerycznym rozwiązywaniu równań różniczkowych.

Istnieją również przypadki, gdzie opis taki wynika wprost z jego natury.

%%%%%%%%%%%%%%%%%%%%%%%%%%%%%%%%%%%%%%%%%%%%%%%%%%%%%%%%%%%
\section{Wprowadzenie}
Najczęściej dyskretne równania układu regulacji uzyskiwane są w rezultacie cyfrowej realizacji algorytmów sterowania układami ciągłymi.
Schemat takiego układu sterowania przedstawiony jest na rysunku (\ref{fig:schemat_regulacji_cyfrowej}).
\begin{figure}[H]
    \centering
    \includegraphics[width=0.75\linewidth]{fig/06_uklady_dyskretne/dyskretny_schemat.png}
    \caption{Schemat układu regulacji cyfrowej - algorytm regulacji realizowany na komputerze.}
    \label{fig:schemat_regulacji_cyfrowej}
\end{figure}
Istotnym faktem jest, że w pamięci komputera przechowywane są tylko poszczególne próbki z pomiarów uzyskanych przez przetwornik analogowo-cyfrowy (A/C), zatem nie można mówić o funkcjach czasu, a o szeregach liczbowych.
Matematycznie można operację próbkowania sygnału w dziedzinie czasu $x(t)$ zapisać jako iloczyn tego sygnału z szeregiem impulsów Diraca:
$$
    x^{D}(Tk) = \sum_{k=-\infty}^{\infty}{x(Tk)\cdot\delta(t-Tk)}.
$$

    %======================================================
    \subsection{Dyskretyzacja}
    W ogólności będziemy rozważać stacjonarne równania różniczkowe w postaci:
    $$
        \dot{x}(t) = f(x(t),u(t)).
    $$
    Jednak ponieważ będą nas interesowały jedynie wartości funkcji tylko w dyskretnych chwilach czasowych, możliwe do wyznaczenia na podstawie poprzednich, to możemy posłużyć się ogólną formułą:
    \begin{equation}
        \label{eq:dyskretyzacja_ogolna}
        x(Tk + T) = \int_{Tk}^{Tk+T}{f(x(t), u(t))dt}.
    \end{equation}
    
        %-------------------------------------------------
        \subsubsection{Schemat różnicowy}
        Podana powyżej formuła jest bardzo ogólna, jednak w wielu przypadkach trudna do analitycznego wyznaczenia.
        Można wówczas wspomagać się pewnymi przybliżeniami.
        Jedną z opcji jest przybliżenie pochodnej ilorazem różnicowym.
        Można to zrobić na różne sposoby i otrzymujemy wówczas odpowiednio:
        \begin{enumerate}[(a)]
            \item aproksymacja lewostronna (\textit{backward}):
            \begin{itemize}
                \item $\dot{x}(Tk) = \lim_{\Delta \rightarrow 0} \frac{x(Tk) - x(Tk-\Delta)}{\Delta}  \approx \frac{x[Tk] -x[T(k-1)]}{T}$,
                \item $\ddot{x}(Tk) = \lim_{\Delta \rightarrow 0} \frac{\dot{x}(Tk) - \dot{x}(Tk-\Delta)}{\Delta}  \approx \frac{x[Tk] - 2x[T(k-1)] + x[T(k-2)]}{T^2}$,
                \item $\dddot{x}(Tk) = \lim_{\Delta \rightarrow 0} \frac{\ddot{x}(Tk) - \ddot{x}(Tk-\Delta)}{\Delta}  \approx \frac{x[Tk] - 3x[T(k-1)] + 3x[T(k-2)] + x[T(k-3)]}{T^3}$,
                \item $\dots$,
            \end{itemize}
            \begin{figure}[H]
                \centering
                \includegraphics[width=0.5\linewidth]{fig/06_uklady_dyskretne/backward_int.png}
                \caption{Aproksymacja lewostronna (\textit{backward}).}
                \label{fig:enter-label}
            \end{figure}

            \item aproksymacja prawostronna (\textit{forward}):
            \begin{itemize}
                \item $\dot{x}(Tk) = \lim_{\Delta \rightarrow 0} \frac{x(Tk+\Delta) - x(Tk)}{\Delta}  \approx \frac{x[T(k+1)] - x[Tk]}{T}$,
                \item $\ddot{x}(Tk) = \lim_{\Delta \rightarrow 0} \frac{\dot{x}(Tk+\Delta) - \dot{x}(Tk)}{\Delta}  \approx \frac{x[T(k+2)] - 2x[T(k+1)] + x[Tk]}{T^2}$,
                \item $\dots$,
            \end{itemize}
            \begin{figure}[H]
                \centering
                \includegraphics[width=0.5\linewidth]{fig/06_uklady_dyskretne/forward_int.png}
                \caption{Aproksymacja prawostronna (\textit{forward}).}
                \label{fig:enter-label}
            \end{figure}

            \item aproksymacja symetryczna:
            \begin{itemize}
                \item $\dot{x}(Tk) = \lim_{\Delta \rightarrow 0} \frac{x(Tk+\Delta) - x(Tk-\Delta)}{\Delta}  \approx \frac{x[T(k+1)] - x[T(k-1)]}{T}$,
                \item $\ddot{x}(Tk) = \lim_{\Delta \rightarrow 0} \frac{\dot{x}(Tk+\Delta) - \dot{x}(Tk-\Delta)}{\Delta}  \approx \frac{x[T(k+2)] - 2x[Tk] + x[T(k-2)]}{T^2}$,
                \item $\dots$..
            \end{itemize}
            \begin{figure}[H]
                \centering
                \includegraphics[width=0.5\linewidth]{fig/06_uklady_dyskretne/tustin_int.png}
                \caption{Aproksymacja symetryczna.}
                \label{fig:enter-label}
            \end{figure}
        \end{enumerate}

        \textbf{Przykład:} rozważmy równanie różniczkowe w postaci:
        $$
            a_{2}\ddot{x} + a_{1}\dot{x} + a_{0}x = b_{1}\dot{u} + b_{0}u.
        $$
        Stosując pokazane powyżej przybliżenia pochodnych, uzyskujemy odpowiednio:
        \begin{enumerate} [(a)]
            \item aproksymacja lewostronna:
                $$
                    a_{2} \frac{x[Tk] - 2x[T(k-1)] + x[T(k-2)]}{T^2}
                    +
                    a_{1} \frac{x[Tk] -x[T(k-1)]}{T}
                    +
                    a_{0} x[Tk]
                    =
                    b_{1} \frac{u[Tk] - u[T(k-1)]}{T}
                    +
                    b_{0} u[Tk],
                $$
                a po grupowaniu odpowiednich próbek sygnałów $x$ i $u$:
                $$
                    \left( \frac{a_{2}}{T^2} + \frac{a_{1}}{T} + a_{0} \right) x[Tk]
                    -           
                    \left( \frac{a_{2}}{T^2} + \frac{a_{1}}{T} \right) x[T(k-1)]
                    +
                    \frac{a_{2}}{T^2} x[T(k-2)]
                    =
                    \left( \frac{b_{1}}{T} + b_{0} \right) u[T(k-1)]
                    +
                    b_{0}  u[Tk],
                $$

            \item aproksymacja prawostronna:
            $$
                a_{2} \frac{x[T(k+2)] - 2x[T(k+1)] + x[Tk]}{T^2}
                +
                a_{1} \frac{x[T(k+1)] - x[Tk]}{T}
                +
                a_{0} x[Tk]
                =
                b_{1} \frac{u[T(k+1)] - u[Tk]}{T}
                +
                b_{0} u[Tk],
            $$
            a po grupowaniu odpowiednich próbek sygnałów $x$ i $u$:
            $$
                \frac{a_{2}}{T^2} x[T(k+2)]
                +           
                \left( \frac{a_{1}}{T} + \frac{a_{2}}{T^2} \right) x[T(k+1)]
                +
                \left( \frac{a_{2}}{T^2} - \frac{a_{1}}{T} + a_{0} \right) x[Tk]
                =
                \frac{b_{1}}{T} u[T(k+1)]
                +
                \left( b_{0} - \frac{b_{1}}{T} \right) u[Tk].
            $$                
        \end{enumerate}

        Widać zatem, że postać uzyskanych równań różnicowych istotnie zależy od przyjętej aproksymacji.
        Zatem nie można bezproblemowo przejść z układu dyskretnego na ciągły nie wiedząc o zastosowanej metodzie aproksymacji.

        %--------------------------------------------------
        \subsubsection{Równania rekurencyjne}
        W przypadku układów liniowych w postaci
        $$
            \dot{x} = Ax + Bu
        $$
        wyrażenie (\ref{eq:dyskretyzacja_ogolna})  ma analityczne rozwiązanie.
        $$
            x(t) = e^{A(t-t_{0})}x(t_{0}) + \int_{t_{0}}^{t}{e^{A(t-\tau)}Bu(\tau)d\tau}
        $$
        Zakładając, że chwila początkowa $t_{0}$ będzie początkiem $k+1$-tego okresu próbkowania $t_{0} = Tk$, można wyznaczyć dokładna wartość na jego końcu $x[T(k+1)]$:
        $$
            x[T(k+1)] = e^{A(T(k+1) - Tk)}x(Tk) + \int_{Tk}^{Tk+s}{e^{A(T(k+1) - \tau)}Bu(\tau)d\tau}
        $$
        Można dokonać podstawienia $\xi = T(k+1) - \tau$, które implikuje $d\tau = -d\xi$ oraz zmianę granic całkowania:
        $$
            \tau=Tk \rightarrow \xi = T,\,\tau =Tk + \tau \rightarrow \xi = T-\tau
        $$
        Zakładamy, że sterowanie jest stałe na przedziale próbkowania, dlatego można $u[\tau]$ wyciągnąć poza całkę:
        $$
            x[T(k+1)] = e^{AT}x[Tk] + \left\{ \int_{T-\tau}^{T}{e^{A\xi}Bd\xi} \right\} u[Tk]
        $$
        Jeżeli $\tau = T$, to całkowanie obejmuje przedział $[0,T]$.
        Równanie przyjmuje więc postać równania różnicowego, opisującego dyskretną postać $n$-tego rzędu:
        \begin{equation}
            \label{eq:rownanie_roznicowe}
            x[T(k+1)] = A_{D}x[Tk] + B_{D}u[Tk]
        \end{equation}
        Zapisując skrótowo $x_{k} = x[Tk]$, mamy postać:
        $$
            x_{k+1} = A_{D}x_{k} + B_{D}u_{k},
        $$
        gdzie:
        \begin{itemize}
            \item $A_{D} = e^{AT}$,
            \item $B_{D} = \int_{0}^{T}{e^{A\tau}Bd\tau}$.
        \end{itemize}

        Macierze te można wyznaczyć np. w następujący sposób:
        \begin{enumerate}
            \item $A_{D} =e^{AT} = Pe^{JT}P^{-1}$, gdzie:
                \begin{itemize}
                    \item $P$ - macierz modalna macierzy $A$ - złożona z wektorów głównych macierzy $A$,
                    \item $J$ - postać Jordana macierzy $A$,
                \end{itemize}

            \item $B_{D} = \int_{0}^{T}{e^{A\tau}Bd\tau} = A^{-1}\left( e^{AT} - I \right)B$.
        \end{enumerate}

        Jak wygląda rozwiązanie?
        $$
            x_{1} = A_{D}x_{0} + B_{D}u_{0}
        $$
        $$
            x_{2}
            =
            A_{D}x_{1} + B_{D}u_{1}
            =
            A_{D}\left( A_{D}x_{0} + B_{D}u_{0} \right) + B_{D}u_{1}
            =
            A_{D}^{2}x_{0} + A_{D}B_{D}u_{0} + B_{D}u_{1}
        $$
        $$
            \vdots
        $$
        $$
            x_{n} = A_{D}^{n}x_{0}
            +
            \begin{bmatrix}
                \begin{array}{ccccc}
                    A_{D}^{n-1}B_{D} & A_{D}^{n-1}B_{D} & \dots & A_{D}B_{D}    & B_{D}
                \end{array}
            \end{bmatrix}
            \cdot
            \begin{bmatrix}
                \begin{array}{c}
                    u_{0}\\ u_{1}\\ \vdots\\    u_{n-2}\\ u_{n-1}
                \end{array}
            \end{bmatrix}
        $$

        \textbf{Przykład:} rozważmy układ:
        $$
            \ddot{y} + \dot{y} = u,
        $$
        który przekształcamy do postaci macierzowej prze podstawienie $x_{1} = y,\,x_{2} = \dot{y}$:
        $$
            \begin{bmatrix}
            \begin{array}{c}
                \dot{x}_{1}\\
                \dot{x}_{2}
            \end{array}                
            \end{bmatrix}
            =
            \begin{bmatrix}
            \begin{array}{rr}
                0 & 1\\
                0 & -1
            \end{array}                
            \end{bmatrix}
            \cdot
            \begin{bmatrix}
            \begin{array}{c}
                x_{1}\\
                x_{2}
            \end{array}                
            \end{bmatrix}
            +
            \begin{bmatrix}
            \begin{array}{c}
                0\\
                1
            \end{array}                
            \end{bmatrix}
            u
            = Ax + Bu
        $$
        Macierze układu dyskretnego wynoszą:
        \begin{itemize}
            \item $ A_{D} = 
            \begin{bmatrix}
            \begin{array}{rc}
                1 & -1\\
                0 & 1
            \end{array}                
            \end{bmatrix}
            \cdot
            \begin{bmatrix}
            \begin{array}{rc}
                1 & 0\\
                0 & e^{-T}
            \end{array}                
            \end{bmatrix}
            \cdot
            \begin{bmatrix}
            \begin{array}{rc}
                1 & 1\\
                0 & 1
            \end{array}                
            \end{bmatrix}
            =
            \begin{bmatrix}
            \begin{array}{rc}
                1 & 1-e^{-T}\\
                0 & e^{-T}
            \end{array}                
            \end{bmatrix}
            $,
            \item $ B_{D} = 
            \begin{bmatrix}
            \begin{array}{c}
                T + e^{-T} - 1\\
                1 - e^{-T}
            \end{array}                
            \end{bmatrix}
            $
        \end{itemize}

        %------------s
        \subsubsection{Transformata $\mathcal{Z}$}
        Układ z idealnym impulsatorem daje sterowanie
        $$
            u^{D}(t) = \sum_{k=0}^{\infty}{u[Tk]\delta(t-Tk)}.
        $$
        Transformata Laplace'a:
        $$
            U^{D(s)}
            =
            \int_{0}^{\infty}{\left[ u[Tk]\delta(t-Tk) \right]e^{-st}dt}
            =
            \sum_{k=0}^{\infty}{u[Tk]}\cdot \int_{0}^{\infty}{\delta(t-Tk)e^{-st}dt}
            =
            \sum_{k=0}^{\infty}{u[Tk]}e^{-Tks}.
        $$
        Robiąc podstawienie $z = e^{Ts}$ otrzymuje się:
        \begin{equation}
            \label{eq:transformata_z}
            U^{D(s)}|_{e^{Ts}=z} = \infty_{k=0}^{\infty}{u[Tk]\cdot z^{-k}}
        \end{equation}


        Niektóre właściwości transformaty $\mathcal{Z}$:
        \begin{enumerate}
            \item $y_{k} = w_{k} + x_{k} \rightarrow Y(z) = W(z) + X(z)$ - transformata sumy jest sumą transformat,
            \item $\mathcal{Z}\left\{ x(n-k) \right\} = z^{-k}\mathcal{Z}\left\{ x(n) \right\} = z^{-k}X(z)$ - opóźnienie sygnału o $k$ próbek odpowiada przemnożeniu sygnału o $z^{-k}$,
            \item $\mathcal{Z}\left\{y_{k} \sum_{m=-\infty}^{\infty}{x_m}y_{n-m} \right\} = \mathcal{Z}\left\{ x_{n} \right\}\cdot \mathcal{Z}\left\{ y_{n} \right\} = X(z)Y(z)$ - transformata splotu jest iloczynem transformat,
            \item $\lim_{n\rightarrow\infty}y_{n} = \lim_{z\rightarrow 1}{(1-z^{-1})}Y(z)$
        \end{enumerate}

        
        \begin{table}[H]
            \centering
            \begin{tabular}{|c|c|c|c|}
                \hline
                Oryginał    &   Transformata $\mathcal{L}$  &   Transformata $\mathcal{Z}$ bez członu ZOH   & Transformata $\mathcal{Z}$ z członem ZOH\\
                \hline
                $1(t)$  &   $\frac{1}{s}$   &   $\frac{z}{z-1}$ &   $\frac{T}{z-1}$\\
                \hline
                $t$     &   $\frac{1}{s^2}$ &   $\frac{Tz}{(z-1)^2}$    &   $\frac{T^{2}(z+1)}{2(z-1)^{2}}$\\
                \hline
                $1(t)\frac{t^2}{2}$ &   $\frac{1}{s^3}$  &  $\frac{Tz(z+1)}{2(z-1)^3}$  &   $\frac{T^{3}(z^{2}+4z+1)}{3(z-1)^{3}}$\\
                \hline
                $1(t)e^{-at}$ & $\frac{1}{s+a}$ &   $\frac{z}{z-e^{-aT}}$  &    $\frac{1-e^{aT}}{a(z-e^{-aT})}$ \\
                \hline
                $1(t)\sin{(\omega t)}$  &   $\frac{\omega^2}{s^2 + \omega^2}$  &    $\frac{z\omega\sin{\omega T}}{z^2 - 2z\cos{(\omega T)} + 1}$  & $\frac{(z+1)(1-\cos{(\omega T)})}{z^2 - 2z\cos{(\omega T)}+1}$ \\
                \hline
            \end{tabular}
            \caption{Transformaty $\mathcal{Z}$ wybranych sygnałów.}
            \label{tab:transformata_z_sygnaly}
        \end{table}

        Można zauważyć, że z podstawienia zastosowanego do transformaty $\mathcal{Z}$ wynika:
        $$
            z = e^{Ts} = e^{T(\alpha + i\beta)} = e^{\alpha T}e^{iT\beta} = e^{\alpha T} \left( \cos{(T\beta)} + i \sin{(T\beta)} \right)
        $$
        zatem $z$ jest punktem na okręgu o promieniu $e^{T\alpha}$ na płaszczyźnie zespolonej.
        Można więc zauważyć, że dla $\alpha < 0$, czyli lewej półpłaszczyzny zespolonej, $z$ będzie leżało w okręgu jednostkowym.
        Oznacz to, że przekształcenie to przekształca lewą półpłaszczyznę w okrąg jednostkowy.

        %++++++++++++++++++++++++++++
        \subsubsection{Aproksymacja transmitancji dyskretnej}
        Przy wprowadzaniu transformaty $\mathcal{Z}$  wprowadzone zostało podstawienie $z = e^{Ts}$, zatem aby transmitancję $G(s)$ przekształciś w transmitancję dyskretną w dziedzinie operatora $z$ wystarczy wykonać podstawienia za $s$.
        Niestety nie istnieje wygodna do użycia dokładna wartość $s$ wyznaczana z $z$, dlatego wprowadzono kilka wygodnych aproksymacji:
        \begin{enumerate} [(a)]
            \item metoda całkowania "do przodu" (\textit{forward})
            $$
                z = e^{sT} = 1+sT + \frac{s^2T^2}{2} + \dots \approx 1 + sT \implies s = \frac{z-1}{T}
            $$
            \begin{figure}[H]
                \centering
                \includegraphics[width=0.5\linewidth]{fig/06_uklady_dyskretne/forward_transform.png}
                \caption{Przekształcenie lewej półpłaszczyzny dla aproksymacji \textit{forward}.}
                \label{fig:enter-label}
            \end{figure}
            
            \item metoda całkowania "do tyłu" (\textit{backward}):
            $$
                z = e^{sT} = \frac{1}{e^{-sT}} \approx \frac{1}{1 - sT} \implies s = \frac{z-1}{Tz}
            $$
            \begin{figure}[H]
                \centering
                \includegraphics[width=0.5\linewidth]{fig/06_uklady_dyskretne/backward_transform.png}
                \caption{Przekształcenie lewej półpłaszczyzny dla aproksymacji \textit{backward}.}
                \label{fig:enter-label}
            \end{figure}
            
            \item całkowania metodą trapezów (metoda \textit{Tustina}):
            $$
                z = e^{sT} = \frac{1 + e^{sT}}{1 + e^{-sT}} \approx \frac{2 + sT}{ 2 - sT} \implies s = \frac{2}{T}\frac{z-1}{z+1}
            $$
            \begin{figure}[H]
                \centering
                \includegraphics[width=0.5\linewidth]{fig/06_uklady_dyskretne/tustin_transform.png}
                \caption{Przekształcenie lewej półpłaszczyzny dla aproksymacji Tustina.}
                \label{fig:enter-label}
            \end{figure}
        \end{enumerate}

        \textbf{Przykład:} rozważmy transmitancję obiektu inercyjnego pierwszego rzędu:
        $$
            G(s) = \frac{a}{s+a}
        $$
        \begin{itemize}
            \item metoda \textit{forward}:
            $$
                G_{D,F}(z) = \frac{a}{\frac{z-1}{T} + a} = \frac{aT}{z-1+aT}
            $$
            \item metoda \textit{backward}:
            $$
                G_{D,B}(z) = \frac{a}{\frac{z-1}{Tz} + a} = \frac{aTz}{z(1+aT)-1}
            $$
            \item metoda \textit{Tustina}:
            $$
                G_{D,T}(z) = \frac{a}{\frac{2}{T}\frac{z-1}{z+1} + a} = \frac{aT(z+1)}{z(2+aT)-2+aT}
            $$
        \end{itemize}

        %++++++++++++++++++++++++++++
        \subsubsection{Ekstrapolator zerowego rzędu}
        Przedstawiona w poprzednim punkcie metoda analizuje sygnał jako \textit{de facto} ciąg impulsów.
        O ile jest to prawda w przypadku pomiaru, gdzie faktycznie otrzymujemy wartości sygnału zmierzone w konkretnych chwilach czasu, to w przypadku sterowania nie jest to już prawdą.
        Ponieważ sterowanie w zdecydowanej większości praktycznych przypadków nie jest podawane jako seria impulsów, lecz jako stała wartość podtrzymywana przez okres próbkowania - tzw. ekstrapolacja zerowego rzędu (ang. \textit{zero order hold} - \textbf{ZOH}).
        Co pokazano na rysunku (\ref{fig:implse_to_zoh}).
        \begin{figure}[H]
            \centering
            \includegraphics[width=0.5\linewidth]{fig/06_uklady_dyskretne/implse_to_zoh.png}
            \caption{Ekstrapolacja zerowego rzędu.}
            \label{fig:implse_to_zoh}
        \end{figure}
        matematycznie można to osiągnąć poprzez wykonanie splotu funkcji impulsowej z funkcją prostokątną:
        $$
            h(s) = \mathbf{1}(t) - \mathbf{1}(t-T) \implies H(s) = \frac{1}{s} - e^{-Ts}\frac{1}{s} = \frac{1 - e^{-Ts}}{s} 
            \approx
            \frac{T}{s},\,{\text{dla $T\approx 0$}}.
        $$
        Zatem w schemacie blokowym można dodać dodatkowy blok \textbf{ZOH}, który realizuje transmitancję $H(s)$:
        \begin{figure}[H]
            \centering
            \includegraphics[width=0.5\linewidth]{fig/06_uklady_dyskretne/zoh.png}
            \caption{Ekstrapolacja zerowego rzędu - schemat blokowy.}
            \label{fig:implse_to_zoh}
        \end{figure}
        Zatem transmitancja dyskretna wynosi:
        $$
            G(z) = \mathcal{Z}\left\{ \mathcal{L}^{-1} \left\{ G(s)\frac{1-e^{-Ts}}{s} \right\} \right\} 
            =
            (1-z^{-1})\mathcal{Z}\left\{ \mathcal{L}^{-1} \left\{ \frac{G(s)}{s} \right\} \right\} 
            =
            \frac{z-1}{z}\mathcal{Z}\left\{ g(kT) \right\},
        $$
        gdzie $g(kT)$ jest ciągiem próbek.
        
        
    %======================================================
    \subsection{Dyskretna wersja twierdzenia Lapunowa}
    Rozważamy dyskretny system dynamiczny opisywany równaniem (\ref{eq:system_dyskretny}):
    \begin{equation}
        \label{eq:system_dyskretny}
        \left\{
        \begin{array}{rcl}
            x(k+1)  &   =   &   f[x(k)]\\
            x(0)    &   =   & x_{0}
        \end{array}
        \right\},
    \end{equation}
    przy czym bez straty ogólności możemy założyć, że $f(0) = 0$.
    To jeśli istnieje ciągła, dodatnio określona na pewnym otoczeniu zerowego punktu równowagi $D$ funkcja $V(x)$
    Która spełnia warunki:
    \begin{itemize}
        \item $V(x) > 0m\, \forall x \in D - \{0\}$,
        \item $\Delta V(x) = V(f[x]) - V(x) \leq 0, \, \forall x\in D$,
    \end{itemize}
    to punkt $0$ jest stabilny, a funkcję $V$ nazywamy dyskretną funkcją Lapunowa.

    \textbf{Przykład:} rozważmy równanie różniczkowe:
    $$
        \dot{x} = -ax,
    $$
    które jest stabilne dla $a>0$, a które można przekształcić do postaci dyskretnej za pomocą
    \begin{enumerate}[(a)]
        \item aproksymacji schematami różnicowymi:
        $$
            \frac{k_{k+1}-x_{k}}{T} = -ax_{k} \implies x_{k+1} = (1-aT)x_{k} = f[x_{k}], \, T>0.
        $$
        Możemy przyjąć funkcję Lapunowa w postaci:
        $$
            V(x_{k}) = x_{k}^{2},
        $$
        a jej $\Delta$ wynosi:
        $$
            \Delta V(x_{k})
            =
            V(f[x_{k}]) - V(x_{k})
            =
            (1-aT)^{2}x_{k}^{2} - x_{k}^{2}
            =
            x_{k}^{2}aT(1-2aT).
        $$
        Zatem, żeby była ona ujemne musi zachodzić warunek:
        $$
            aT \in (0, 2) \implies a>0,\,a<\frac{2}{T} \implies a>0\,\text{gdy}\,T\rightarrow0.
        $$
        widać zatem, że przybliżenie schematem różnicowym zawęża zakres stabilności wraz ze wzrostem okresu dyskretyzacji $T$,

        \item równaniami różnicowymi:
        $$
            x_{k+1} = A_{D} x_{k},\,\text{gdzie}\,A_{D} = e^{-aT}.
        $$
        Funkcję Lapunowa zakładamy jak poprzednio, zatem w tym przypadku jej $\Delta$ wynosi:
        $$
            \Delta V(x_{k}) = V(f(x_{k})) - V(x_{k}) = e^{-2aT}x_{k}^{2} - x_{k}^{2} = x_{k}^{2}(e^{-2aT} -1) < 0,\,\text{dla}\, a>0,
        $$
        zatem widać, iż taka metoda wyznaczania równania dyskretnego nie jest obarczona błędem jak aproksymacja schematami różnicowymi.
    \end{enumerate}

    Podobnie jak w przypadku ciągłym nie istnieje uniwersalna metoda wyznaczania funkcji Lapunowa.
    Jednak również i tym razem dla szczególnego przypadku, równania liniowego:
    \begin{equation}
        \label{eq:lin_dysk}
        x_{k+1} = Ax_{x}    
    \end{equation}
    
    \textbf{Twierdzenie:} jeśli dla równania dyskretnego, liniowego (\ref{eq:lin_dysk}) istnieje rozwiązanie $P=P^{T}>0$ dyskretnego równania Lapunowa (\ref{eq:dysk_lapunowa})
    \begin{equation}
        \label{eq:dysk_lapunowa}
        A^{T}PA - P = -Q,
    \end{equation}
    gdzie $Q=Q^{T}>0$, to układ (\ref{eq:lin_dysk}) jest globalnie asymptotycznie stabilny, a $V(x) = x^{T}Px$ jest jego funkcją Lapunowa.
    A jej delta na trajektoriach systemu ma postać:
    $$
        \Delta V(x_{k}) = V(Ax_{k}) - V(x_{k}) = (Ax_{x})^{T}P(Ax_{k}) - x_{k}^{T}Px_{k} =x_{k}^{T}A^{T}PAx_{k} - x_{k}^{T}Px_{k} = x_{k}^{T}(A^{T}PA-P)x_{k} < 0
    $$

    %======================================================
    \subsection{Kryteria stabilności}
    Mówimy, że wielomian dyskretny
    \begin{equation}
            \label{eq:wielomian}
            w(z) = a_{n}z^n + a_{n-1}z^{n-1} + \dots + a_{1}z + a_{0}
        \end{equation}
    jest stabilny asymptotycznie, gdy jego pierwiastki leżą w kole jednostkowym.
    Do stwierdzenia, czy jego pierwiastki spełniają ten warunek bez ich bezpośredniego wyznaczania, można użyć np. jednego z poniższych kryteriów.

        %--------------------------------------------------
        \subsubsection{Kryterium stabilności I}
        Znając współczynniki $a_{i}$ wielomianu $w(z)$ możemy wyznaczyć tzw. macierz symetryczną Schura-Cohna w postaci:
        \begin{equation}
            \label{eq:schu_cohen}
            P = S_{1}^{T}S_{1} - S_{2}^{T}S_{2},
        \end{equation}
        gdzie:
        \begin{itemize}
            \item $
                S_{1}
                =
                \begin{bmatrix}
                \begin{array}{ccccc}
                    a_{n}   &   a_{n-1}     &   \dots   &   a_{2}       &   a_{1}\\
                    0       &   a_{n}       &   \dots   &   a_{3}       &   a_{2}\\
                    \vdots  &   \vdots      &   \dots   &   \vdots      &   \vdots\\
                    0       &   0           &   \dots   &   0           &   a_{n}
                \end{array}
                \end{bmatrix}
            $,

            \item $
                S_{2}
                =
                \begin{bmatrix}
                \begin{array}{ccccc}
                    a_{0}   &   a_{1}   &   \dots   &   a_{n-2} &   a_{n-1}\\
                    0       &   a_{0}   &   \dots   &   a_{n-3} &   a_{n-2}\\
                    \vdots  &   \vdots  &   \dots   &   \vdots  &   \vdots\\
                    0       &   0       &   \dots   &   0       &   a_{0}
                \end{array}
                \end{bmatrix}
            $.
        \end{itemize}
        Podstawiając macierze $S_{1}$ i $S_{2}$ do równania (\ref{eq:schu_cohen}) otrzymuje się wyrażenie na element $p_{ij}$ macierzy $P$:
        \begin{equation}
            \label{eq:schu_cohen_pij}
            p_{ij} = \sum_{t=0}^{i-1}{
            \left( a_{n-i+t}a_{n-j+t+1} - a_{i-1-t}a_{j-1-t} \right)
            },\,j\geq0.
        \end{equation}

        Wówczas:
        \begin{enumerate}
            \item Wielomian (\ref{eq:wielomian}) jest stabilny asymptotycznie wtedy i tylko wtedy, gdy macierz $P$ (równanie (\ref{eq:schu_cohen})) jest dodatnio określona, czyli zachodzą warunki:
            \begin{itemize}
                \item $P_{1} = p_{11} > 0$,
                \item $P_{2}
                =
                \begin{vmatrix}
                \begin{array}{cc}
                    p_{11}  &   p_{12}\\
                    p_{21}  &   p_{22}
                \end{array}
                \end{vmatrix} > 0
                $,
                \item $\vdots$
                \item $P_{n} = |P| > 0$.
            \end{itemize}

            \item Jeżeli $P_{i} \neq 0$ dla $i = 1, \dots, n$, to wielomian (\ref{eq:wielomian}) ma $k$ pierwiastków wewnątrz koła jednostkowego oraz $n-k$ na zewnątrz koła jednostkowego, przy czym zachodzi równość:
            \begin{equation}
                \label{eq:stab_kryt_k}
                k = n - V(1, P_{1}, \dots, P_{n}),
            \end{equation}
            gdzie $V(e_{0}, e_{1}, \dots, e_{n})$ oznacza liczbę zmian znaku ciągu $e_{0}, e_{1}, \dots, e_{n}$.
        \end{enumerate}

        %--------------------------------------------------
        \subsubsection{Kryterium stabilności II}
        Znając współczynniki $a_{i}$ wielomianu (\ref{eq:wielomian}) można wyznaczyć tablicę:
        \begin{equation}
            \label{eq:kryterium_tablica}
            \left\{
            \begin{array}{cccc}
                c_{1,1} &   c_{1,2} &   \dots   &   c_{1,n+1} \\
                d_{1,1} &   d_{1,2} &   \dots   &   d_{1,n+1} \\
                c_{2,1} &   c_{2,2} &   \dots   &   c_{2,n} \\
                d_{2,1} &   d_{2,2} &   \dots   &   d_{2,n} \\
                c_{3,1} &   c_{3,2} &   \dots   &   c_{2,n-1} \\
                d_{3,1} &   d_{3,2} &   \dots   &   d_{2,n-1} \\
                \vdots  &   \vdots  &   \vdots  &   \vdots\\
                c_{n+1, 1}  & & &\\
                d_{n+1, 1}  & & &
            \end{array}
            \right\},
        \end{equation}
        gdzie:
        \begin{itemize}
            \item $c_{1,j+1} = a_{n-j}$, $d_{1,j+1} = a_{j}$, dla $j=0, 1, \dots, n$,
            \item 
            $
                c_{ij}
                =
                \begin{vmatrix}
                \begin{array}{cc}
                    c_{i-1,1}   &   c_{i-1,j+1}\\
                    d_{i-1,1}   &   d_{i-1,j+1}
                \end{array}
                \end{vmatrix}
            $, dla $i = 1, \dots, n$, $j = 0, \dots, n-1$,
            \item $d_{ij} = c_{n-i,j+i-3}$.
        \end{itemize}

        Wówczas:
        \begin{enumerate}
            \item Wielomian (\ref{eq:wielomian}) jest stabilny asymptotycznie wtedy i tylko wtedy, gdy elementy $d_{i,1}$ pierwszej kolumny tablicy (\ref{eq:kryterium_tablica}) spełniają warunki:
            $$
                d_{2, 1} > 0, d_{i,1} < 0,\, \text{dla}\,i = 3, \dots, n+1.
            $$

            \item Jeżeli $d_{i,1} \neq 0$ dla $i=1, \dots, n+1$ to wielomian (\ref{eq:wielomian}) ma $k$ pierwiastków wewnątrz koła jednostkowego i $n-k$ poza kołem jednostkowym, przy czym $k$ jest liczbą ujemnych iloczynów ciągu:
            $$
                d_{k} = (-1)^{k} \cdot d_{2,1} \cdot d_{3,1} \cdot \dots \cdot d_{k+1, 1}, \, \text{dla}\, k = 1, 2, \dots, n.
            $$
        \end{enumerate}

    %======================================================
    \subsection{Kryteria niestabilności}
    Mówimy, że wielomian dyskretny
    $$
        w(z) = a_{n}z^n + a_{n-1}z^{n-1} + \dots + a_{1}z + a_{0}
    $$
    jest niestabilny gdy jego pierwiastki leżą poza kołem jednostkowym.
    Do stwierdzenia czy jego pierwiastki spełniają ten warunek bez ich bezpośredniego wyznaczania można użyć np. jednego z poniższych kryteriów.

        %--------------------------------------------------
        \subsubsection{Kryterium niestabilności I}
        Wielomian (\ref{eq:wielomian}) jest niestabilny jeżeli zachodzi warunek:
        $$
            w(1) = \sum_{k=0}^{n}{a_{k}}
            \left\{
            \begin{array}{ccl}
                > 0 & \text{dla}    &   a_{n} > 0\\
                < 0 & \text{dla}    &   a_{n} < 0
            \end{array}
            \right\},
        $$
        lub
        $$
            (-1)^{n}w(-1) =
            \left\{
            \begin{array}{ccl}
                > 0 & \text{dla}    &   a_{n} > 0\\
                < 0 & \text{dla}    &   a_{n} < 0
            \end{array}
            \right\}.
        $$

        %--------------------------------------------------
        \subsubsection{Kryterium niestabilności II}
        Liniowy układ dyskretny opisany równaniem
        $$
            x_{k+1} = Ax_{k},
        $$
        gdzie:
        \begin{itemize}
            \item $A
            =
            \begin{bmatrix}
            \begin{array}{cccc}
                a_{11}  &   a_{12}  &   \dots   &   a_{1n}\\
                a_{21}  &   a_{22}  &   \dots   &   a_{2n}\\
                \vdots  &   \dots   &   \dots   &   \vdots\\
                a_{n1}  &   a_{n2}  &   \dots   &   a_{nn}
            \end{array}
            \end{bmatrix}
            \in \mathbb{R}^{n \times n},
            $
            \item $x_{i} \in \mathbb{R}^{n}$.
        \end{itemize}
        jest niestabilny asymptotycznie, jeżeli:
        $$
            \sum_{i=1}^{n}{|a_{ii}|} \geq n
        $$

    %======================================================
    \subsection{Dobór okresu próbkowania}
    Podstawowym twierdzeniem dotyczącym doboru okresu próbkowania/dyskretyzacji jest tzw. twierdzenie Shannona.
    Mówi ono nam, że aby sygnał ciągły mógł być jednoznacznie odtworzony z dyskretnych próbek, musi być on próbkowany z częstotliwością $f_{p}$ co najmniej dwa razy większą niż maksymalna częstotliwość widma sygnału $f_{g}$, zatem:
    $$
        f_{p} > 2f_{g}.
    $$
    W przypadku układów sterowania, gdzie nie mamy z góry zadanego sygnału wyjściowego, zatem i jego częstotliwości, to graniczna wartość częstotliwości może być wyznaczona jako częstotliwość odcięcia (czyli częstotliwość, dla której tłumienie staje się większe od 3 dB), odczytana z charakterystyki Bodego obiektu sterowania.

    Nie bez znaczenia w przypadku sterowania cyfrowego jest też minimalny czas konieczny na realizację sterowania.
    Czyli czas wykonywania pomiaru, przesłania z czujnika do układu mikroprocesorowego, czas wyznaczenia sterowania oraz czas konieczny na przesłanie sygnału sterującego do elementu wykonawczego.


%%%%%%%%%%%%%%%%%%%%%%%%%%%%%%%%%%%%%%%%%%%%%%%%%%%%%%%%%%%
\section{Przykłady}
    %======================================================
    \subsection{Przykład 1}
    Rozważamy układ o transmitancji:
    $$
        G(s) = \frac{s+1}{s^2 + s+ 1}.
    $$
    Sprawdźmy realizację różnych metod dyskretyzacji dostępnych w MATLABie na podstawie odpowiedzi skokowej układu po dyskretyzacji.
    Wyniki przedstawiono na rysunku (\ref{fig:przyklad_1}).
    \begin{figure}[H]
        \centering
        \includegraphics[width=0.75\linewidth]{fig/06_uklady_dyskretne/przyklad_1.png}
        \caption{Realizacja różnych metod dyskretyzacji.}
        \label{fig:przyklad_1}
    \end{figure}
    
    %======================================================
    \subsection{Przykład 2}
    Rozważmy ciągły układ liniowy:
    $$
        \left\{
        \begin{array}{l}

            \dot{x}
            =
            \begin{bmatrix}
            \begin{array}{rr}
                -1 & 5\\
                0 & -2
            \end{array}                
            \end{bmatrix}
            x
            +
            \begin{bmatrix}
            \begin{array}{c}
                0\\
                1
            \end{array}                
            \end{bmatrix}
            u
            \\

            \\

            y
            =
            \begin{bmatrix}
            \begin{array}{cc}
                1 & 0
            \end{array}                
            \end{bmatrix}
            x
            
        \end{array}
        \right\},
    $$
    Z regulatorem liniowym:
    $$
        u = Ky.
    $$
    Można zauważyć, np. korzystając z kryterium Nyquista, że układ ten będzie stabilizowany, dla każdego $K>0$, dla ustalenia uwagi przyjmijmy $K=5$.

    Załóżmy, że czas próbkowania w układzie sterowania cyfrowego wynosi $T=0.1$, wówczas macierze układu zdyskretyzowanego mają postać:
    \begin{itemize}
        \item $A_{D}
        =
        \left[\begin{array}{cc} {\mathrm{e}}^{-T} & 5\,{\mathrm{e}}^{-T}-5\,{\mathrm{e}}^{-2\,T}\\ 0 & {\mathrm{e}}^{-2\,T} \end{array}\right]|_{T=0.1}
        =
        \left[\begin{array}{cc} 0.9048 & 0.4305\\ 0 & 0.8187 \end{array}\right]
        $,
        \item $B_{D}
        =
        \left[\begin{array}{c} \frac{5\,{\mathrm{e}}^{-2\,T}}{2}-5\,{\mathrm{e}}^{-T}+\frac{5}{2}\\ \frac{1}{2}-\frac{{\mathrm{e}}^{-2\,T}}{2} \end{array}\right]|_{T=0.1}
        =
        \left[\begin{array}{c} 0.0226\\ 0.0906 \end{array}\right]
        $,
        \item $C_{D} = \begin{bmatrix}
            \begin{array}{cc}
                1 & 0
            \end{array}                
            \end{bmatrix}$,
        \item $D_{D} = 0$.
    \end{itemize}

    Budujemy model symulacyjny, w którym jednocześnie wykonamy symulację układu ciągłego i dyskretnego:
    \begin{figure}[H]
        \centering
        \includegraphics[width=0.5\linewidth]{fig/06_uklady_dyskretne/Screenshot from 2024-05-12 15-53-29.png}
        \caption{Model symulacyjny dla przykładu 2.}
        \label{fig:enter-label}
    \end{figure}

    Wyniki symulacji przedstawiono na rysunku(\ref{fig:wyjscie_przyklad_2}).
    \begin{figure}[H]
        \centering
        \includegraphics[width=0.5\linewidth]{fig/06_uklady_dyskretne/przyklad_2_dyskretne.png}
        \caption{Wyjście układu sterowania z czasem ciągłym i dyskretnym dla przykładu 2.}
        \label{fig:wyjscie_przyklad_2}
    \end{figure}
    Na wykresie tym widać znaczące rozbieżności między przebiegiem układu z czasem ciągłym i dyskretnym.
    Wynikają one z dyskretyzacji i faktu, że sterowanie podawane jako stała wartość na przedziale inaczej wpływa na układ.
    Ogólnie można zauważyć, iż wraz ze wzrostem czasu dyskretyzacji różnice te będą się powiększać.
    A przy odpowiednio dużym okresie dyskretyzacji może powodować destabilizację układu starowania, co pokazano na rysunku (\ref{fig:wyjscie_przyklad_2_2}).
    \begin{figure}[H]
        \centering
        \includegraphics[width=0.5\linewidth]{fig/06_uklady_dyskretne/przyklad_2_dyskretne_2.png}
        \caption{Wyjście układu sterowania z czasem ciągłym i dyskretnym, z różnymi okresami dyskretyzacji, dla przykładu 2.}
        \label{fig:wyjscie_przyklad_2_2}
    \end{figure}
    

    %======================================================
    \subsection{Przykład 3}
    Rozważmy ciągły układ liniowy:
    $$
        \left\{
        \begin{array}{l}

            \dot{x}
            =
            \begin{bmatrix}
            \begin{array}{rr}
                -1 & 5\\
                0 & -2
            \end{array}                
            \end{bmatrix}
            x
            +
            \begin{bmatrix}
            \begin{array}{c}
                0\\
                1
            \end{array}                
            \end{bmatrix}
            u
            \\

            \\

            y
            =
            \begin{bmatrix}
            \begin{array}{cc}
                1 & 0
            \end{array}                
            \end{bmatrix}
            x
            
        \end{array}
        \right\},
    $$
    Z regulatorem liniowym:
    $$
        u = Ky.
    $$
    Czyli ten sam układ co poprzednio, jednak tym razem regulator nie będzie projektowany dla układu ciągłego lecz dla układu po dyskretyzacji.
    Oznacz to, że będziemy poszukiwać $K$ zapewniającego, iż wartości własne macierzy
    $$
        A_{z} = A_{D} - B_{D}KC_{D} = 
        \left[\begin{array}{cc} {\mathrm{e}}^{-T}-\mathrm{K}\,\left(\frac{5\,{\mathrm{e}}^{-2\,T}}{2}-5\,{\mathrm{e}}^{-T}+\frac{5}{2}\right) & 5\,{\mathrm{e}}^{-T}-5\,{\mathrm{e}}^{-2\,T}\\ +\mathrm{K}\,\left(\frac{{\mathrm{e}}^{-2\,T}}{2}-\frac{1}{2}\right) & {\mathrm{e}}^{-2\,T} \end{array}\right].
    $$
    będą leżeć wewnątrz koła jednostkowego.

    Wielomian charakterystyczny układu zamkniętego ma postać:
    $$
        w(z) = z^2+\left(-\frac{{\mathrm{e}}^{-3\,T}\,\left(2\,{\mathrm{e}}^{2\,T}+2\,{\mathrm{e}}^T+5\,\mathrm{K}\,{\mathrm{e}}^T-10\,\mathrm{K}\,{\mathrm{e}}^{2\,T}+5\,\mathrm{K}\,{\mathrm{e}}^{3\,T}\right)}{2}\right)\,z-\frac{{\mathrm{e}}^{-3\,T}\,\left(5\,\mathrm{K}-10\,\mathrm{K}\,{\mathrm{e}}^T+5\,\mathrm{K}\,{\mathrm{e}}^{2\,T}-2\right)}{2}.
    $$
    Do znalezienia zakresu $K$ stabilizującego układ zamknięty możemy wykorzystać pierwsze z podanych kryteriów, zatem wyznaczamy:
    \begin{itemize}
        \item $
        S_{1}
        =
        \left[\begin{array}{cc} 1 & -\frac{{\mathrm{e}}^{-3\,T}\,\left(2\,{\mathrm{e}}^{2\,T}+2\,{\mathrm{e}}^T+5\,\mathrm{K}\,{\mathrm{e}}^T-10\,\mathrm{K}\,{\mathrm{e}}^{2\,T}+5\,\mathrm{K}\,{\mathrm{e}}^{3\,T}\right)}{2}\\ 0 & 1 \end{array}\right],
        $
        \item $
        S_{2}
        =
        \left[\begin{array}{cc} -\frac{{\mathrm{e}}^{-3\,T}\,\left(5\,\mathrm{K}-10\,\mathrm{K}\,{\mathrm{e}}^T+5\,\mathrm{K}\,{\mathrm{e}}^{2\,T}-2\right)}{2} & -\frac{{\mathrm{e}}^{-3\,T}\,\left(2\,{\mathrm{e}}^{2\,T}+2\,{\mathrm{e}}^T+5\,\mathrm{K}\,{\mathrm{e}}^T-10\,\mathrm{K}\,{\mathrm{e}}^{2\,T}+5\,\mathrm{K}\,{\mathrm{e}}^{3\,T}\right)}{2}\\ 0 & -\frac{{\mathrm{e}}^{-3\,T}\,\left(5\,\mathrm{K}-10\,\mathrm{K}\,{\mathrm{e}}^T+5\,\mathrm{K}\,{\mathrm{e}}^{2\,T}-2\right)}{2} \end{array}\right],
        $
        \item $
        P = S_{1}^{T}S_{1} - S_{2}^{T}S_{2}
        $
        %\left[\begin{array}{cc} 1-\frac{{\mathrm{e}}^{-6\,T}\,{\left(5\,\mathrm{K}-10\,\mathrm{K}\,{\mathrm{e}}^T+5\,\mathrm{K}\,{\mathrm{e}}^{2\,T}-2\right)}^2}{4} & -\frac{{\mathrm{e}}^{-3\,T}\,\left(2\,{\mathrm{e}}^{2\,T}+2\,{\mathrm{e}}^T+5\,\mathrm{K}\,{\mathrm{e}}^T-10\,\mathrm{K}\,{\mathrm{e}}^{2\,T}+5\,\mathrm{K}\,{\mathrm{e}}^{3\,T}\right)}{2}-\frac{{\mathrm{e}}^{-6\,T}\,\left(5\,\mathrm{K}-10\,\mathrm{K}\,{\mathrm{e}}^T+5\,\mathrm{K}\,{\mathrm{e}}^{2\,T}-2\right)\,\left(2\,{\mathrm{e}}^{2\,T}+2\,{\mathrm{e}}^T+5\,\mathrm{K}\,{\mathrm{e}}^T-10\,\mathrm{K}\,{\mathrm{e}}^{2\,T}+5\,\mathrm{K}\,{\mathrm{e}}^{3\,T}\right)}{4}\\ -\frac{{\mathrm{e}}^{-3\,T}\,\left(2\,{\mathrm{e}}^{2\,T}+2\,{\mathrm{e}}^T+5\,\mathrm{K}\,{\mathrm{e}}^T-10\,\mathrm{K}\,{\mathrm{e}}^{2\,T}+5\,\mathrm{K}\,{\mathrm{e}}^{3\,T}\right)}{2}-\frac{{\mathrm{e}}^{-6\,T}\,\left(5\,\mathrm{K}-10\,\mathrm{K}\,{\mathrm{e}}^T+5\,\mathrm{K}\,{\mathrm{e}}^{2\,T}-2\right)\,\left(2\,{\mathrm{e}}^{2\,T}+2\,{\mathrm{e}}^T+5\,\mathrm{K}\,{\mathrm{e}}^T-10\,\mathrm{K}\,{\mathrm{e}}^{2\,T}+5\,\mathrm{K}\,{\mathrm{e}}^{3\,T}\right)}{4} & 1-\frac{{\mathrm{e}}^{-6\,T}\,{\left(5\,\mathrm{K}-10\,\mathrm{K}\,{\mathrm{e}}^T+5\,\mathrm{K}\,{\mathrm{e}}^{2\,T}-2\right)}^2}{4} \end{array}\right].
    \end{itemize}
    Niestety warunki staja się skomplikowane i trudno wyznaczyć ogólny warunek na $K$ w zależności od czasu dyskretyzacji $T$.
    Dlatego skupimy się na konkretnym czasie $T=0.5$, który destabilizował układ regulacji przy $K$ dobranym dla układu z czasem ciągłym.
    Wówczas mamy:
    $$
        A_z = 
        \begin{bmatrix}
            \begin{array}{rr}
                0.3870\cdot K + 0.6065 & 1.1933\\
                0.3161\cdot K & 0.3679
            \end{array}                
            \end{bmatrix}
    $$
    Na tej podstawie otrzymujemy warunek $K<3.3093$, a przykładowe przebiegi pokazano na rysunku.
    \begin{figure}[H]
        \centering
        \includegraphics[width=0.5\linewidth]{fig/06_uklady_dyskretne/przyklad_3_dyskretne.png}
        \caption{Przykładowe przebiegi dla układu z przykładu 3.}
        \label{fig:enter-label}
    \end{figure}

    %======================================================
    \subsection{Przykład 4}
    Rozważmy układ liniowy:
    $$
        x_{k+1}
        = Ax_{k} =
        \begin{bmatrix}
        \begin{array}{ccc}
            0   &   1   &   0\\
            0   &   0   &   1\\
            -0.3&   -0.6&   -3.2
        \end{array}
        \end{bmatrix}
        x_{k}.
    $$
    Sprawdzić czy układ jest niestabilny?
    \begin{enumerate}
        \item \textbf{Kryterium I}: wielomian charakterystyczny ma postać:
        $$
            w(z) = det(Iz - A)
            =
            \begin{vmatrix}
            \begin{array}{ccc}
                z   &   1   &   0\\
                0   &   z   &   1\\
                0.3 &   0.6 &   z+3.2
            \end{array}
            \end{vmatrix}
            =
            z^{3} + 3.2z^{2} + 0.6z + 0.3
        $$.
        Zgodnie z kryterium sprawdzamy:
        $$
            w(1) = 5.1 > 0,
        $$
        zatem ponieważ $a_{n} > 0$, to układ jest niestabilny.

        \item \textbf{Kryterium II}: zgodnie z kryterium sprawdzamy odpowiednią sumę:
        $$
            \sum_{i=1}^{3}{|a_{ii}|} = 3.2 > 3,
        $$
        zatem również na podstawie tego kryterium stwierdzamy niestabilność.
    \end{enumerate}

    %======================================================
    \subsection{Przykład 5}
    Wyznaczyć liczbę pierwiastków znajdujących się wewnątrz koła jednostkowego wielomianu
    $$
        w(z) = z^{3} - z^{2} + 2z + 3.
    $$

    Zgodnie z \textbf{kryterium stabilności I} wyznaczamy macierze:
    \begin{itemize}
        \item
        $
        S_{1}
        =
        \begin{bmatrix}
        \begin{array}{crr}
            1   &   -1  &   2\\
            0   &   1   &   -1\\
            0   &   0   &   1
        \end{array}
        \end{bmatrix},
        $

        \item
        $
        S_{2}
        =
        \begin{bmatrix}
        \begin{array}{ccr}
            3   &   2   &   -1\\
            0   &   3   &   2\\
            0   &   0   &   3
        \end{array}
        \end{bmatrix},
        $

        \item 
        $
        P = S_{1}^{T}S_{1} - S_{2}^{T}S_{2}
        =
        \begin{bmatrix}
        \begin{array}{rrr}
            -8  &   -7   &   5\\
            -7  &   -11  &   -7\\
            5   &   -7   &   -8
        \end{array}
        \end{bmatrix}.
        $
    \end{itemize}
    Minory główne macierzy $P$ wynoszą:
    \begin{itemize}
        \item $P_{1} = -8$,
        \item $P_{2} = 39$,
        \item $P_{3} = 845$.
    \end{itemize}
    Zatem ciąg $1, P_{1}, P_{2}, P_{3}$ ma dwie zmiany znaku, a z tego wynika, że liczba pierwiastków wewnątrz koła jednostkowego wynosi:
    $$
        k = n - V(1, -8, 39, 845) = 1.
    $$

    %======================================================
    %\subsection{Przykład 5}
   
%%%%%%%%%%%%%%%%%%%%%%%%%%%%%%%%%%%%%%%%%%%%%%%%%%%%%%%%%%%
\section{Jak to zrobić w MATLABie?}
    %======================================================
    \subsection{Dyskretyzacja systemu liniowego}
    Control System Toolbox MATLABa posiada funkcję \texttt{c2d}, która służy do przejścia z modelu czasem ciągłym na model z czasem dyskretnym z zadanym czasem dyskretyzacji i podaną metodą.
    Funkcja ta działa z modelem ciągłym podanym w  postaci równań stanu jak i transmitancji.
    \begin{lstlisting}[style=Matlab-editor]
        %% Systenm ciagly
        % Ddefinicja rownan stanu systemu liniowego
        A = [-1 2; 0.1 -1];
        B = [0;1];
        C = [1, 0];
        D = 0;
        ct_ss = ss(A, B, C, D);
        % Wyznaczenie transmitancji
        [num, den] = ss2tf(A, B, C, D);
        ct_tf = tf(num, den);
        
        %% Dyskretyzacja
        T = 1;
        dt_ss = c2d(ct_ss, T, 'zoh');
        dt_tf = c2d(ct_tf, T, 'tustin');
    \end{lstlisting}

    Funkcja ta obsługuje następujące metody dyskretyzacji:
    \begin{itemize}
        \item \texttt{zoh},
        \item \texttt{foh},
        \item \texttt{impulse},
        \item \texttt{tustin},
        \item \texttt{matched},
        \item \texttt{least-squares},
        \item \texttt{damped}.
    \end{itemize}

    %======================================================
    \subsection{Analityczne wyznaczenie macierzy układu dyskretnego}
    Funkcja \texttt{c2d} wyznacza numeryczne wartości układu dyskretnego.
    Wartości symboliczne dla równania (\ref{eq:rownanie_roznicowe}) można wyznaczać w następujący sposób:
    \begin{lstlisting}[style=Matlab-editor]
        A = [-1 1; 0 -1];
        B = [0;1];
        C = [1, 0];
        D = 0;
        
        [P,J] = jordan(A);
        syms T
        
        Ad = P * expm(J*T) * inv(P);
        Bd = inv(A) * (expm(A*T) - eye(size(A))) * B;
        Cd = C;
        Dd = D;
    \end{lstlisting}

    %======================================================
    \subsection{Łączenie czasu dyskretnego i ciągłego w jednej symulacji}
    W przypadku gdy chcemy w jednym modelu Simulinka połączyć symulację działania obiektu ciągłego wraz z naszym projektowanym regulatorem realizowanym cyfrowo działającym z czasem dyskretnym możemy stosować 2 bloki:
    \begin{itemize}
        \item ZOH - ekstrapolator zerowego rzędu, wówczas "obudowując" model takimi blokami zapewniamy, że symulacja modelu będzie wykonywana w czasie ciągłym, natomiast na zewnątrz czas może być dyskretny,
        \item Rate-Transition - blok ten pozwala ustalać czasy symulacji z jakimi będą wykonywane otoczone nim elementy modelu, jest to bardziej ogólna metoda niż ZOH, pozwala na ustawianie wielu różnych bloków czasowych w modelu.
    \end{itemize}
    Przykład użycia tych bloków pokazano na rysunku (\ref{fig:zoh_rt}).
    \begin{figure}[H]
        \centering
        \includegraphics[width=0.75\linewidth]{fig/06_uklady_dyskretne/simulink_model.png}
        \caption{Zastosowanie bloków ZOH i Rate-Transition w modelu Simulinka}
        \label{fig:zoh_rt}
    \end{figure}
        

%%%%%%%%%%%%%%%%%%%%%%%%%%%%%%%%%%%%%%%%%%%%%%%%%%%%%%%%%%%
\section{Przebieg ćwiczenia}
W trakcie realizacji ćwiczeń należy dla każdego z przykładów:
\begin{itemize}
    \item wyznaczyć regulator proporcjonalny stabilizujący układ po zamknięciu pętli sprzężenia zwrotnego,
    \item wyznaczyć maksymalny okres dyskretyzacji wynikający z charakterystyki częstotliwościowej,
    \item zamodelować układ w Simulinku - użyć modelu ciągłego i "obudować" go blokami ZOH,
    \item wyznaczyć odpowiedź na wymuszenie skokowe układu ciągłego i dyskretnego - wyznaczyć analitycznie lub symulacyjnie maksymalny okres próbkowania, dla którego regulator wyznaczony dla układu ciągłego zapewni stabilność przy sterowaniu dyskretnym,
    \item dokonać dyskretyzacji układu ciągłego - metodą z punktu 2.1.2 dla układu danego w postaci równań stanu, metodą Tustina dla układów danych w postaci transmitancji,
    \item wyznaczyć dyskretny regulator proporcjonalny stabilizujący układ sterowania,
    \item wyznaczyć odpowiedź na wymuszenie skokowe układu ciągłego i dyskretnego,
    \item wyznaczyć model dyskretne dla różnych metod dyskretyzacji dostępnych w metodzie \texttt{c2d}, porównać odpowiedzi na skok jednostkowy układu ciągłego i układów dyskretnych, porównać zakres $K$ stabilizujących dla różnych metod dyskretyzacji.
\end{itemize}
    %======================================================
    \subsection{System 1}
    $$
        \dot{x}
        =
        \begin{bmatrix}
        \begin{array}{rr}
            -1 & 1\\
            0 & -1
        \end{array}                
        \end{bmatrix}
        x
        +
        \begin{bmatrix}
        \begin{array}{c}
            0\\
            1
        \end{array}                
        \end{bmatrix}
        u
    $$
    $$
        y
        =
        \begin{bmatrix}
        \begin{array}{cc}
            1 & 0
        \end{array}                
        \end{bmatrix}
        x
    $$

    %======================================================
    \subsection{System 2}
    $$
        \dot{x}
        =
        \begin{bmatrix}
        \begin{array}{rr}
            -1 & 1\\
            0 & 1
        \end{array}                
        \end{bmatrix}
        x
        +
        \begin{bmatrix}
        \begin{array}{c}
            0\\
            1
        \end{array}                
        \end{bmatrix}
        u
    $$
    $$
        y
        =
        \begin{bmatrix}
        \begin{array}{cc}
            1 & 0
        \end{array}                
        \end{bmatrix}
        x
    $$

    %======================================================
    \subsection{System 3}
    $$
        G(s) = \frac{1}{s+1}
    $$

    %======================================================
    \subsection{System 3}
    $$
        G(s) = \frac{1}{s(s^2+s+1)}
    $$
   

%%%%%%%%%%%%%%%%%%%%%%%%%%%%%%%%%%%%%%%%%%%%%%%%%%%%%%%%%%%
\newpage
\begin{thebibliography}{9}

\bibitem{Mitkowski2007}
  Mitkowski, W., Baranowski, J., Hajduk, K., Korytowski, A., Tutaj, A.,
  \emph{Teoria Sterowania: Materiały Pomocnicze do Ćwiczeń Laboratoryjnych},
  AGH Uczelniane wydawnictwo Naukowo-Dydaktyczne,
  2007.

\bibitem{Amborski1978}
  Amborski, K., Marusak, A.,
  \emph{Teoria Sterowania w Ćwiczeniach},
  Państwowe Wydawnictwo Naukowe,
  1978.

\bibitem{Gessing1981}
  Gessing, R., Latarnik, M., Skrzywan-Kossek, A.,
  \emph{Zbiór Zadań z Teorii Nieliniowych Układów Regulacji i Sterowania},
  Wydawnictwo Naukowo-Techniczne,
  1981.

\bibitem{Gibson1968}
  Gibson, J. E.,
  \emph{Nieliniowe Układy Sterowania Automatycznego},
  Wydawnictwo Naukowo-Techniczne,
  1968.

\bibitem{Grabowski1999}
  Grabowski P.,
  \emph{Stabilność Układów Lurie},
  AGH Uczelniane Wydawnictwo Naukowo-Dydaktyczne,
  1999.

  \bibitem{Byrski2007}
  Byrski W.,
  \emph{Obserwacja i sterowanie w systemach dynamicznych},
  Wydawnictwo Naukowo-Dydaktyczne AGH,
  2007.

  \bibitem{Kaczorek1993}
  Kaczorek T.,
  \emph{Teoria sterowania i systemów},
  Państwowe Wydawnictwo Naukowe,
  1993.

\end{thebibliography}

\end{document}
