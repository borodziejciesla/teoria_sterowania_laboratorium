\documentclass{article}

\usepackage[margin=0.75in]{geometry}

\usepackage{polski}
\usepackage[utf8]{inputenc}
\usepackage{graphicx} % Required for inserting images
\usepackage{amsmath}
\usepackage{amssymb}
\usepackage{float}
\usepackage[shortlabels]{enumitem}
\usepackage{matlab-prettifier}
\usepackage{tcolorbox}

\DeclareMathOperator*{\argmax}{arg\,max}

\title{
    Teoria Sterowania\\
    \Large Metoda Lapunowa
}
\author{Maciej Różewicz}
\date{2025}

\makeatletter         
\def\@maketitle{
\raggedright
\begin{center}
    \includegraphics[width=0.55\linewidth]{fig/agh_logo.PNG}\\[8ex]
\end{center}
\begin{center}
    {\huge \bfseries \sffamily \@title }\\[4ex] 
    {\large  \@author}\\[4ex] 
    \@date\\[8ex]
\end{center}}
\makeatother

\begin{document}
\maketitle

%%%%%%%%%%%%%%%%%%%%%%%%%%%%%%%%%%%%%%%%%%%%%%%%%%%%%%%%%%%%
\newpage
\section{Cel ćwiczenia}
Celem ćwiczenia jest zapoznanie się z metodami badania stabilności układów nieliniowych.
W czasie zajęć przedstawione i przećwiczone zostaną dwie metody badania stabilności:
\begin{itemize}
    \item pośrednia metoda Lapunowa - prosta metoda, dająca zadowalające wyniki w wielu praktycznych przypadkach, jednak mająca pewne ograniczenia,
    \item bezpośrednia metoda Lapunowa - bardzo uniwersalna metoda, pozwalająca zbadać stabilność szerokiej gamy układów (zarówno stacjonarnych, jak i niestacjonarnych).
\end{itemize}
Ponadto po sprawdzeniu stabilności punktów równowagi układu badany i estymowany będzie 
obszar przyciągania asymptotycznego za pomocą twierdzenia La Salle'a.
Dodatkowo sprawdzana będzie stabilność nie tylko punktów równowagi, ale i cykli granicznych.

%%%%%%%%%%%%%%%%%%%%%%%%%%%%%%%%%%%%%%%%%%%%%%%%%%%%%%%%%%%%
\section{Wprowadzenie}
W ćwiczeniu tym rozważamy nieliniowe układy dynamiczne, które mogą być opisane skończonym układem równań różniczkowych zwyczajnych w postaci:
\begin{equation}\label{eq:sys}
    \dot{x}(t) = f(t, x)
\end{equation}
gdzie: $t\in \mathbb{R}$, $x\in\mathbb{R}^{n}$ i $f: \mathbb{R} \times \mathbb{R}^{n} \rightarrow \mathbb{R}^{n}$.

\textbf{Definicja 1}
Punktem równowagi układu (\ref{eq:sys}) nazywamy każdy stan $x_{e}$, dla którego $f(t, x_{e}) = 0$ dla każdego $t>t_{0}$.

\textbf{Definicja 2}
Punkt równowagi $x_{e}$ jest nazywany izolowanym punktem równowagi, jeśli istnieje jego otoczenie, niezawierające żadnych innych punktów równowagi.

\textbf{Definicja 3}
Jeżeli istnieje zmienne w czasie rozwiązanie równania (\ref{eq:sys}) spełniające warunek
$$
    x(t+T) = x(t)
$$
dla pewnego $T>0$ i każdego $t>t_{0}$, to nazywamy je cyklem granicznym. Trajektoria w przestrzeni stanów, odpowiadająca cyklowi granicznemu, ma postać zamkniętej krzywej.

\begin{tcolorbox}[colback=green!5, colframe=green!75!black, title=Przykład]
Przykładem cyklu granicznego jest tak zwany oscylator Van der Pola:
\begin{equation}\label{eq:van_der_pol}
    \left\{
    \begin{array}{l}
         \dot{x}_{1} = x_{2}\\
         \dot{x}_{2} = -x_{1} + (1-x_{1})x_{2} 
    \end{array}
    \right\}.
\end{equation}
Na rysunku (\ref{fig:van_der_pol}) przedstawiono portret tego układu, gdzie wyraźnie widoczny jest występujący tam cykl graniczny.
\begin{figure}[H]
    \centering
    \includegraphics[width=0.5\linewidth]{fig/03_lapunow/van_der_pol.png}
    \caption{Portret fazowy dla oscylatora Van der Pola (równanie \ref{eq:van_der_pol}).}
    \label{fig:van_der_pol}
\end{figure}
\end{tcolorbox}

Liczba cykli granicznych nie musi być ograniczona tylko do jednego' lub nawet
skończonej liczby - może być ich dowolna ilość, nawet nieskończenie wiele.

\begin{tcolorbox}[colback=green!5, colframe=green!75!black, title=Przykład]
Rozważmy układ (\ref{eq:cykle_graniczne}):
\begin{equation}\label{eq:cykle_graniczne}
    \left\{
    \begin{array}{l}
         \dot{x}_{1} = x_{2}\\
         \dot{x}_{2} = -\left[ 1 + \sin{(x_{1}^{2}+x_{2}^{2})} \right] x_{2} - x_{1}
    \end{array}
    \right\}.
\end{equation}
Stosunkowo łatwo zauważyć, że cykl graniczny będzie istniał dla każdego $x_{1}$ i $x_{2}$ spełniającego równanie:
$$
    1 + \sin{(x_{1}^{2} + x_{2}^{2})} = 0.
$$
Zatem musi zachodzić:
$$
    x_{1}^{2} + x_{2}^{2} = \frac{3}{2}\pi + 2n\pi.
$$
Portret fazowy tego układu pokazany jest na rysunku (\ref{fig:cykle_graniczne}).
\begin{figure}[H]
    \centering
    \includegraphics[width=0.5\linewidth]{fig/03_lapunow/cykle_graniczne.png}
    \caption{Portret fazowy dla układu opisanego układem równań \ref{eq:cykle_graniczne}.}
    \label{fig:cykle_graniczne}
\end{figure}
\end{tcolorbox}

\textbf{Definicja 4}
Punkt równowagi $x_{e}$  układu (\ref{eq:sys}) nazywamy stabilnym w sensie Lapunowa, jeśli dla dowolnych $t_{0} > 0$, $\epsilon > 0$ istnieje liczba $\delta_{\epsilon} > 0$ (może zależeć od $t_{0}$ i $\epsilon$) taka, że:
\begin{equation}\label{eq:lyap_def}
    \lVert x_{e} - x_{0} \rVert < \delta_{\epsilon}(t_{0}) \implies \lVert x_{e} - x(t;t_{0},x_{0}) \rVert < \epsilon.
\end{equation}
%Układ ten ma jeden punkt równowagi w początku układu współrzędnych, a jego obszarem atrakcji jest cała płaszczyzna $\mathbb{R}^{2}$, lecz nie jest on stabilny. Ponieważ nie można dobrać takiego $\delta$, aby implikacja (\ref{eq:lyap_def}) była spełniona.

\textbf{Definicja 5}
Punkt równowagi $x_{e}$ układu (\ref{eq:sys}) nazywamy niestabilnym, jeśli nie jest stabilny w sensie definicji 4.

\textbf{Definicja 6}
Punkt równowagi $x_{e}$ układu (\ref{eq:sys}) nazywamy jednostajnie stabilnym, jeśli $\delta_{e}$ z definicji 4 jest niezależne od czasu początkowego $t_{0}$.

\textbf{Definicja 7}
Punkt równowagi $x_{e}$  układu (\ref{eq:sys}) nazywamy stabilnym asymptotycznie, jeśli jest stabilny (w sensie definicji 4) oraz istnieje $\delta_{\epsilon} > 0$ takie że:
\begin{equation}\label{eq:as}
    \lVert x_{e} - x_{0} \rVert < \delta_{\epsilon}(t_{0}) \implies 
 \lim_{t\rightarrow\infty}\lVert x_{e} - x(t;t_{0},x_{0}) \rVert = 0.
\end{equation}

\begin{figure}[H]
    \centering
    \includegraphics[width=0.5\linewidth]{fig/03_lapunow/stabilnosc_zbiory.PNG}
    \caption{Interpretacja geometryczna definicji stabilności.}
    \label{fig:enter-label}
\end{figure}

\textbf{Definicja 8}
Punkt równowagi $x_{e}$  układu (\ref{eq:sys}) nazywamy stabilnym wykładniczo, jeśli istnieje $\alpha > 0$ taka, że dla dowolnego $\epsilon > 0$ istnieje $\delta > 0$ takie, że dla dowolnego $t\geq t_{0}$:
\begin{equation}\label{eq:as}
    \lVert x_{e} - x_{0} \rVert < \delta_{\epsilon}(t_{0}) \implies \lVert x_{e} - x(t;t_{0},x_{0}) \rVert \leq \epsilon e^{-\alpha(t-t_{0})}.
\end{equation}s

\textbf{Definicja 9}
Zbiór wszystkich takich punktów $x_{0}$ w przestrzeni stanów, dla s $\lim_{t\rightarrow\infty}\lVert x_{e} - x(t;t_{0},x_{0}) \rVert = 0$ dla pewnego $t_{0} \geq 0$ nazywamy zbiorem przyciągania (obszarem atrakcji) punktu równowagi $x_{e}$.

\textbf{Definicja 10}
Jeżeli punkt równowagi $x_{e}$ jest asymptotycznie stabilny, a jego obszar atrakcji rozciąga się na całą przestrzeń stanów, to punkt równowagi nazywamy globalnie asymptotycznie stabilnym.

\textbf{Definicja 11}
Funkcja $V: D \rightarrow \mathbb{R}$ jest nazywana:
\begin{itemize}
    \item dodatnio określoną, jeśli $V(0) = 0$ i $\forall_{x\neq 0}V(x) > 0$,
    \item dodatnio półokreśloną, jeśli $V(0) = 0$ i $\forall_{x\neq 0}V(x) \geq 0$,
    \item ujemnie określoną (półokreśloną), jeśli $-V(x)$ jest dodatnio określona (półokreślona).
\end{itemize}

\begin{tcolorbox}[colback=green!5, colframe=green!75!black, title=Przykład]
Rozważmy układ opisany układem równań (\ref{eq:przyklad_zbiezny_niestabilny}).
\begin{equation}
    \label{eq:przyklad_zbiezny_niestabilny}
    \left\{
    \begin{array}{l}
         \dot{x}_{1} = \frac{x_{1}^{2}(x_{2}-x_{1}) + x_{2}^{5}}{\left( x_{1}^{2} + x_{2}^{2} \right) \left( 1 + \left( x_{1}^{2} + x_{2}^{2} \right)^{2} \right)}\\
         \dot{x}_{2} = \frac{x_{2}^{2}(x_{2}-2x_{1})}{\left( x_{1}^{2} + x_{2}^{2} \right) \left( 1 + \left( x_{1}^{2} + x_{2}^{2} \right)^{2} \right)}
    \end{array}
    \right\}.
\end{equation}
Układ ten ma jeden punkt równowagi w początku układu współrzędnych, a jego obszarem atrakcji jest cała płaszczyzna $\mathbb{R}^{2}$, lecz nie jest on stabilny. Ponieważ nie można dobrać takiego $\delta$, aby implikacja (\ref{eq:lyap_def}) była spełniona.
Kilka jego przykładowych trajektorii okazano na rysunku (\ref{fig:enter-label}).
\begin{figure}[H]
    \centering
    \includegraphics[width=0.5\linewidth]{fig/03_lapunow/przyklad_atr_stab.PNG}
    \caption{Przykładowe trajektorie dla układu (\ref{eq:przyklad_zbiezny_niestabilny}).}
    \label{fig:enter-label}
\end{figure}
\end{tcolorbox}

Dalsze definicje i twierdzenia odnoszą się do systemów stacjonarnych:
\begin{equation}\label{eq:sys_stac}
    \dot{x} = f(x).
\end{equation}
\textbf{Definicja 12}
Funkcja $\dot{V}: D \rightarrow \mathbb{R}$ oraz 
$$
    \dot{V}(x)=\nabla V(x)f(x) = \begin{bmatrix} \frac{\partial V}{\partial x_{1}} & \frac{\partial V}{\partial x_{2}} & \cdots & \frac{\partial V}{\partial x_{n}} \end{bmatrix}
    \begin{bmatrix} f_{1}(x) \\ f_{2}(x) \\ \vdots \\ f_{n}(x) \end{bmatrix}
$$
nazywana jest pochodną wzdłuż trajektorii systemu (\ref{eq:sys_stac}) lub pochodną systemową funkcji $V$ albo pochodną Lie'go wzdłuż pola wektorowego $f$ i oznacza $\mathcal{L}_{f}V(x)$.

\textbf{Twierdzenie 1 (Lapunowa)}
Jeżeli istnieje funkcja $V: D \rightarrow \mathbb{R}$, która jest:
\begin{itemize}
    \item różniczkowalna w sposób ciągły,
    \item dodatnio określona w $D$,
    \item jej pochodna systemowa $\dot{V}:D\rightarrow \mathbb{R}$ jest ujemnie półokreślona w $D$
\end{itemize}
to punkt równowagi $x_{e} = 0$ układu (\ref{eq:sys_stac}) jest stabilny.
A jeśli pochodna systemowa jest ujemnie określona w $D$, to punkt równowagi $x_{e} = 0$ układu (\ref{eq:sys_stac}) jest asymptotycznie stabilny.

\begin{tcolorbox}[colback=blue!5, colframe=blue!75!black, title=Ważne]
Twierdzenie Lapunowa dostarcza warunek konieczny i wystarczający stabilności punktu równowagi, jednak nie dostarcza żadnej informacji o tym, jak odnaleźć funkcję $V(x)$. Odnalezienie tej funkcji jest zasadniczą trudnością w badaniu stabilności.
W punkcie \ref{sec:dobor_funkcji} podano kilka metod poszukiwania funkcji Lapunowa.
\end{tcolorbox}

\textbf{\textit{Dowód:}}
\textit{Stabilność:}
Zgodnie z \textit{definicją 4 }weźmy liczbę $\epsilon > 0$.
Załóżmy, że kula o promieniu $\epsilon$ zawiera się w~zbiorze $D$:
$$
    K_{\epsilon} = \left\{ x\in \mathbb{R}^{n}: \lVert x \rVert \leq \epsilon \right\} \subset D.
$$
Niech:
$$
    \alpha = \min_{\lVert x \rVert = \epsilon }V(x).
$$
Weźmy liczbę $\beta$, taką że $0 < \beta < \alpha$ i oznaczmy zbiór:
$$
    \Omega_{\beta} = \left\{ x \in K_{\epsilon}: V(x) < \beta \right\}.
$$
Zachodzi więc $\Omega_{\beta} \subset K_{\epsilon}$.
Rozważając dowolną trajektorię $x(t) = x(t; t_{0}, x_{0})$ układu (\ref{eq:sys_stac}), taką że $x_{0} \in \Omega_{\beta}$.
Wzdłuż tej trajektorii mamy $\dot{V}(x) \leq 0$, więc $V(x(t)) \leq V(x_{0}) \leq \beta$, czyli cała trajektoria $x(t)$ pozostanie w zbiorze $\Omega_{\beta} \subset K_{\beta}$ dla dowolnego $t$.
Ponadto z ciągłości funkcji $V(x)$ wynika, że istnieje $\delta > 0$ taka, że:
$$
    \lVert x \rVert < \delta \implies V(x) < \beta
$$
czyli zachodzi $K_{\delta} = \left\{ x \in \mathbb{R}^{n}| \Vert x \rVert < \delta \right\} \subset \Omega_{\beta} \subset K_{\beta}$. Tak więc, z takim wyborem $\delta$ implikacja
$$
    \lVert x \rVert < \delta \implies \lVert x \rVert < \epsilon
$$
jest prawdziwa i zgodnie z \textit{definicją 4} $x_{e} = 0$ jest stabilny w sensie Lapunowa.

\textit{Asymptotyczna stabilność:}
Jeżeli pochodna systemowa $\dot{V}:D \rightarrow \mathbb{R}$ jest ujemnie określona w $D$, to wzdłuż trajektorii $x(t) = x(t; t_{0}, x_{0})$ układ (\ref{eq:sys_stac}) funkcja $V(x)$ jest ściśle malejąca i ograniczona od dołu przez 0, więc zbiega do nieujemnej granicy $c$:
$$
    \lim_{t \rightarrow \infty} V(x(t)) = c \geq 0.
$$
Jeśli przypuścimy, że $c > 0$, to oznaczałoby to, że cała trajektoria $x(t)$ pozostaje na zewnątrz zbioru $\Omega_{c} = \left\{ x \in K_{\epsilon}: V(x) \leq c \right\}$, a tym samym na zewnątrz dowolnej kuli $K_{r} = \left\{ x \in \mathbb{R}^{n}: \Vert x \rVert \leq r \right\}$ zawartej w $\Omega_{c}$.
Wtedy wzdłuż trajektorii $x(t)$
$$
    \dot{V}(x) \leq \max_{r \leq \lVert x \rVert \leq \epsilon} \dot{V}(x) = -v < 0
$$
czyli
$$
    V(x(t)) = V(x(t_{0})) + \int_{t_{0}}^{t}{\dot{V}(x(\tau))d\tau} \leq V(x(t_{0})) - v(t - t_{0}).
$$
Dla dostatecznie dużego $t$ funkcja $V(x(t))$ przyjmowałaby więc wartości ujemne, co jest sprzeczne z założeniami co do funkcji $V(x)$.
Musi więc zachodzić:
$$
    \lim_{t \rightarrow \infty}V(x(t)) = 0
$$
co pociąga za sobą:
$$
    \lim_{t \rightarrow\infty} x(t) = 0
$$
i udowadnia asymptotyczną stabilność.
\begin{figure}[H]
    \centering
    \includegraphics[width=0.5\linewidth]{fig/03_lapunow/lap_dowod_zbiory.PNG}
    \caption{Zbiory definiowane w dowodzie twierdzenia Lapunowa o stabilności.}
    \label{fig:enter-label}
\end{figure}

\textbf{Definicja 13}
Funkcja $V:D\rightarrow\mathbb{R}$ spełniająca założenia twierdzenia Lapunowa jest nazywana funkcją Lapunowa.

\begin{figure}[H]
    \centering
    \includegraphics[width=0.5\linewidth]{fig/03_lapunow/funkcja_lapunowa.PNG}
    \caption{Graficzna interpretacja funkcji Lapunowa dla $n=2$.}
    \label{fig:enter-label}
\end{figure}

\textbf{Twierdzenie 2 (Lapunowa o niestabilności)}
Jeżeli istnieje funkcja $V:D\rightarrow \mathbb{R}$ różniczkowalna w sposób ciągły, dodatnio określona w pewnym otoczeniu punktu równowagi $x_{e} = 0$ i taka, że jej pochodna systemowa $\dot{V}:D\rightarrow\mathbb{R}$ jest dodatnio określona w pewnym zbiorze $G = \left\{ x \in K_{\epsilon} : V(x) > 0 \right\}$, gdzie $K_{\epsilon} = \left\{ x \in \mathbb{R}^{n}: \lVert x \rVert \leq \epsilon \right\} \subset D$, to punkt równowagi $x_{e} = 0$ jest niestabilny.

\begin{tcolorbox}[colback=blue!5, colframe=blue!75!black, title=Ważne]
\textbf{Funkcje Lapunowa dla układów liniowych}
W przypadku układów liniowych, które mają ściśle określoną strukturę:
\begin{equation}\label{eq:sys_lin}
    \dot{x} = Ax
\end{equation}
można wykazać, że jeśli istnieje dodatnio określona i symetryczna macierz $P \in \mathbb{R}^{n}$, która spełnia tzw. równanie Lapunowa:
\begin{equation}\label{eq:rownanie_lapunowa}
    A^{T}P + PA = -Q
\end{equation}
gdzie: $Q=Q^{T} > 0$,
to dla układu (\ref{eq:sys_lin}) funkcja Lapunowa istnieje i ma postać:
\begin{equation}
    V(x) = x^{T}Px.
\end{equation}
A jej pochodna systemowa ma postać:
\begin{equation}\label{eq:lin_lap_div}
    \dot{V}(x) = \dot{x}^{T}Px + xP\dot{x} = x^{T}(A^{T}P + PA)x = -x^{T}Qx
\end{equation}
Jest to równoważne z położeniem widma macierzy $A$ w lewej półpłaszczyźnie
zespolonej.
\end{tcolorbox}

\textbf{Twierdzenie 3 (pośrednia metoda Lapunowa)}
Jeżeli badany układ \ref{eq:sys_stac} jest tzw. systemem \textit{słabo-nieliniowym}, tzn. można go zapisać w postaci:
\begin{equation}\label{eq:sys_slabo_niel}
    \dot{x} = Ax + r(x)
\end{equation}
gdzie:
\begin{itemize}
    \item $A \in \mathbb{R}^{n}$,
    \item $r(x): \mathbb{R}^{n} \rightarrow \mathbb{R}^{n}$ jest różniczkowalną, lokalnie lipschitzowską funkcją, która ponadto spełnia warunek:
    \begin{equation}\label{eq:warunek_slabo_niel}
        \lim_{\lVert x \rVert \rightarrow 0} \frac{\lVert r(x) \rVert}{\lVert x \rVert} = 0
    \end{equation}
\end{itemize}
to stabilność punktu równowagi może być analizowana w oparciu o przybliżenie liniowe, zaniedbujące składnik $r(x)$, to znaczy:
$$  
    \dot{x} = Ax.
$$
Wówczas stabilność punktu równowagi zależy od widma macierzy $A$ (jej wartości własnych) w następujący sposób:
\begin{enumerate}
    \item Jeżeli wszystkie wartości własne macierzy $A$ spełniają warunek $Re(\lambda) < 0$, czyli leżą w lewej półpłaszczyźnie zespolonej,  to punkt równowagi jest asymptotycznie stabilny.
    \item Jeżeli którakolwiek z wartości własnych macierzy $A$ leży w prawej półpłaszczyźnie zespolonej, to punkt równowagi jest niestabilny.
    \item Jeżeli wartości własne macierzy $A$ leżą w lewej półpłaszczyźnie zespolonej i choć jeden na osi urojonych, to nie można wnioskować o stabilności punktu równowagi na podstawie przybliżenia liniowego.
\end{enumerate}

\begin{tcolorbox}[colback=orange!5, colframe=orange!75!black, title=Uwaga]
Możliwość zastosowania analizy układu zlinearyzowanego wynika z~twierdzenia
Hartmana-Grobmana, które mówi, że w~otoczeniu punktu równowagi układ nieliniowy
jest topologicznie równoważny z~jego przybliżeniem liniowym (możemy znaleźć lokalny
homeomorfizm z~układu nieliniowego w~liniowy), o~ile wszystkie
wartości własne macierzy $A$ mają część rzeczywistą różną od zera.
Oznacza to, że w~takim przypadku zachowanie układu nieliniowego w~otoczeniu
punktu równowagi jest podobne do zachowania układu liniowego. 
\end{tcolorbox}

\begin{tcolorbox}[colback=green!5, colframe=green!75!black, title=Przykład]
Dla \textbf{przykładu} rozpatrzmy prosty układ pierwszego rzędu:
$$
    \dot{x} = ax + bx^3.
$$
Jego przybliżenie liniowe ma postać:
$$
    \dot{x} = ax
$$
Ponadto $r(x) = bx^3$ jest ciągła, lipschitzowska i spełnia warunek
$$
    \lim_{\lVert x \rVert \rightarrow 0} \frac{\lVert bx^3 \rVert}{\lVert x \rVert} = \lim_{\lVert x \rVert \rightarrow 0} |b| \lVert x^2 \rVert = 0.
$$
Mamy zatem 3 przypadki:
\begin{itemize}
    \item $a<0$ - wówczas układ stabilny,
    \item $a>0$ - wówczas układ niestabilny,
    \item $a=0$ - wtedy o stabilności decyduje $r(x) = bx^3$ - jeśli $b<0$ to stabilny, a jeśli $b>0$ to niestabilny.
\end{itemize}
Prosto jest to zweryfikować bezpośrednią metodą Lapunowa, przyjmując funkcję Lapunowa $V(x) = \frac{1}{2}x^{2}$.
\end{tcolorbox}

\textbf{Definicja 14}
Funkcja $V:D\rightarrow\mathbb{R}$ jest nazywana radialnie nieograniczoną, jeśli $\lVert x \rVert \rightarrow \infty$ implikuje $V(x) \rightarrow \infty$.

\textbf{Twierdzenie 4 (o globalnej stabilności asymptotycznej)}
jeśli istnieje radialnie nieograniczona, ciągle różniczkowalna, dodatnio określona w $\mathbb{R}^{n}$ funkcja $V: D \rightarrow \mathbb{R}$, taka, że jej pochodna systemowa jest ujemnie określona w $\mathbb{R}^{n}$, to punkt równowagi $x_{e} = 0$ układu (\ref{eq:sys_stac}) jest globalnie asymptotycznie stabilny.

\begin{tcolorbox}[colback=green!5, colframe=green!75!black, title=Przykład]
    TODO: Przykład na to, że radialna ograniczoność jest konieczna do globalnej
    stabilności asymptotycznej.
\end{tcolorbox}

\textbf{Twierdzenie 5 (o stabilności wykładniczej)}
Jeżeli istnieje funkcja $V:D\rightarrow\mathbb{R}$ różniczkowalna w sposób ciągły i taka, że dla pewnych dodatnich stałych $k_{1}$, $k_{2}$, $k_{3}$, $h$ są spełnione na zbiorze $D$ warunki:
\begin{equation}\label{eq:stab_wyk}
    \begin{array}{c}
         k_{1}\lVert x \rVert^{h} \leq V(x) \leq k_{2}\lVert x \rVert^{h}\\
         \dot{V}(x) \leq -k_{3}\lVert x \rVert^{h}
    \end{array}
\end{equation}
to punkt równowagi $x_{e} = 0$ układu (\ref{eq:sys_stac}) jest wykładniczo stabilny. Jeżeli $D = \mathbb{R}^{n}$, to punkt równowagi jest globalnie wykładniczo stabilny.

\textbf{Definicja 15}
Zbiór $M$ w przestrzenie stanów układu (\ref{eq:sys_stac}) jest nazywany niezmienniczym (inwariantnym), jeśli każda trajektoria rozpoczynająca się w $M$ w nim pozostaje, czyli:
\begin{equation}\label{eq:zbior_inwariantny}
    x(t_{0}) \in M \implies \forall_{t > t_{0}}x(t) \in M.
\end{equation}

Przykładami zbioru niezmienniczego są: cała przestrzeń stanów, punkt równowagi, cykl graniczny, czy zbiór zdefiniowany jako:
$$
    \omega_{p} = \left\{ x \in \mathbb{R}^{n}:V(x) \leq l \right\}
$$
gdzie $V(x)$ jest funkcją Lapunowa.

\textbf{Definicja 16}
Zbiór $N$ w przestrzeni stanów nazywamy zbiorem granicznym dla trajektorii $x(t)$ układu (\ref{eq:sys_stac}), jeśli dla dowolnego punktu $p \in N$ istnieje ciąg $x(t_{n}) \rightarrow p$, $t_{1} < t_{2} < \cdots$.

Zbiorami granicznymi są zatem punkty asymptotycznie stabilnej równowagi i cykle graniczne dla trajektorii z ich obszarów atrakcji.

\textbf{Twierdzenie 6 (LaSalle'a)}
Jeżeli istnieje funkcja $V:D\rightarrow\mathbb{R}$, różniczkowalna w~sposób ciągły, która spełnia warunki:
\begin{enumerate}[(a)]
    \item $M \subset D$ jest ograniczonym, domkniętym zbiorem niezmienniczym układu (\ref{eq:sys_stac}),
    \item $\dot{V}(x) \leq 0$ dla każdego $x\in M$,
    \item $E = \left\{ x\in M:\dot{V}(x) = 0 \right\}$,
    \item $N$ jest największym zbiorem niezmienniczym zawartym w $E$ (to znaczy każdy zbiór niezmienniczy zawarty w $E$ jest zawarty w $N$),
\end{enumerate}
to każda trajektoria rozpoczynająca się w zbiorze $M$ zbiega do zbioru $N$ dla $t\rightarrow\infty$.

\textit{Dowód:}
wzdłuż dowolnej trajektorii $x(t)$, rozpoczynającej się w zbiorze $M$ funkcja $V(x(t))$ jest nierosnąca (z założenia (b)), ograniczona z dołu (jako funkcja ciągła na zwartym zbiorze $M$), ma więc skończoną granicę $v$ dla $t \rightarrow \infty$.

Niech $N_{x}$ będzie zbiorem granicznym trajektorii $x(t)$.
Oczywiście $N_{x} \subset M$. Zgodnie z definicją 16, dla dowolnego punktu $p\in N_{x}$ istnieje ciąg $x(t_{n}) \rightarrow p$.
Na mocy ciągłości funkcji $V: V(p) = v$, czyli na całym zbiorze $N_{x}$, funkcja $V$ przyjmuje stałą wartość $v$, więc dla $p \in N_{x} : \dot{V}(x)=0$.
Ostatecznie
$$
    N_{x} \subset N \subset E \subset M.
$$
Jako że $x(t)$ zbiega do zbioru $N_{x}$, zbiega też do zbioru $N$.

\textbf{Twierdzenie 7 (wniosek z twierdzenia LaSalle'a)}
\begin{enumerate}
    \item Jeżeli istnieje funkcja $V:D\rightarrow\mathbb{R}$ różniczkowalna w sposób ciągły i spełnione są warunki:
    \begin{enumerate}[(a)]
        \item $V$ jest dodatnio określona na $D$,
        \item $\dot{V}(x) \leq 0$ w ograniczonym podzbiorze $S \subset D$, $0 \in S$,
        \item $\dot{V}(x)$ nie staje się zerem wzdłuż żadnej trajektorii $x(t) \subset S$, z wyjątkiem $x(t) \equiv 0$,
    \end{enumerate}
    to punkt równowagi $x_{e} = 0$ układu (\ref{eq:sys_stac}) jest asymptotycznie stabilny.

    \item Jeżeli założenia są spełnione w całej przestrzeni stanów ($S = \mathbb{R}^{n}$), a funkcja $V$ jest radialnie nieograniczona, to punkt równowagi $x_{e} = 0$ jest globalnie asymptotycznie stabilny.
\end{enumerate}

\textbf{Twierdzenie 8 (Barbaszina i Krasowskiego)}
\begin{enumerate}
    \item Jeżeli istnieje funkcja $V:D\rightarrow\mathbb{R}$ różniczkowalna w sposób ciągły i spełnione są warunki:
    \begin{enumerate}[(a)]
        \item $V$ jest dodatnio określona na $D$,
        \item $\dot{V}(x) \leq 0$ w ograniczonym podzbiorze $S \subset D$, $0 \in Int(S)$,
        \item $E = \left\{ x \in D: \dot{V}(x)=0 \right\}$ i żadna trajektoria $x(t)$ nie zawiera się w zbiorze $E$, z wyjątkiem $x(t) \equiv 0$,
    \end{enumerate}
    to punkt równowagi $x_{e} = 0$ układu (\ref{eq:sys_stac}) jest asymptotycznie stabilny.

    \item Jeżeli założenia są spełnione w całej przestrzeni stanów ($S = \mathbb{R}^{n}$), a funkcja $V$ jest radialnie nieograniczona, to punkt równowagi $x_{e} = 0$ jest globalnie asymptotycznie stabilny.
\end{enumerate}

\textbf{Twierdzenie 9 (o zbiorze niezmienniczym i zbiorze przyciągania)}
Jeżeli istnieje funkcja $V:D\rightarrow\mathbb{R}$ różniczkowalna w sposób ciągły i spełnione są warunki:
\begin{enumerate}[(a)]
    \item $M \subset D$ jest ograniczonym, domkniętym zbiorem niezmienniczym układu (\ref{eq:sys_stac}) i $0 \in M$,
    \item $\dot{V}(x) < 0$ dla $0 \neq x \in M$, $\dot{V}(0) = 0$,
\end{enumerate}
to zbiór $M$ jest zawarty w zbiorze przyciągania punktu równowagi $x_{e} = 0$.

\textbf{Twierdzenie 10 (o obszarze przyciągania)}
Jeżeli istnieje funkcja Lapunowa $V(x)$ oraz zachodzą warunki:
\begin{itemize}[(a)]
    \item istnieje zbiór $G = \left\{x:  V(x) < \rho \right\}$,
    \item dla każdego $x \in G$ zachodzi $\dot{V}(x) \leq 0$
\end{itemize}
to $G$ jest obszarem atrakcji, a trajektorie zbiegają do maksymalnego zbioru inwariantnego zawartego w $G$.


\begin{tcolorbox}[colback=pink!5, colframe=pink!75!black, title=Intuicja]
Z twierdzenia tego wynika, iż zbiór poziomicowy funkcji Lapunowa jest zbiorem niezmienniczym.
Ponieważ warunek $\dot{V}(x) = <\nabla V(x), f(x)> < 0$ oznacza, że kąt pomiędzy gradientem funkcji Lapunowa a trajektorią systemu należy do przedziału $(\frac{\pi}{2}, \frac{3\pi}{2})$, a więc są skierowane do wewnątrz zbioru.
Jak pokazano na rysunku \ref{fig:invariant_set}.
\begin{figure}[H]
    \centering
    \includegraphics[width=0.75\linewidth]{fig/03_lapunow/invariant_set.png}
    \caption{Geometryczna interpretacja warunku na inwariantność zbioru poziomicowego.}
    \label{fig:invariant_set}
\end{figure}
\end{tcolorbox}

\begin{tcolorbox}[colback=green!5, colframe=green!75!black, title=Przykład]
Rozważmy prosty układ pierwszego rzędu:
$$
    \dot{x} = -x + x^3.
$$
Graficznie przedstawiono to na rysunku (\ref{fig:la_calle_przyklad}):
\begin{figure}[H]
    \centering
    \includegraphics[width=0.25\linewidth]{fig/03_lapunow/la_salle_przyklad.PNG}
    \caption{Ilustracja do twierdzenia o obszarze przyciągania.}
    \label{fig:la_calle_przyklad}
\end{figure}
Dość łatwo zauważyć, że obszar pomiędzy $-1$ a $1$ jest stabilny - dla dodatnich położeń prędkość jest ujemna i ściąga stan do zera, dla ujemnych jest dodatnia i też przyciąga stan do zera.

Można rozważyć funkcję Lapunowa:
$$
    V(x) = \frac{1}{2}x^{2}.
$$
Jej pochodna systemowa:
$$
    \dot{V}(x) = -x^{2}+x^{4} = x^{2}\left(x^{2}-1\right) \implies \dot{V}(x) \leq 0 \, dla \, x \in(-1, 1).
$$
Zatem $x \in(-1, 1)$ jest obszarem atrakcji stabilnego punktu równowagi.
\end{tcolorbox}

\textbf{Twierdzenie 11 (Poincare)}
Jeżeli cykl graniczny istnieje, to musi okrążać co najmniej jeden punkt równowagi, a~do tego zachodzi równość:
\begin{equation}
    \label{eq:poincare}
    N = S + 1,
\end{equation}
gdzie:
\begin{itemize}
    \item $N$ - liczba punktów równowagi typu węzeł, ognisko, gwiazda,
    \item $S$ - liczba punktów równowagi typu siodło.
\end{itemize}

\begin{tcolorbox}[colback=pink!5, colframe=pink!75!black, title=Intuicja]
Każdy zamknięty kontur posiada tak zwany indeks, czyli liczbę obiegów wektora $\varphi(x_{1}, x_{2}) = \arctan{\left( \frac{\dot{x}_{1}}{\dot{x}_{2}} \right)}$.
Można pokazać, że dla konturu otaczającego pojedynczy punkt równowagi indeks ten wynosi 1 dla punktów równowagi typu węzeł, ognisko i gwiazda, natomiast dla punktu typu siodło indeks wynosi -1.
Dodatkowo można udowodnić, że indeks konturu w przypadku okrążania wielu punktów równowagi jest równy sumie indeksów wszystkich punktów równowagi w konturze.
Zatem jeśli ich suma wynosi 1 ($N-S = 1$), to  można je traktować poniekąd jak pojedynczy punkt równowagi - jeśli zawiera on stabilny cykl graniczny, to wszystkie trajektorie będą do niego zmierzać - jak do stabilnego punktu równowagi.
\end{tcolorbox}

\textbf{Twierdzenie 12 (Poincare-Bendixona)}
Jeśli trajektorie nieliniowego, stacjonarnego układu drugiego rzędu (\ref{eq:sys_stac}) pozostają w ograniczonym obszarze $\Omega$, to wówczas zachodzi jedna z możliwości:
\begin{enumerate}[(a)]
    \item trajektoria zbiega do punktu równowagi,
    \item zbiega do cyklu granicznego,
    \item jest cyklem granicznym.
\end{enumerate}
\begin{figure}[H]
    \centering
    \includegraphics[width=0.35\linewidth]{fig/03_lapunow/poincare_bendixon.PNG}
    \caption{Ilustracja do twierdzenia Poincare-Bendixona.}
    \label{fig:poincare_bendixon}
\end{figure}

\textbf{Twierdzenie 13 (Bendixona)}
Dla układu nieliniowego (\ref{eq:sys_stac}) drugiego rzędu, na obszarze $\Omega$ nie istnieje żaden cykl graniczny, jeśli $\frac{\partial f_{1}}{\partial x_{1}} + \frac{\partial f_{2}}{\partial x_{2}}$ nie zmienia na nim znaku ani nie zanika tożsamościowo do zera.

\begin{tcolorbox}[colback=green!5, colframe=green!75!black, title=Przykład]
Rozważmy układ:
\begin{equation}
    \label{eq:bendixon}
    \left\{
    \begin{array}{l}
         \dot{x}_{1} = x_{2} + x_{1}\left( 1 - x_{1}^{2} - x_{2}^{2} \right)\\
         \dot{x}_{1} = -x_{1} + x_{2}\left( 1 - x_{1}^{2} - x_{2}^{2} \right)
    \end{array}
    \right\}.
\end{equation}
Wówczas mamy pochodne cząstkowe:
$$
    \frac{\partial f_{1}}{\partial x_{1}} + \frac{\partial f_{2}}{\partial x_{1}} = 2 - 4(x_{1}^{2} + x_{2}^{2})
$$
suma ta zmienia znak na okręgu o środku w początku układu współrzędnych i promieniu $\frac{\sqrt{2}}{2}$.
Zatem można szukać cyklu granicznego wokół punktu $(0,0)$.
Portret fazowy tego układu pokazano na rysunku.
\begin{figure}[H]
    \centering
    \includegraphics[width=0.5\linewidth]{fig/03_lapunow/bendixon.png}
    \caption{Portret fazowy układu (\ref{eq:bendixon}).}
    \label{fig:bendixon}
\end{figure}
\end{tcolorbox}

    %=======================================================
    \subsection{Metody poszukiwania funkcji Lapunowa}\label{sec:dobor_funkcji}
    Jak wcześniej wspomniano nie istnieje uniwersalna metoda poszukiwania funkcji Lapunowa i odnalezienie jej jest zasadniczym problemem udowodnienia stabilności danego punktu równowagi systemu nieliniowego.
    Jednak w literaturze można odnaleźć szereg metod, które sugerują pewne obszary i metodologie poszukiwań.
        %-------------------------------------------------------
        \subsubsection{Energetyczna}
        W celu zobrazowania metody energetycznej funkcji Lapunowa wykorzystany zostanie przykład układu masa-sprężyna, jak na rysunku (\ref{fig:masa-sprezyna}).
        \begin{figure}[H]
            \centering
            \includegraphics[width=0.25\linewidth]{fig/03_lapunow/energia_lapunow.PNG}
            \caption{Układ masa-sprężyna.}
            \label{fig:masa-sprezyna}
        \end{figure}
        Wówczas można założyć, że siły tarcia i sprężystości wyrażają się zależnością:
        \begin{itemize}
            \item $F_{t} = f(x, \dot{x})$,
            \item $F_{s} = g(x)$.
        \end{itemize}
        Gdzie $f(x,\dot{x})$ i $g(x)$ są funkcjami ciągłymi spełniającymi warunki:
        \begin{itemize}
            \item $g(x) > 0$ dla $x \neq 0$,
            \item $f(x,\dot{x}) > 0$.
        \end{itemize}
        Ostatecznie więc dynamika układu wyraża się równaniem:
        \begin{equation}\label{eq:lienard}
            \ddot{x} + f(x,\dot{x})\dot{x} + g(x) = 0
        \end{equation}

        Ponieważ występuje tarcie, to suma energii kinetycznej i potencjalnej powinna maleć.
        Suma tych energii to:
        $$
            E = E_{k} + E_{p}
        $$
        gdzie:
        \begin{itemize}
            \item $E_{k} = \frac{\dot{x}^{2}}{2}$ - energia kinetyczna,
            \item $E_{p} = \int_{0}^{x_{1}}{g(z)dz}$ - energia potencjalna.
        \end{itemize}
        Zatem potencjalnej funkcji Lapunowa można poszukiwać jako sumy energii układu:
        \begin{equation}\label{eq:lap_en}
            V(x_{1},x_{2}) = \frac{\dot{x}^{2}}{2} + \int_{0}^{x_{1}}{g(z)dz}.
        \end{equation}

        %-------------------------------------------------------
        \subsubsection{Forma kwadratowa}
        Często można szukać funkcji Lapunowa w postaci formy kwadratowej w oparciu o przybliżenie liniowe układu nieliniowego:
        $$
            V(x) = x^{T}Px
        $$
        gdzie $P$ jest rozwiązaniem równania Lapunowa.
        
        Bardzo często szuka się też funkcji Lapunowa w postaci sumy kwadratów poszczególnych zmiennych stanu:
        $$
            V(x) = \frac{1}{2}\sum_{1=1}^{n}{x_{i}^{2}}.
        $$

        %-------------------------------------------------------
        \subsubsection{Metoda Krasowskiego}
        Jeśli dla układu (\ref{eq:sys_stac}) macierz Jacobiego:
        $$
            A(x) = \frac{\partial f(x)}{\partial x}
        $$
        spełnia warunek:
        $$
            F = A^{T} + A < 0.
        $$
        To funkcji Lapunowa można poszukiwać w formie:
        $$
            V(x) = f(x)^{T}Pf(x).
        $$
        Ponieważ:
        $$
            \dot{V}(x) = f^{T}\dot{f} + \dot{f}^{T}f = f^{T}Af + f^{T}A^{T}f = f^{T}Ff < 0.
        $$

        %-------------------------------------------------------
        \subsubsection{Metoda zmiennych gradientów}
        Metoda zmiennych gradientów jest formalnym podejściem do konstrukcji funkcji Lapunowa.
        Zakłada ona, że funkcja Lapunowa jest związana z jej nachyleniem $\nabla V$ przez zależność całkową:
        \begin{equation}
            \label{eq:mzg_0}
            V(x) = \int_{0}^{x}{\nabla Vdx}
        \end{equation}
        gdzie $\nabla V = \begin{bmatrix} \frac{\partial V}{\partial x_{1}} & \frac{\partial V}{\partial x_{2}} & \cdots & \frac{\partial V}{\partial x_{n}}  \end{bmatrix}^{T}$.
        Żeby uzyskać funkcję skalarną $V(x)$, gradient musi spełniać szereg warunków:
        \begin{equation}
            \label{eq:mzg_1}
            \frac{\partial \nabla V_{i}}{\partial x_{j}} = \frac{\partial \nabla V_{j}}{\partial x_{i}}, \, dla\,i,j=1, 2, ..., n
        \end{equation}
        gdzie $\nabla V_{i} = \frac{\partial V}{\partial x_{i}}$.
        Metoda ta zakłada postać gradientu funkcji Lapunowa w postaci:
        \begin{equation}
            \label{eq:mzg_2}
            \nabla V_{i} = \sum_{j=1}^{n}{a_{ij}x_{j}}
        \end{equation}
        gdzie $a_{ij}$ są określonymi współczynnikami.

        Procedurę te mozna zalgorytmizować w następujący sposób:
        \begin{enumerate}
            \item Zakładamy, że $\nabla V$ jest dane z zależności (\ref{eq:mzg_2}).
            \item Rozwiązujemy równanie dla współczynników $a_{ij}$ tak, aby spełniały warunki (\ref{eq:mzg_1}).
            \item Obliczamy $V$ przez całkowanie:
            $$
                V(x) = \int_{0}^{x_{1}}{\nabla V_{1}(x_{1}, 0,\cdots, 0)} + \int_{0}^{x_{2}}{\nabla V_{1}(x_{1}, x_{2}, 0 \cdots, 0)} + \cdots + \int_{0}^{x_{n}}{\nabla V_{1}(x_{1}, \cdots, x_{n})}.
            $$
            \item Sprawdzamy, czy $V$ jest dodatnio określona.
        \end{enumerate}

        \begin{tcolorbox}[colback=green!5, colframe=green!75!black, title=Przykład]
        Dla zobrazowania tej metody rozważmy układ:
        $$
            \left\{
            \begin{array}{l}
                 \dot{x}_{1} = -2x_{1}\\
                 \dot{x}_{1} = -2x_{2} + 2x_{1}x_{2}^{2}
            \end{array}
            \right\}.
        $$
        Przyjmujemy, że gradienty funkcji Lapunowa mają postać:
        $$
            \begin{array}{l}
                 \nabla V_{1} = a_{11}x_{1} + a_{12}x_{2}\\
                 \nabla V_{2} = a_{21}x_{1} + a_{22}x_{2}
            \end{array}
        $$
        Muszą one spełniać warunek:
        $$
            \frac{\partial \nabla V_{1}}{\partial x_{2}} = \frac{\partial \nabla V_{2}}{\partial x_{1}}
        $$
        czyli:
        $$
            x_{1}\frac{\partial a_{11}}{\partial x_{2}} + a_{12} = a_{21} + x_{2}\frac{\partial a_{22}}{\partial x_{1}}.
        $$
        Współczynniki $a_{ij}$ można dobrać np. jako:
        $$
            a_{11} = a_{22} = 1,\,a_{12}=a_{21}=0
        $$
        Mamy wówczas:
        $$
            \nabla V_{1} = x_{1},\,\nabla V_{2} = x_{2}.
        $$
        A funkcja Lapunowa ma postać:
        $$
            V(x) = \frac{1}{2}(x_{1}^{2} + x_{2}^{2}).
        $$
        \end{tcolorbox}

        %-------------------------------------------------------
        \subsubsection{Neuralna}
        Coraz częściej do poszukiwania funkcji Lapunowa wykorzystuje się metody uczenia maszynowego.
        Można w tej kategorii wydzielić najogólniej dwie grupy metod:
        \begin{itemize}
            \item aproksymujące funkcję Lapunowa w konkretnych punktach, zakładając odpowiednią gładkość aproksymowanej funkcji,
            \item wyznaczające funkcję Lapunowa i jej pochodną w postaci analitycznej - w tym podejściu ujemność pochodnej udowadniana jest za pomocą tzw. Satisfiability Modulo Theories - takie podejście pokazane jest między innymi w pracy \cite{Chang2019}. 
        \end{itemize}        

    %===========================================================
    \subsection{Przykłady}
        %-------------------------------------------------------
        \subsubsection{Przykład 1}
        Zbadać stabilność punktów równowagi układu równań różniczkowych:
        \begin{equation}\label{eq:example_11_sys_transformed}
            \left\{
            \begin{array}{l}
                 \dot{x}_{1} = x_{1}^{2} - x_{2}\\
                 \dot{x}_{2} = x_{1} - x_{2}
            \end{array}
            \right\}.
        \end{equation}
        W celu wyznaczenia punktów równowagi przyrównujemy prawą stronę równania (\ref{eq:example_11_sys_transformed}) do zera i otrzymujemy dwa punkty równowagi:
        \begin{enumerate}
            \item $x_{e}^{1} = \begin{bmatrix} 0 & 0 \end{bmatrix}^{T}$:
            Badany układ można zapisać w postaci (\ref{eq:sys_slabo_niel}) z \textbf{twierdzenia 2}:
            $$
                \begin{bmatrix} \dot{x}_{1} \\ \dot{x}_{2} \end{bmatrix}
                =
                \begin{bmatrix} 0 & -1 \\ 1 & -1 \end{bmatrix}
                \begin{bmatrix} x_{1} \\ x_{2} \end{bmatrix}
                +
                \begin{bmatrix} x_{1}^{2} \\ 0 \end{bmatrix}
            $$
            gdzie:
            \begin{itemize}
                \item $A = \begin{bmatrix} 0 & -1 \\ 1 & -1 \end{bmatrix}$,
                \item $r(x) = \begin{bmatrix} x_{1}^{2} & 0 \end{bmatrix}^{T}$ i $\lim_{\lVert x \rVert \rightarrow 0}\frac{\lVert r(x) \rVert}{\lVert x \rVert} = \lim_{\lVert x \rVert \rightarrow 0}\frac{x_{1}^{2}}{\lVert x \rVert} = 0$.
            \end{itemize}
            Badamy zatem wartości własne macierzy $A$:
            $$
                \begin{vmatrix} \lambda & 1\\ -1 & \lambda + 1 \end{vmatrix} = \lambda^{2} + \lambda + 1 = 0.
            $$
            Pierwiastki równania mają części rzeczywiste ujemne, zatem $x_{e}^{1}$ jest stabilnym punktem równowagi.
            
            \item $x_{e}^{2} = \begin{bmatrix} 1 & 1 \end{bmatrix}^{T}$: Aby móc zastosować pośrednią metodę Lapunowa przesuwamy punkt równowagi do początku układu współrzędnych: $z_{1} = x_{1} - 1$, $z_{2} = x_{2} - 1$. Wówczas układ (\ref{eq:example_11_sys_transformed}) przybiera formę:
            $$
                \left\{
                \begin{array}{l}
                     \dot{z}_{1} = z_{1}^{2} + 2z_{1} - z_{2}\\
                     \dot{z}_{2} = z_{1} - z_{2}
                \end{array}
                \right\}.
            $$
            Więc badany układ można zapisać w postaci (\ref{eq:sys_slabo_niel}) z \textbf{twierdzenia 2}:
            $$
                \begin{bmatrix} \dot{z}_{1} \\ \dot{z}_{2} \end{bmatrix}
                =
                \begin{bmatrix} 2 & -1 \\ 1 & -1 \end{bmatrix}
                \begin{bmatrix} z_{1} \\ z_{2} \end{bmatrix}
                +
                \begin{bmatrix} z_{1}^{2} \\ 0 \end{bmatrix}
            $$
            gdzie:
            \begin{itemize}
                \item $A = \begin{bmatrix} 2 & -1 \\ 1 & -1 \end{bmatrix}$,
                \item $r(z) = \begin{bmatrix} z_{1}^{2} & 0 \end{bmatrix}^{T}$ i $\lim_{\lVert z \rVert \rightarrow 0}\frac{\lVert r(z) \rVert}{\lVert z \rVert} = \lim_{\lVert z \rVert \rightarrow 0}\frac{z_{1}^{2}}{\lVert z \rVert} = 0$.
            \end{itemize}
            Badamy zatem wartości własne macierzy $A$:
            $$
                \begin{vmatrix} \lambda-2 & 1\\ -1 & \lambda + 1 \end{vmatrix} = \lambda^{2} - \lambda + 1 = 0.
            $$
            Pierwiastki równania mają części rzeczywiste dodatnie, zatem $x_{e}^{2}$ jest niestabilnym punktem równowagi.
        \end{enumerate}
        Powyższe wnioski zobrazowane są na rysunku (\ref{fig:przyklad_1}), przedstawiającym portret trajektorii badanego układu.
        \begin{figure}[H]
            \centering
            \includegraphics[width=0.7\linewidth]{fig/03_lapunow/przyklad_2.png}
            \caption{Portret trajektorii stanu układu (\ref{eq:example_11_sys_transformed}).}
            \label{fig:przyklad_1}
        \end{figure}

        %-------------------------------------------------------
        \subsubsection{Przykład 2}
        Zbadać stabilność punktów równowagi równania:
        \begin{equation}\label{eq:example_12_sys}
            \ddot{x} + \dot{x} + (x+1)x(x-1) = 0.
        \end{equation}
        W celu analizy stabilności przekształcamy równanie do układu równań przez wprowadzenie zmiennych:
        $$
            x_{1} = x, \, x_{2} = \dot{x} = \dot{x}_{1}.
        $$
        Mamy więc:
        \begin{equation}\label{eq:przyklad_2}
            \left\{
            \begin{array}{l}
                \dot{x}_{1} = x_{2}\\
                \dot{x}_{2} = -(x_{1} + 1)x_{1}(x_{1} - 1)- x_{2}
            \end{array}
            \right\}.
        \end{equation}
        Punkty równowagi szukamy przyrównując prawą stronę układu równań (\ref{eq:przyklad_2}) do zera:
        \begin{enumerate}
            \item $x_{e}^{1} = \begin{bmatrix} 0 & 0 \end{bmatrix}^{T}$:
            Przybliżenie liniowe można uzyskać poprzez przybliżenie szeregiem Taylora:
            $$
                \begin{bmatrix} \dot{x}_{1} \\ \dot{x}_{2} \end{bmatrix}
                =
                \frac{\partial f}{\partial x}|_{x_{e}^{1}} + r(x)
                =
                \begin{bmatrix} 0 & 1 \\ 1 & -1 \end{bmatrix}
                \begin{bmatrix} x_{1} \\ x_{2} \end{bmatrix}
                +
                \begin{bmatrix} 0 \\ -x_{1}^{3} \end{bmatrix}
            $$
            gdzie:
            \begin{itemize}
                \item $A = \begin{bmatrix} 0 & 1 \\ 1 & -1 \end{bmatrix}$,
                \item $r(x) = \begin{bmatrix} 0 & x_{1}^{3} \end{bmatrix}^{T}$ i $\lim_{\lVert x \rVert \rightarrow 0}\frac{\lVert r(x) \rVert}{\lVert x \rVert} = \lim_{\lVert x \rVert \rightarrow 0}\frac{|x_{1}^{3}|}{\lVert x \rVert} = 0$.
            \end{itemize}
            Badamy zatem wartości własne macierzy $A$:
            $$
                \begin{vmatrix} \lambda & -1\\ 1 & \lambda + 1 \end{vmatrix} = \lambda^{2} + \lambda - 1 = 0
            $$
            jeden pierwiastek jest dodatni, a drugi ujemny, zatem $x_{e}^{2}$ jest niestabilnym punktem równowagi typu siodło.
            
            \item $x_{e}^{2} = \begin{bmatrix} 1 & 0 \end{bmatrix}^{T}$:
            Aby móc zastosować pośrednią metodę Lapunowa przesuwamy punkt równowagi do początku układu współrzędnych: $z_{1} = x_{1} - 1$, $z_{2} = x_{2}$. Wówczas układ (\ref{eq:example_11_sys_transformed}) przybiera formę:
            $$
                \left\{
                \begin{array}{l}
                     \dot{z}_{1} = z_{2}\\
                     \dot{z}_{2} = -z_{2} - z_{1}^{3} - 3z_{1}^{2} - 2z_{1}
                \end{array}
                \right\}.
            $$
            Więc badany układ można zapisać w postaci (\ref{eq:sys_slabo_niel}) z \textbf{twierdzenia 2}:
            $$
                \begin{bmatrix} \dot{z}_{1} \\ \dot{z}_{2} \end{bmatrix}
                =
                \begin{bmatrix} 0 & 1 \\ -2 & -1 \end{bmatrix}
                \begin{bmatrix} z_{1} \\ z_{2} \end{bmatrix}
                +
                \begin{bmatrix} 0 \\ -3z_{1}^{2} - z_{1}^{3} \end{bmatrix}
            $$
            gdzie:
            \begin{itemize}
                \item $A = \begin{bmatrix} 0 & 1 \\ -2 & -1 \end{bmatrix}$,
                \item $r(z) = \begin{bmatrix} 0 \\ -3z_{1}^{2} - z_{1}^{3} \end{bmatrix}$ i $\lim_{\lVert z \rVert \rightarrow 0}\frac{\lVert r(z) \rVert}{\lVert z \rVert} = 0$.
            \end{itemize}
            Badamy zatem wartości własne macierzy $A$:
            $$
                \begin{vmatrix} \lambda & -1\\ 2 & \lambda + 1 \end{vmatrix} = \lambda^{2} + \lambda + 2 = 0.
            $$
            Pierwiastki równania mają części rzeczywiste ujemne, zatem $x_{e}^{2}$ jest asymptotycznie stabilnym punktem równowagi.
            
            \item $x_{e}^{3} = \begin{bmatrix} -1 & 0 \end{bmatrix}^{T}$:
            Aby móc zastosować pośrednią metodę Lapunowa przesuwamy punkt równowagi do początku układu współrzędnych: $z_{1} = x_{1} + 1$, $z_{2} = x_{2}$. Wówczas układ (\ref{eq:example_11_sys_transformed}) przybiera formę:
            $$
                \left\{
                \begin{array}{l}
                     \dot{z}_{1} = z_{2}\\
                     \dot{z}_{2} = -z_{2} - z_{1}^{3} - 3z_{1}^{2} - 2z_{1}
                \end{array}
                \right\}.
            $$
            Więc badany układ można zapisać w postaci (\ref{eq:sys_slabo_niel}) z \textbf{twierdzenia 2}:
            $$
                \begin{bmatrix} \dot{z}_{1} \\ \dot{z}_{2} \end{bmatrix}
                =
                \begin{bmatrix} 0 & 1 \\ -2 & -1 \end{bmatrix}
                \begin{bmatrix} z_{1} \\ z_{2} \end{bmatrix}
                +
                \begin{bmatrix} 0 \\ -3z_{1}^{2} - z_{1}^{3} \end{bmatrix}
            $$
            gdzie:
            \begin{itemize}
                \item $A = \begin{bmatrix} 0 & 1 \\ -2 & -1 \end{bmatrix}$,
                \item $r(z) = \begin{bmatrix} 0 \\ -3z_{1}^{2} - z_{1}^{3} \end{bmatrix}$ i $\lim_{\lVert z \rVert \rightarrow 0}\frac{\lVert r(z) \rVert}{\lVert z \rVert} = 0$.
            \end{itemize}
            Badamy zatem wartości własne macierzy $A$:
            $$
                \begin{vmatrix} \lambda & -1\\ 2 & \lambda + 1 \end{vmatrix} = \lambda^{2} + \lambda + 2 = 0.
            $$
            Pierwiastki równania mają części rzeczywiste ujemne, zatem $x_{e}^{3}$ jest asymptotycznie stabilnym punktem równowagi.
        \end{enumerate}

        Portret fazowy przedstawiony na rysunku (\ref{fig:przyklad_2}) potwierdza powyższe wnioski.
        \begin{figure}[H]
            \centering
            \includegraphics[width=0.7\linewidth]{fig/03_lapunow/przyklad_1.png}
            \caption{Portret fazowy układu (\ref{eq:przyklad_2}).}
            \label{fig:przyklad_2}
        \end{figure}

        %-------------------------------------------------------
        \subsubsection{Przykład 3}
        Określić obszary stabilności asymptotycznej wokół stabilnych asymptotycznie punktów równowagi równania:
        \begin{equation}\label{eq:example_1_sys}
            \ddot{x} + \dot{x} + (x+1)x(x-1) = 0.
        \end{equation}
    
        W pierwszej kolejności wprowadzamy zmienne $x_{1} = x$ i $x_{2} = \dot{x}$ i przekształcamy równanie do postaci:
        \begin{equation}\label{eq:example_1_sys_transformed}
            \left\{
            \begin{array}{l}
                 \dot{x}_{1} = x_{2}\\
                 \dot{x}_{2} = -(x_{1}+1)x_{1}(x_{1}-1) - x_{2}
            \end{array}
            \right\}.
        \end{equation}
        Jak pokazano w poprzednim przykładzie układ ten ma dwa stabilne punkty równowagi: $x_{e}^{1} = (-1, 0)$ i $x_{e}^{3} = (1, 0)$.
        Będziemy rozważać każdy z nich osobno.

        \begin{enumerate}
            \item $x_{e}^{1}$: wprowadzamy nowe zmienne, aby rozważać środek układu współrzędnych jako punkt równowagi:
            $$
                z_{1} = x_{1} + 1,\, z_{2} = x_{2}.
            $$
            Wówczas uzyskujemy równania:
            $$
                \left\{
                \begin{array}{l}
                     \dot{z}_{1} = z_{2}\\
                     \dot{z}_{2} = z_{2} - z_{1}(z_{1} - 1) (z_{1} - 2)
                \end{array}
                \right\}.
            $$
            Można przyjąć energetyczną funkcję Lapunowa:
            $$
                V(z) = \frac{z_{2}^{2}}{2} + \int_{0}^{z_{1}}{\gamma(\gamma - 1)(\gamma -2) d\gamma} = \frac{z_{2}^{2}}{2} + \frac{z_{1}^{4}}{4} - z_{1}^{3} + z_{1}^{2}.
            $$
            \begin{figure}[H]
                \centering
                \includegraphics[width=0.7\linewidth]{fig/03_lapunow/przyklad_3_v.png}
                \caption{Poziomice funkcji Lapunowa nałożone na portret fazowy układu z przykładu 3.}
                \label{fig:przyklad_3_v}
            \end{figure}
            Jej pochodna systemowa wynosi:
            $$
                V(z) = -z_{2}^{2}.
            $$
            \begin{figure}[H]
                \centering
                \includegraphics[width=0.7\linewidth]{fig/03_lapunow/przyklad_3_vp.png}
                \caption{Poziomice pochodnej systemowej funkcji Lapunowa nałożone na portret fazowy układu z przykładu 3.}
                \label{fig:przyklad_3_v}
            \end{figure}
            Jest ona ujemna wszędzie poza osią $OX$, a jedynymi zbiorami inwariantnymi na niej są punkty równowagi.
            Zatem poziomica ograniczająca obszar atrakcji "ociera się" o niestabilny punkt równowagi.
            Mamy więc:
            $$
                D^{1} = \left\{ x \in \mathbb{R}^{2} : V(x_{1} + 1, x_{2}) < \frac{1}{4}\, i\, x_{1} < 0 \right\}.
            $$
            \item $x_{e}^{3}$: analogicznie postępujemy dla drugiego stabilnego punktu równowagi.
            Wprowadzamy nowe zmienne:
            $$
                z_{1} = x_{1} - 1,\, z_{2} = x_{2}.
            $$
            Wówczas uzyskujemy równania:
            $$
                \left\{
                \begin{array}{l}
                     \dot{z}_{1} = z_{2}\\
                     \dot{z}_{2} = z_{2} - z_{1}(z_{1} + 1) (z_{1} + 2)
                \end{array}
                \right\}.
            $$
            Można przyjąć energetyczną funkcję Lapunowa:
            $$
                V(z) = \frac{z_{2}^{2}}{2} + \int_{0}^{z_{1}}{\gamma(\gamma + 1)(\gamma +2) d\gamma} = \frac{z_{2}^{2}}{2} + \frac{z_{1}^{4}}{4} + z_{1}^{3} + z_{1}^{2}.
            $$
            Jej pochodna systemowa wynosi:
            $$
                V(z) = -z_{2}^{2}.
            $$
            Zatem obszar atrakcji jest symetryczny do poprzedniego względem osi $OY$.
        \end{enumerate}
        \begin{figure}[H]
            \centering
            \includegraphics[width=0.7\linewidth]{fig/03_lapunow/przyklad_3_obszar.png}
            \caption{Granice estymowanego obszaru atrakcji dla stabilnych punktów równowagi z przykładu 3.}
            \label{fig:przyklad_3_obszar}
        \end{figure}
        
        %-------------------------------------------------------
        \subsubsection{Przykład 4}
        Rozpatrzmy raz jeszcze układ z \textbf{przykładu 1}:
        \begin{equation}\label{eq:example_2_sys}
            \left\{
            \begin{array}{l}
                 \dot{x}_{1} = x_{1}^{2} - x_{2}\\
                 \dot{x}_{2} = x_{1} - x_{2}
            \end{array}
            \right\}.
        \end{equation}
        Pokazano tam, że punkt równowagi $x_{e}^{1} = \begin{bmatrix} 0 & 0 \end{bmatrix}^{T}$ jest asymptotycznie stabilny.
        Oszacować jego obszar atrakcji.


        Rozważmy dwie funkcje Lapunowa:
        \begin{enumerate}
            \item Opartą na przybliżeniu liniowym i rozwiązaniu macierzowego równania Lapunowa:
            $$
                \dot{x}
                =
                \begin{bmatrix}
                    0 & -1\\
                    1 & -1
                \end{bmatrix}
                x
                +
                \begin{bmatrix}
                    x_{1}^{2}\\
                    0
                \end{bmatrix}.
            $$
            Przyjmując macierz $Q$ w postaci:
            $$
                Q
                =
                q
                \begin{bmatrix}
                    1 & 0\\
                    0 & 1
                \end{bmatrix},\,q>0.
            $$
            Otrzymuje się macierz $P$:
            $$
                P
                =
                \frac{1}{2}
                q
                \begin{bmatrix}
                    3 & -1\\
                    -1 & 2
                \end{bmatrix}.
            $$
            \begin{figure}[H]
                \centering
                \includegraphics[width=0.7\linewidth]{fig/03_lapunow/przyklad_4_v_1.png}
                \caption{Poziomice funkcji Lapunowa na tle portretu fazowago z przykładu 4.}
                \label{fig:przyklad_3_obszar}
            \end{figure}
            Jej pochodna systemowa ma postać:
            $$
                \dot{V}(x)
                =
                \frac{1}{2}q
                x
                \begin{bmatrix}
                    6x_{1}-2 & -x_{1}\\
                    -x_{1} & -2
                \end{bmatrix}
                x.
            $$
            Można zauważyć, że jej znak nie zależy od wartości parametru $q$ oraz dla $x_{1} \in (12.36; 0.66)$.
            \begin{figure}[H]
                \centering
                \includegraphics[width=0.7\linewidth]{fig/03_lapunow/przyklad_4_vp_1.png}
                \caption{Poziomice pochodnej systemowej funkcji Lapunowa na tle portretu fazowago z przykładu 4.}
                \label{fig:przyklad_3_obszar}
            \end{figure}
            Na tej podstawie wybieramy największy zbiór poziomicowy $V(x)$ spełniający te warunki.
            $$
                D = \left\{ x \in \mathbb{R}^{2}: V(x) < \frac{4}{9} \right\}.
            $$
            Zbiór ten przedstawiono na rysunku (\ref{fig:przyklad_4_obszar}).
            \begin{figure}[H]
                \centering
                \includegraphics[width=0.7\linewidth]{fig/03_lapunow/przyklad_4_obszar_1.png}
                \caption{Estymata obszaru atrakcji na tle portretu fazowago z przykładu 4.}
                \label{fig:przyklad_4_obszar}
            \end{figure}
            
            \item Funkcję Lapunowa w postaci:
            \begin{equation}
                \label{eq:przyklad_4_2}
                V(x) = (-x_{1} + x_{2})^{2} + 2 \int_{0}^{x^{1}}{(-z^{2} + z)dz} = -\frac{2}{3}x_{1}^{3}+2x_{1}^{2}-2x_{1}x_{2}+x_{2}^{2}.
            \end{equation}
            Jest ona dodatnia na pewnym otczeniu stabilnego punktu równowagi.
            \begin{figure}[H]
                \centering
                \includegraphics[width=0.7\linewidth]{fig/03_lapunow/przyklad_4_v_2.png}
                \caption{Poziomice funkcji Lapunowa (\ref{eq:przyklad_4_2}) na tle portretu fazowago z przykładu 4.}
                \label{fig:przyklad_4_obszar}
            \end{figure}
            Jej pochodna systemowa wynosi:
            $$
                \dot{V}(x) = -2(x_{1}-1)^{2}x_{1}^{2}.
            $$
            \begin{figure}[H]
                \centering
                \includegraphics[width=0.7\linewidth]{fig/03_lapunow/przyklad_4_vp_2.png}
                \caption{Poziomice pochodnej systemowej funkcji Lapunowa (\ref{eq:przyklad_4_2}) na tle portretu fazowego z przykładu 4.}
                \label{fig:przyklad_4_obszar}
            \end{figure}
            Jest ona ujemnie określona wszędzie poza prostymi $x=0$ i $x=1$. Badając te dwie proste można łatwo zauważyć, że tylko znajdujące się na nich punkty równowagi są zbiorami inwariantnymi.
            Zatem dobierając zbiór poziomicowy w ten sposób, by "zahaczał" tylko o niestabilny punkt równowagi, uzyskujemy estymatę obszaru asymptotycznego przyciągania stabilnego punktu równowagi.
            Zbiór ten został przedstawiony na rysunku (\ref{fig:przyklad_4_obszar_2}).
            \begin{figure}[H]
                \centering
                \includegraphics[width=0.7\linewidth]{fig/03_lapunow/przyklad_4_obszar_2.png}
                \caption{Estymata obszaru atrakcji na podstawie funkcji Lapunowa (\ref{eq:przyklad_4_2}) na tle portretu fazowego z przykładu 4.}
                \label{fig:przyklad_4_obszar_2}
            \end{figure}
            
        \end{enumerate}

        %-------------------------------------------------------
        \subsubsection{Przykład 5}
        Znaleźć obszar stabilności asymptotycznej wokół punktu równowagi $x_{1}=x_{2}=0$ dla układu równań:
        \begin{equation}\label{eq:example_2_sys}
            \left\{
            \begin{array}{l}
                 \dot{x}_{1} = -x_{1} + 2x_{1}^{3}x_{2}\\
                 \dot{x}_{2} = -x_{2}
            \end{array}
            \right\}.
        \end{equation}

        Również w tym przypadku, dla zobrazowania jak dobór funkcji Lapunowa wpływa na oszacowanie obszaru atrakcji, rozważymy dwie funkcje $V(x)$.
        Obydwie będą uzyskane na podstawie przybliżenia liniowego układu i równania Lapunowa (\ref{eq:rownanie_lapunowa}), lecz z innym doborem macierzy $Q$.
        Zatem funkcje Lapunowa będą mieć postać:
        $$
            V(x) = x^{T}Px.
        $$

        Przed rozpoczęciem poszukiwania funkcji Lapunowa wyznaczymy przybliżenie liniowe systemu (\ref{eq:example_2_sys}):
        $$
            \begin{bmatrix} \dot{x}_{1} \\ \dot{x}_{2} \end{bmatrix}
            =
            \begin{bmatrix} -1 & 0 \\ 0 & -1 \end{bmatrix}
            \begin{bmatrix} x_{1} \\ x_{2} \end{bmatrix}.
        $$

        \begin{enumerate}
            \item W pierwszym przypadku macierz $Q$ dobrana do równania Lapunowa miała postać (\ref{eq:przyklad_5_q_case_1}):
            \begin{equation}\label{eq:przyklad_5_q_case_1}
                Q = q\begin{bmatrix} 1 & 0\\ 0 & 1\end{bmatrix},\,q>0.
            \end{equation}
            Wówczas macierz $P$ wygląda następująco:
            $$
                P = \frac{q}{2}\begin{bmatrix} 1 & 0 \\ 0 & 1 \end{bmatrix}.
            $$
            Wykres poziomicowy funkcji Lapunowa nałożony na portret fazowy pokazano na rysunku (\ref{fig:przyklad_5_v_1}).

            \begin{figure}[H]
                \centering
                \includegraphics[width=0.7\linewidth]{fig/03_lapunow/przyklad_5_v_1.png}
                \caption{Portret fazowy układu (\ref{eq:example_2_sys}) z naniesionymi poziomicami funkcji Lapunowa.}
                \label{fig:przyklad_5_v_1}
            \end{figure}

            Dla uproszczenia zapisu można układ (\ref{eq:example_2_sys})  w postaci pseudoliniowej (\ref{eq:przyklad_5_form_pseudo}):
            \begin{equation}\label{eq:przyklad_5_form_pseudo}
                \begin{bmatrix} \dot{x}_{1} \\ \dot{x}_{2} \end{bmatrix}
                =
                \begin{bmatrix} -1 + 2x_{1}^{2}x_{2} & 0 \\ 0 & -1 \end{bmatrix}
                \begin{bmatrix} x_{1} \\ x_{2} \end{bmatrix}
                =
                \hat{A}x.
            \end{equation}
            Pochodną systemową można wyznaczyć następująco:
            $$
                \dot{V}(x) = \dot{x}^{T}Px + x^{T}P\dot{x}
                =
                (\hat{A}x)^{T}Px + x^{T}P\hat{A}x
                = x^{T}\left( \hat{A}P + P\hat{A} \right)x
                = x^{T}2P\hat{A}x
                = x^{T}Cx
            $$
            gdzie:
            $$
                C
                =
                q\begin{bmatrix} -1 + 2x_{1}^{2}x_{2} & 0 \\ 0 & -1 \end{bmatrix}.
            $$
            Łatwo zauważyć, że aby macierz $C$ była ściśle ujemnie określona wystarczy, żeby spełniona była nierówność:
            \begin{equation}\label{eq:przyklad_5_negative_vp}
                2x_{1}x_{2} - 1 < 0.
            \end{equation}
            Warunek ten jest niezależny od parametru $q$, ponadto zauważyć można, że kształt poziomic funkcji $V(x)$ również nie zależy od tego parametru.
            %Zatem w dalszych rozważaniach można go pominąć.
            Kształt poziomic pochodnej systemowej przedstawiono na rysunku (\ref{fig:przyklad_5_vp_1})

            \begin{figure}[H]
                \centering
                \includegraphics[width=0.7\linewidth]{fig/03_lapunow/przyklad_5_vp_1.png}
                \caption{Portret fazowy okładu (\ref{eq:example_2_sys}) z naniesionymi poziomicami pochodnej systemowej funkcji Lapunowa.}
                \label{fig:przyklad_5_vp_1}
            \end{figure}

            Zbiór $S$, w którym pochodna systemowa jest ujemna, można ustalić jako:
            $$
                S = \left\{ (x_{1}, x_{2}):2x_{1}^{2}x_{2}-1 < 0 \right\}.
            $$
            Szukamy zatem największego zbioru inwariantnego $D$ w $S$ dla którego zachodzi:
            $$
                D = \left\{ (x_{1},x_{2}): V(x) < l_{max} \right\}.
            $$
            Taki obszar zaznaczono na rysunku (\ref{fig:Przyklad_5_obszar_1}).

            \begin{figure}[H]
                \centering
                \includegraphics[width=0.7\linewidth]{fig/03_lapunow/przyklad_5_obszar_1.png}
                \caption{Portret fazowy układu (\ref{eq:example_2_sys}) z obszarem atrakcji, wynikającym z funkcji Lapunowa.}
                \label{fig:Przyklad_5_obszar_1}
            \end{figure}

            \item Drugą rozpatrywaną formą macierzy $Q$ jest:
            $$
                Q = \begin{bmatrix} a & 0\\ 0 & 1\end{bmatrix}, \, a > 0.
            $$
            Z równania Lapunowa wynika następująca postać macierzy $P$:
            $$
                P = \begin{bmatrix} \frac{1}{2}a & 0\\ 0 & 1 \end{bmatrix}.
            $$
            Funkcja Lapunowa natomiast ma postać:
            $$
                V(x) = x^{T} P x = \frac{1}{2}ax_{1}^{2} + x_{2}^{2}.
            $$
            Można więc zauważyć, że jej poziomice będą tworzyły elipsy, gdzie dla $a>1$ dłuższa będzie oś wzdłuż osi $OY$, natomiast dla $a<1$ dłuższa będzie oś wzdłuż osi $OX$.
            
            Jej pochodna systemowa ma formę:
            $$
                \dot{V}(x) = x^{T}\left( \hat{A}^{T}P + P\hat{A} \right)x
                =
                x^{T}
                \begin{bmatrix} a(-1 + 2x_{1}^{2}x_{2}) & 0\\ 0 & -1 \end{bmatrix}
                x.
            $$
            Identycznie jak poprzednio widać, że zbiór $S$, w którym pochodna systemowa funkcji Lapunowa jest ujemna, jest identyczny i nie zależy od parametru $a$.

            Ponieważ dla różnych $a$ różne elipsy utworzą różne obszary atrakcji:
            $$
                D_{a} = \left\{ (x_{1},x_{2}): V(x) < l_{a,max} \right\}
            $$
            rozpatrując całą ich rodzinę - sumę tych obszarów, uzyskamy jeden większy obszar przyciągania asymptotycznego.
            Obszar ten zaznaczono na rysunku (\ref{fig:przyklad_5_obszar_2}).
            \begin{figure}[H]
                \centering
                \includegraphics[width=0.7\linewidth]{fig/03_lapunow/przyklad_5_obszar_2.png}
                \caption{Portret fazowy układu (\ref{eq:example_2_sys}) z obszarem atrakcji wynikającym z funkcji Lapunowa}
                \label{fig:przyklad_5_obszar_2}
            \end{figure}
            Poza obszarem atrakcji ograniczonym przez czerwone linie na rysunku (\ref{fig:przyklad_5_obszar_2}) pozostaje do rozważenia obszary:
            \begin{itemize}
                \item \textit{powyżej} - z warunku na ujemny znak pochodnej - równanie (\ref{eq:przyklad_5_negative_vp}) - jest to obszar niestabilny,
                \item \textit{poniżej} - pochodna systemowa funkcji $V(x)$ jest tutaj ujemna, można więc sprawdzić jakie zachowanie wynika z równania dynamiki układu. A mianowicie widać, że $\dot{x}_{2} = -x_{2}$, zatem wszystkie trajektorie się tam zaczynające schodzą w kierunku osi $OX$, zatem wpadają ostatecznie przez obszar ograniczony liniami czerwonymi. Wnioskujemy więc, że obszar ten jest obszarem przyciągania punktu równowagi.
            \end{itemize}   
        \end{enumerate}
    
        %-------------------------------------------------------
        \subsubsection{Przykład 6}
        Dla układu regulacji automatycznej, przedstawionego na rysunku (\ref{fig:example_4_system_schem}), określić obszar stabilności asymptotycznej wokół stabilnego asymptotycznie punktu równowagi.
        Nieliniowy element statyczny ma charakterystykę:
        $$
            g(e) = -e(e+1)(e-1).
        $$
        \begin{figure}[H]
            \centering
            \includegraphics[width=0.75\linewidth]{fig/03_lapunow/example_system.PNG}
            \caption{Schemat układu regulacji z przykładu 6.}
            \label{fig:example_4_system_schem}
        \end{figure}
        Na podstawie podanego schematu układu sterowania można wyznaczyć równanie różniczkowe:
        $$
            \ddot{x} + \dot{x} - (x+1)x(x-1) = 0.
        $$
        A dalej, wprowadzając zmienne $x_{1} = x$ i $x_{2} = \dot{x}$, mamy układ równań:
        \begin{equation} \label{eq:przyklad_6_rownanie}
            \left\{
            \begin{array}{l}
                 \dot{x}_{1} = x_{2}\\
                 \dot{x}_{2} = -x_{2} + (x_{1}+1)x_{1}(x_{1}-1)
            \end{array}
            \right\}.
        \end{equation}
        Powyższy układ ma 3 punkty równowagi: $x_{e}^{1} = \begin{bmatrix} -1 & 0\end{bmatrix}^{T}$, $x_{e}^{2} = \begin{bmatrix} 0 & 0\end{bmatrix}^{T}$, $x_{e}^{3} = \begin{bmatrix} 1 & 0\end{bmatrix}^{T}$.
        Na podstawie pośredniego kryterium Lapunowa łatwo stwierdzić, że tylko $x_{e}^{2}$ jest stabilnym punktem.
        Portret fazowy rozważanego układu pokazany jest na rysunku (\ref{fig:przyklad_6_pf}).
        \begin{figure}[H]
            \centering
            \includegraphics[width=0.7\linewidth]{fig/03_lapunow/przyklad_6.png}
            \caption{Portret fazowy układu z przykładu 6.}
            \label{fig:przyklad_6_pf}
        \end{figure}

        Aby znaleźć obszar stabilności rozważamy funkcję Lapunowa dla systemu.
        Podobnie jak poprzednio rozważane będą dwa przypadki.
        \begin{enumerate}
            \item Funkcja Lapunowa uzyskana na interpretacji energetycznej:
            \begin{equation}
                \label{eq:przyklad_6_f_lap_1}
                V(x) = \frac{x_{2}^{2}}{2} + \int_{0}^{x_{1}}{\left[-(z-1)z(z+1) \right]dz} = \frac{x_{2}^{2}}{2} - \frac{x_{1}^{4}}{4} + \frac{x_{1}^{2}}{2}.
            \end{equation}
            Poziomice tej funkcji nałożone na portret fazowy przedstawiono na rysunku:
            \begin{figure}[H]
                \centering
                \includegraphics[width=0.7\linewidth]{fig/03_lapunow/przyklad_6_v_1.png}
                \caption{Portret fazowy układu (\ref{eq:przyklad_6_rownanie}) z naniesionymi poziomicami funkcji Lapunowa (\ref{eq:przyklad_6_rownanie}).}
                \label{fig:przyklad_6_v_1}
            \end{figure}
            Jej pochodna systemowa ma postać:
            \begin{equation}
                \label{eq:przyklad_6_vp}
                \dot{V}(x) = -x_{1}^{2}.
            \end{equation}
            Funkcja ta jest ujemna wszędzie poza całą osią $OX$.
            Dlatego konieczne jest sprawdzenie istnienia zbiorów inwariantnych na niej - łatwo zauważyć, że jest to tylko punkt równowagi układu.
            \begin{figure}[H]
                \centering
                \includegraphics[width=0.7\linewidth]{fig/03_lapunow/przyklad_6_vp_1.png}
                \caption{Portret fazowy układu (\ref{eq:przyklad_6_rownanie}) z naniesionymi poziomicami pochodnej systemowej funkcji Lapunowa (\ref{eq:przyklad_6_rownanie}).}
                \label{fig:przyklad_6_v_1}
            \end{figure}
            Dla wszystkich punktów równowagi wartości $V(x)$ są dodatnie, a $\dot{V}(x)$ równe 0. Dlatego ograniczamy zakres poszukiwań obszaru atrakcji do tych poziomic funkcji Lapunowa, które sięgają niestabilnych punktów równowagi, zatem:
            $$
                D = \left\{ x \in \mathbb{R}^{2}: V(x) < \frac{1}{4}, |x_{1}| < 1 \right\}.
            $$
            \begin{figure}[H]
                \centering
                \includegraphics[width=0.7\linewidth]{fig/03_lapunow/przyklad_6_obszar_1.png}
                \caption{Portret fazowy układu (\ref{eq:przyklad_6_rownanie}) z naniesionym obszarem atrakcji stabilnego punktu równowagi (obszar zaznaczony czerwoną przerywaną linią).}
                \label{fig:przyklad_6_v_1}
            \end{figure}
            
            \item Funkcja Lapunowa uzyskana na podstawie przybliżenia liniowego.
        \end{enumerate}
    
        %-------------------------------------------------------
        \subsubsection{Przykład 7}
        Określić obszar atrakcji dla układu z przykładu 6, przy założeniu, że element nieliniowy ma charakterystykę:
        $$
            g(e) = e(e-1)^{2}.
        $$
        Analogicznie jak w \textbf{przykładzie 6} wyznaczamy równanie różniczkowe, opisujące układ:
        $$
            \ddot{x} + \dot{x} - x(x-1)^2 = 0
        $$
        oraz wprowadzamy zmienne $x_{1} = x$ i $x_{2} = \dot{x}$, uzyskując układ równań:
        \begin{equation} \label{eq:przyklad_7_rownanie}
            \left\{
            \begin{array}{l}
                 \dot{x}_{1} = x_{2}\\
                 \dot{x}_{2} = -x_{2} - x_{1}(x_{1}-1)^{2}
            \end{array}
            \right\}
        \end{equation}
        Z czego wynikają dwa punkty równowagi: $x_{e}^{1} = \begin{bmatrix} 1 & 0\end{bmatrix}^{T}$, $x_{e}^{2} = \begin{bmatrix} 0 & 0\end{bmatrix}^{T}$.
        Z czego na podstawie pośredniej metody Lapunowa wynika, iż tylko $x_{e}^{2}$ jest stabilny.
        Portret fazowy układu przedstawiono na rysunku (\ref{fig:przyklad_7_pf}).
        \begin{figure}[H]
            \centering
            \includegraphics[width=0.7\linewidth]{fig/03_lapunow/przyklad_7.png}
            \caption{Portret fazowy układu (\ref{eq:przyklad_7_rownanie}).}
            \label{fig:przyklad_7_pf}
        \end{figure}
        Również tym razem można przyjąć energetyczną funkcję Lapunowa:
        \begin{equation}
            \label{eq:przyklad_7_v}
            V(x) = \frac{x_{2}^2}{2} + \int_{0}^{x_{1}}{z(z-1)dz} = \frac{x_{2}^{2}}{2} + \frac{x_{1}^{4}}{4} - \frac{2x_{1}^{3}}{3} + \frac{x_{1}^{2}}{2}
        \end{equation}
        Jej wykres poziomicowy naniesiony na portret fazowy pokazano na rysunku (\ref{fig:przyklad_7_v_1}).
        \begin{figure}[H]
            \centering
            \includegraphics[width=0.7\linewidth]{fig/03_lapunow/przyklad_7_v_1.png}
            \caption{Portret fazowy układu (\ref{eq:przyklad_7_rownanie}) z naniesionymi poziomicami funkcji Lapunowa (\ref{eq:przyklad_7_v}).}
            \label{fig:przyklad_7_v_1}
        \end{figure}
        Jej pochodna systemowa ma postać:
        \begin{equation}
            \label{eq:przyklad_7_vp}
            \dot{V}(x) = -x_{2}^{2}.
        \end{equation}
        Jej wykres naniesiony na portret fazowy został przedstawiony na rysunku (\ref{fig:przyklad_7_v_1}).
        \begin{figure}[H]
            \centering
            \includegraphics[width=0.7\linewidth]{fig/03_lapunow/przyklad_7_vp_1.png}
            \caption{Portret fazowy układu (\ref{eq:przyklad_7_rownanie}) z naniesionymi poziomicami pochodnej systemowej funkcji Lapunowa (\ref{eq:przyklad_7_v}).}
            \label{fig:przyklad_7_v_1}
        \end{figure}
        Pochodna systemowa (\ref{eq:przyklad_7_vp}) zeruje się na osi $OX$, dlatego sprawdzamy, jak na niej zachowują się trajektorie systemu. Z tej analizy wynika, że zbiór inwariantny zawiera tylko punkty równowagi.
        
        Zatem obszar atrakcji jest ograniczony przez poziomicę funkcji Lapunowa, sięgającej do niestabilnego punktu równowagi. Otrzymujemy więc:
        $$
            D = \left\{ x \in \mathbb{R}^{2}: V(x) < \frac{1}{12} \right\}.
        $$
        Obszar ten został przedstawiony na rysunku (\ref{fig:przyklad_7_v_1}).
        \begin{figure}[H]
            \centering
            \includegraphics[width=0.7\linewidth]{fig/03_lapunow/przyklad_7_obszar_1.png}
            \caption{Portret fazowy układu (\ref{eq:przyklad_7_rownanie}) z naniesionym obszarem atrakcji stabilnego punktu równowagi (obszar zaznaczony czerwoną przerywaną linią).}
            \label{fig:przyklad_7_v_1}
        \end{figure}

        %-------------------------------------------------------
        \subsubsection{Przykład 8}
        Znaleźć cykl graniczny i zbadać jego stabilność dla układu:
        \begin{equation}\label{eq:example_8}
            \left\{
            \begin{array}{l}
                 \dot{x}_{1} = x_{2} + x_{1}\left( 1 - x_{1}^{2} -x_{2}^{2} \right)\\
                 \dot{x}_{2} = -x_{1} + x_{2}\left( 1 - x_{1}^{2} -x_{2}^{2} \right)
            \end{array}
            \right\}.
        \end{equation}
        Jest to ten sam układ, co w przykładzie dla zobrazowania kryterium Bendixona.
        Zatem wiemy, że istnieje cykl graniczny - portret fazowy pokazany na rysunku (\ref{fig:przyklad_8}).
        
        \begin{figure}[H]
            \centering
            \includegraphics[width=0.7\linewidth]{fig/03_lapunow/przyklad_8_pf.png}
            \caption{Portret fazowy układu z przykładu 8.}
            \label{fig:przyklad_8}
        \end{figure}

        Można zauważyć, że jeżeli $(x_{1}, x_{2})$ znajduje się na okręgu o promieniu 1, to układ redukuje się do układu liniowego z punktem równowagi typu centrum.
        Zatem możemy szukać cyklu granicznego na tym właśnie okręgu.

        Aby to zrobić można sprawdzić, czy istnieje funkcja Lapunowa, która pokaże stabilność/niestabilność takiej trajektorii zamkniętej.
        Można zaproponować funkcję:
        \begin{equation}
            \label{eq:przyklad_8_v}
            V(x) = (1 - x_{1}^{2} -x_{2}^{2})^{2}.
        \end{equation}
        Jej kształt pokazano na rysunku (\ref{fig:przyklad_8_v}).

        \begin{figure}[H]
            \centering
            \includegraphics[width=0.7\linewidth]{fig/03_lapunow/przyklad_8_v.png}
            \caption{Funkcja Lapunowa (\ref{eq:przyklad_8_v}) na tle portretu fazowego układu z przykładu 8.}
            \label{fig:przyklad_8_v}
        \end{figure}

        Jej pochodna systemowa ma postać (rysunek (\ref{fig:przyklad_8_vp})):
        \begin{equation}
            \label{eq:przyklad_8_vp}
            V(x) = -2(x_{1}^{2} + x_{2}^{2})(1 - x_{1}^{2} -x_{2}^{2})^{2}.
        \end{equation}
        Jest ona ujemna wszędzie poza okręgiem jednostkowym o środku w początku układu współrzędnych i poza tym właśnie początkiem, gdzie ma wartość 0.
        Zatem na podstawie La Salle'a wszystkie trajektorie dążą do tego zbioru inwariantnego.
        Ponadto na podstawie tej samej funkcji Lapunowa i twierdzenia o niestabilności można sprawdzić, że początek układu współrzędnych jest niestabilnym punktem równowagi (sprawdzić, czemu nie na podstawie pośredniej metody Lapunowa).
        Ostatecznie mamy więc wniosek, że okrąg jednostkowy jest stabilnym cyklem granicznym.

        \begin{figure}[H]
            \centering
            \includegraphics[width=0.7\linewidth]{fig/03_lapunow/przyklad_8_vp.png}
            \caption{Poziomice pochodnej funkcji Lapunowa (\ref{eq:przyklad_8_v}) na tle portretu fazowego układu z przykładu 8.}
            \label{fig:przyklad_8_vp}
        \end{figure}

        %-------------------------------------------------------
        \subsubsection{Przykład 9}
        Rozważmy równanie:
        \begin{equation}
            \ddot{x} + a\dot{x} + 2bx + 3x^{2} = 0,\,a, b>0.
        \end{equation}
        Wprowadzamy zmienne:
        $$
            x_{1} = x,\,x_{2} = \dot{x} = \dot{x}_{1}.
        $$
        Zatem:
        $$
            \left\{
            \begin{array}{l}
                 \dot{x}_{1} = x_{2}\\
                 \dot{x}_{2} = -2bx_{1} - 3x_{1}^{2} - ax_{2}
            \end{array}
            \right\}.
        $$
        Układ ma dwa punkty równowagi, których stabilność można określić przy użyciu pośredniej metody Lapunowa:
        \begin{enumerate}
            \item $x_{e}^{1} = \left( 0, 0 \right)$: wówczas wielomian charakterystyczny ma postać:
            $$
                \begin{vmatrix}
                    \lambda & -1\\ 2b & \lambda + a
                \end{vmatrix}
                =
                \lambda^{2} + \lambda a + 2b.
            $$
            Z wzorów Viète’a otrzymuje się:
            $$
                \lambda_{1} + \lambda_{2} = -a,\, \lambda_{1}\lambda_{2}=2b.
            $$
            Oznacza to, że pierwiastki wielomianu charakterystycznego albo są rzeczywiste ujemne, albo sprężone z ujemnymi częściami rzeczywistymi.
            Czyli punkt równowagi jest punktem stabilnym typu węzeł stabilny lob ognisko stabilne.
            \item $x_{e}^{1} = \left( -\frac{2}{3}b, 0 \right)$: w tym przypadku wykonujemy zmianę zmiennych, aby rozpatrywać początek układu współrzędnych:
            $$
                z_{1} = x_{1} + \frac{2}{3}b,\,z_{2} = x_{2}.
            $$
            Mamy więc układ w postaci:
            $$
                \left\{
                \begin{array}{l}
                     \dot{z}_{1} = z_{2}\\
                     \dot{z}_{2} = 2bz_{1} - 3x_{1}^{2} - az_{2}
                \end{array}
                \right\}.
            $$
            Zatem w tym przypadku wielomian charakterystyczny części liniowej układu ma postać:
            $$
                \begin{vmatrix}
                    \lambda & -1\\ -2b & \lambda + a
                \end{vmatrix}
                =
                \lambda^{2} + \lambda a - 2b.
            $$
            Z wzorów Viète’a otrzymuje się:
            $$
                \lambda_{1}\lambda_{2}=-2b,
            $$
            co oznacza, że pierwiastki są przeciwnego znaku - czyli punkt równowagi jest niestabilny, typu siodło.
        \end{enumerate}
        Tylko pierwszy punkt równowagi jest stabilny i dla niego szukamy estymaty obszaru atrakcji.
        W tym celu sprawdzamy funkcję $V(x)$ w postaci:
    \begin{equation}
        \label{eq:przyklad_9_v}
        V(x) = \frac{1}{2}x_{2} + bx_{1}^{2} + x_{1}^{3}.
    \end{equation}
    Jej pochodna systemowa ma postać:
    \begin{equation}
        \label{eq:przyklad_9_vp}
        \dot{V}(x) = - ax_{2}^{2}.
    \end{equation}
    Pochodna ta jest ujemna wszędzie poza osią $OX$, jednak łatwo sprawdzić, że jedynymi zbiorami inwariantnymi na tej osi są punkty równowagi.
    Ograniczeniem więc estymaty zbioru inwariantnego będzie poziomica funkcji Lapunowa (\ref{eq:przyklad_9_v}), "zahaczająca" o niestabilny punkt równowagi:
    $$
        V(x) = \frac{1}{2}x_{2} + bx_{1}^{2} + x_{1}^{3} = \frac{4}{27}b^{3}.
    $$
    Rysunki (\ref{fig:przyklad_9_a_1_b_1}), (\ref{fig:przyklad_9_a_1_b_2}), (\ref{fig:przyklad_9_a_2_b_1}) pokazują dobór estymaty obszaru przyciągania dla kilku doborów parametrów.
    \begin{figure}[H]
        \centering
        \includegraphics[width=0.7\linewidth]{fig/03_lapunow/przyklad_9_obszara_1_b_1.png}
        \caption{Oszacowanie obszaru przyciągania punktu równowagi układu z przykładu 9 dla funkcji Lapunowa w postaci (\ref{eq:przyklad_9_v}). Dla parametrów $a = 1$ i $b = 1$.}
        \label{fig:przyklad_9_a_1_b_1}
    \end{figure}
    \begin{figure}[H]
        \centering
        \includegraphics[width=0.7\linewidth]{fig/03_lapunow/przyklad_9_obszara_1_b_2.png}
        \caption{Oszacowanie obszaru przyciągania punktu równowagi układu z przykładu 9 dla funkcji Lapunowa w postaci (\ref{eq:przyklad_9_v}). Dla parametrów $a = 1$ i $b = 2$.}
        \label{fig:przyklad_9_a_1_b_2}
    \end{figure}
    \begin{figure}[H]
        \centering
        \includegraphics[width=0.7\linewidth]{fig/03_lapunow/przyklad_9_obszara_2_b_1.png}
        \caption{Oszacowanie obszaru przyciągania punktu równowagi układu z przykładu 9 dla funkcji Lapunowa w postaci (\ref{eq:przyklad_9_v}). Dla parametrów $a = 2$ i $b = 1$.}
        \label{fig:przyklad_9_a_2_b_1}
    \end{figure}

%%%%%%%%%%%%%%%%%%%%%%%%%%%%%%%%%%%%%%%%%%%%%%%%%%%%%%%%%%%% 
\section{Jak wykorzystać to do projektowania sterowania?}
    %=======================================================
    \subsection{Sterująca Funkcja Lapunowa (Control Lyapunov Function - CLF)}
    W znacznej ilości przypadków spotykanych w praktyce, model dynamiki układ może być zapisany w tzw. postaci afinicznej ze względu na sterowanie, pokazanej wzorem (\ref{eq:affine}):
    \begin{equation} \label{eq:affine}
        \dot{x} = f(x) + g(x)u,
    \end{equation}
    gdzie: $x \in \mathbb{R}^{n}$, $u \in \mathbb{R}$ oraz $f(0) = 0$.
    
    Wówczas, aby punkt równowagi $x_{e} = 0$ był asymptotycznie stabilny, pochodna systemowa musi spełniać warunek:
    \begin{equation} \label{eq:affine_diff}
        \dot{V}(x) = \mathcal{L}_{f}V(x) + \mathcal{L}_{g}V(x)u < 0.
    \end{equation}
    
    Można zatem zauważyć, że można odnaleźć sterowanie stabilizujące punkt równowagi gdy zachodzi warunek: dla każdego $x \in \{ x \in \mathbb{R}^{n}: g(x) = 0 \land x \neq 0 \}$ musi zachodzić $\mathcal{L}_{f}V(x) < 0$.

    Wówczas sterowanie można odnaleźć np.:
    \begin{description}
        \item[Metodą Sonntaga]:
        Sterowanie można wyznaczyć za pomocą tzw. formuły Sonntaga:
        $$
            u(x)
            =
            \left\{
            \begin{array}{cl}
                 -\frac{\mathcal{L}_{f}V(x) + \sqrt{(\mathcal{L}_{f}V(x))^{2} + (\mathcal{L}_{g}V(x))^{4}}}{\mathcal{L}_{g}V(x)} &   \text{dla}\, \mathcal{L}_{g}V(x) \neq 0\\
                 0  &   \text{dla}\, \mathcal{L}_{g}V(x) = 0
            \end{array}
            \right\}.
        $$
        Po podstawieniu takiego sterowania do równania (\ref{eq:affine_diff}) przyjmuje ono postać:
        $$
            \dot{V}(x) = - \sqrt{(\mathcal{L}_{f}V(x))^{2} + (\mathcal{L}_{g}V(x))^{4}} < 0
        $$
        co zapewnia asymptotyczną stabilność punktu równowagi.
        Metodę tę można uogólnić dla $u \in \mathbb{R}^{m}$.

        \begin{tcolorbox}[colback=green!5, colframe=green!75!black, title=Przykład]
        Stabilizujemy układ w punkcie $x_{e} = 0$:
        $$
            \dot{x} = x - 2x^{3} + (x+1)u(x),
        $$
        dla którego zakładamy funkcję Lapunowa:
        $$
            V(x) = \frac{1}{2}x^{2}.
        $$
        Wówczas mamy:
        \begin{itemize}
            \item $f(x) = x - 2x^{3}$, $g(x) = x+1$
            \item $\mathcal{L}_{f}V(x) = x(x - 2x^{3})$,
            \item $\mathcal{L}_{g}V(x) = x(x+1)$, z tego wynika, że $\mathcal{L}_{g}V(x) = 0$ dla $x = 0$ i $x = -1$,
            \item $\mathcal{L}_{f}V(0) = 0$ oraz $\mathcal{L}_{f}V(-1) = -1 < 0$ - zatem spełnione są warunki metody Sonntaga.
        \end{itemize}
        Definiujemy sterowanie jako:
        $$
            u(x)
            =
            \left\{
            \begin{array}{cl}
                 -\frac{x^{2} - 2x^{4} + \sqrt{(x^{2} - 2x^{4})^{2} + (x^{2} + x)^{4}}}{x^{2} + x} &   \text{dla}\, x \neq 0 \, i\, x \neq -1 \\\
                 0  &   \text{inaczej}
            \end{array}
            \right\}.
        $$
        \end{tcolorbox}
    
        \item[Rozwiązywanie online problemu programowania kwadratowego]:
        Obecnie, gdy sterowniki dysponują mikroprocesorami o dużej mocy obliczeniowej, można na bieżąco wyznaczać sterowanie poprzez rozwiązywanie problemu programowania kwadratowego.
        Dla przykładu gdy chcemy zachować stabilność i minimalizować koszty sterowania odpowiada to następującemu problemowi:
        $$
            \begin{array}{c}
                 u = \underset{u}{\mathrm{argmin}} (u)^{2}}\\
                 \text{przy ograniczeniu:}\, \mathcal{L}_{f}V(x) + \mathcal{L}_{g}V(x)u < 0
            \end{array}.
        $$
        Dość łatwo zauważyć, że metoda ta jest bardzo elastyczna, można
        wprowadzić inny wskaźnik jakości oraz inne ograniczenia.
        

        \begin{tcolorbox}[colback=magenta!5, colframe=magenta!75!black, title=Zastosowanie w robotyce]
        Nowoczesne roboty, takie jak te z Boston Dynamics (np. Atlas),
        wykorzystują koncepcję Sterujących Funkcji Lapunowa (CLF) połączoną z~Programowaniem Kwadratowym (QP).

        \begin{description}
            \item[Na czym to polega?] Robot kroczący jest skrajnie niestabilny - wystarczy
            chwila, by upadł. Algorytm sterowania w~każdym ułamku sekundy (np. co 1
            ms) rozwiązuje zadanie optymalizacyjne.
            \item[Rola Lapunowa:] Funkcja Lapunowa definiuje tutaj "bezpieczny margines".
            Sterownik szuka takich sił w~silnikach nóg, aby pochodna funkcji Lapunowa $V$
            była jak najbardziej ujemna. To gwarantuje, że robot "pcha" swój stan
            w~stronę stabilnego pionu, nawet jeśli ktoś go popchnie.
            \item[Rola Programowania Kwadratowego:] W każdym kroku czasowym sterownik rozwiązuje
            problem optymalizacji kwadratowej, aby znaleźć najlepsze możliwe siły
            w~silnikach nóg.
            Celem jest minimalizacja zużycia energii lub sił wymaganych do
            utrzymania równowagi, przy jednoczesnym spełnieniu warunku stabilności
            narzuconego przez funkcję Lapunowa.
        \end{description}
        
        Dzięki temu robot może dynamicznie dostosowywać swoje ruchy, reagując na
        zmiany w~terenie lub zewnętrzne zakłócenia, jednocześnie zachowując
        stabilność
        i~efektywność energetyczną.
        \end{tcolorbox}
    \end{description}
    
    %=======================================================
    \subsection{Metoda "kroków wstecz" (Backstepping)}
    Metoda "kroków wstecz" jest to rekursywna, systematyczna metoda projektowania stabilizującego sprzężenia zwrotnego w pewnej klasie układów nieliniowych.
    W metodzie tej projektuje się kolejne elementy funkcji Lapunowa, zakładając, że możemy sterować poszczególnymi zmiennymi stanu i dzielimi wirtualnie układ na podsystemy i zaczynając od pierwszego idziemy "do końca".
    Dokładny opis tej metody jest dość obszerny, dlatego tutaj zostanie zaprezentowany tyko przykład.

    \begin{tcolorbox}[colback=green!5, colframe=green!75!black, title=Przykład]
    Celem jest stabilizacja zerowego punktu równowagi układu:
    $$
        \left\{
        \begin{array}{l}
             \dot{x}_{1} = x_{1} - x_{1}^{3} + x_{2}\\
             \dot{x}_{2} = u
        \end{array}
        \right\}
    $$
    \begin{itemize}
        \item Zakładamy, że można sterować układem bezpośrednio za pomocą zmiennej $x_{2}$ - rozważamy tylko pierwsze równanie i stusujemy do niego funkcję Lapunowa:
        $$
            V_{1}(x_{1}) = \frac{1}{2}x_{1}^{2}.
        $$
        \item Jej pochodna na trajektoriach systemu wynosi:
        $$
            \dot{V}_{1} = x_{1}(x_{1} - x_{1}^{3} + x_{2}) = x_{1}^{2} - x_{1}^{4} + x_{1}x_{2}.
        $$
        \item Pochodna ta ma pożądany zawsze ujemny składnik $-x_{1}^{4}$ oraz "przeszkadzający" dodatni składnik $x_{1}^{2}$, który powinien być wyeliminowany przez dobór $x_{2}$, na przykład jako:
        $$
            x_{2} =\alpha(x_{1}) = -x_{1} - kx_{1}, \, k>0.
        $$
        Wówczas:
        $$
            \dot{V}_{1}(x_{1}) = -k{x}_{1}^{2} - x_{1}^{4},
        $$
        zatem układ asymptotycznie stabilny.
        Jednak ponieważ nie możemy bezpośrednio sterować zmienną $x_{2}$ sterowanie $\alpha(x_{1})$ nazywamy sterowaniem wirtualnym.
        \item Wyznaczona wartość $x_{2}$ może być uznana jako wartość zadana dla kolejnego równania dynamiki - będzie to zadanie nadążania za wartością zadaną.
        \item Definiujemy uchyb regulacji:
        $$
            e = x_{2} - \alpha.
        $$
        Wówczas pierwsze równanie zapisujemy jako:
        $$
            \dot{x}_{1} = x_{1} - x_{1}^{3} + \alpha + e = -kx_{1} - x_{1}^{3} + e.
        $$
        Natomiast pochodna funkcji Lapunowa ma postać:
        $$
            \dot{V}(x_{1}) = x_{1}(-kx_{1} - x_{1}^{3} + e) = -kx_{1}^2 - x_{1}^{4} + x_{1}e.
        $$
        Dodatkowo dynamika błędu wynosi:
        $$
            \dot{e} = \dot{x}_{2} - \dot{\alpha} = u + (k+1)\dot{x}_{1} = u + (k+1)(-kx_{1} - x_{1}^{3} + e).
        $$
        \item Interesuje nas stabilizacja "w zerze" zmiennej $x_{1}$ oraz uchybu $e$, budujemy więc funkcję Lapunowa:
        $$
            V(x_{1}, e) = V_{1}(x_{1}) + \frac{1}{2}e^{2} = \frac{1}{2}x_{1}^2 + \frac{1}{2}e^{2}.
        $$
        Jej pochodna wynosi:
        $$
        \dot{V}(x_{1}, e) = \dot{V}_{1}(x_{1}) + e\dot{e} = -kx_{1}^2 - x_{1}^{4}+e \left[ u + x_{1} + (k+1)(-kx_{1} - x_{1}^{3} + e) \right]
        $$
        \item Dobierając sterowanie $u$ jako:
        $$
            u = -K - x_{1} + (k+1)(-kx_{1} - x_{1}^{3} + e), K>0
        $$
        redukujemy wyrażenie na pochodną funkcji Lapunowa do postaci:
        $$
            \dot{V}(x_{1}, e) = - kx_{1}^{2} - x_{1}^{4} - Ke^{2} < 0,
        $$
        zatem uzyskujemy stabilność asymptotyczną.
    \end{itemize}
    \end{tcolorbox}
    Można ogólnie wymienić trzy kluczowe elementy metody "kroków wstecz":
    \begin{itemize}
        \item sterowanie wirtualne,
        \item funkcja stabilizująca,
        \item uchyb śledzenia zadanego przebiegu sterowania wirtualnego.
    \end{itemize}

    \begin{tcolorbox}[colback=magenta!5, colframe=magenta!75!black, title=Zastosowanie w robotyce]
        W dronach (quadrocopterach) funkcje Lapunowa wykorzystuje się do projektowania tzw. algorytmów Backstepping. Pozwalają one na stabilne sterowanie dronem nawet podczas wykonywania gwałtownych uników czy lotu z dużymi prędkościami.
        \begin{description}
            \item [Na czym to polega?] Dron jest układem kaskadowym: najpierw sterujemy kątami nachylenia (orientacją), a to z kolei powoduje ruch w przestrzeni (pozycję).
            \item[Rola Lapunowa:] Projektuje się funkcję Lapunowa dla "błędu orientacji", a następnie rozszerza ją o "błąd pozycji". Sumaryczna funkcja $V(e)$ gwarantuje, że błąd śledzenia trasy zawsze będzie dążył do zera, niezależnie od tego, jak mocno dronem "rzuci" wiatr.
        \end{description}
    \end{tcolorbox}
    
%%%%%%%%%%%%%%%%%%%%%%%%%%%%%%%%%%%%%%%%%%%%%%%%%%%%%%%%%%%% 
\section{Jak to zrobić w MATLABie?}
    %=======================================================
    \subsection{Rozwiązanie macierzowego równania Lapunowa}
    MATLAB dostarcza w toolboxie Control System funkcję rozwiązującą numerycznie macierzowe równanie Lapunowa (\ref{eq:rownanie_lapunowa}).
    Jest to funkcja:
    \begin{lstlisting}[style=Matlab-editor]
        P = lyap(A, Q);
    \end{lstlisting}
    gdzie: \texttt{A} jest macierzą stanu układu liniowego, natomiast natomiast \texttt{Q} jest symetryczną macierzą dodatnio określoną.

    %=======================================================
    \subsection{Rysowanie zbiorów poziomicowych}
    Przydatne może być wyznaczanie zbiorów poziomicowych funkcji Lapunowa.
    Można to zrobić w następujący sposób:
    \begin{lstlisting}[style=Matlab-editor]
        [X1, X2] = mesh(-2:0.01:2, -2:0.01:2);  % Definiowanie obszaru, na ktorym obliczamy funkcje Lapunowa
        V = X1.^2 + 2*X1.*X2 + X2.^2;   % Obliczenie wartosci funkcji Lapunowa

        figure();
        hold on;
        grid on;
        xlabel("x_{1}");
        ylabel("x_{2}");
        contour(X, Y, V, 20);       % Rysujemy 20 konturow funkcji Lapunowa
        contour(X, Y, V, [1 1]);    % Rysujemy kontur, na ktorym V(x) == 1
    \end{lstlisting}

    %=======================================================
    \subsection{Symboliczne obliczenia macierzy Jacobiego}
    MATLAB dostarcza poprzez Symbolic Toolbox funkcje do symbolicznego wyznaczania macierzy Jacobiego funkcji nieliniowej:
    \begin{lstlisting}[style=Matlab-editor]
        % Definiowanie zmiennych symbolicznych
        syms x1 x2 x3

        % Dodawanie zalozen o zmiennych symbolicznych
        assume([x1 x2 x3], 'real')   % Zakladamy, ze x1, x2 i x3 przyjmuja tylko wartosci rzeczywiste

        % Definiujemy model nieliniowy
        f = [x2; x3; x1^2 * x3 - x3^3];

        % Wyznaczamy macierz Jacobiego
        A = jacobian(f,[x1, x2, x3]);
    \end{lstlisting}

    %=======================================================
    \subsection{Symboliczne obliczenia funkcji Lapunowa i jej pochodnej systemowej}
    Niejednokrotnie obliczenie pochodnej systemowej funkcji Lapunowa może dostarczyć trudności. Dlatego warto skorzystać z możliwości obliczeń symbolicznych dostarczanych przez Symblic Toolbox MATLABa.
    Można to zrobić w~następujący sposób:
    \begin{lstlisting}[style=Matlab-editor]
        % Definiowanie zmiennych symbolicznych
        syms x1(t) x2(t)    % Funkcje zmiennej t

        % Dodawanie zalozen o zmiennych symbolicznych
        assume([x1(t) x2(t)], 'real')   % Zakladamy, ze x1(t) i x2(t) przyjmuja tylko wartosci rzeczywiste

        % Definiujemy wektor zmiennych X i model obiektu
        X = [x1; x2];
        
        dx1 = -x1;
        dx2 = -x2;

        % Definiowanie funkcji Lapunowa
        P  = [1 0; 0 1];
        V = X' * P * X;

        % Wyznaczenie pochodnej systemowej
        dV = diff(V, t);

        % Uproszczenie postaci
        dV = subs(dV, {diff(x1,t), diff(x2,t)}, {dx1, dx2});
        dV = simplify(dV);

        % Sprawdzamy czy ujemna
        tf = isAlways(dV(t) <= 0)
    \end{lstlisting}

    %=======================================================
    \subsubsection{Wyznaczanie trajektorii bez Simulinka}
    MATLAB dostarcza kilku funkcji służących do numerycznego rozwiązywania równań różniczkowych.
    Jedną z nich jest \texttt{ode45}, rozwiązująca równanie metodą Rungego-Kutty 4. rzędu.
    \begin{lstlisting}[style=Matlab-editor]
        model_dynamiki = @(t,y)[x(2); -x(1) + x(2)x(1)^3];
        tspan = 0 : 0.01 : 10;
        y0 = [1; 1];
        
        [y, t] = ode45(model_dynamiki, tspan, y0);
    \end{lstlisting}
    Innymi funkcjami, rozwiązującymi równania różniczkowe, są np.: \texttt{ode23}, \texttt{ode113}.

%%%%%%%%%%%%%%%%%%%%%%%%%%%%%%%%%%%%%%%%%%%%%%%%%%%%%%%%%%%% 
\section{Przebieg ćwiczenia}
    %=======================================================
    \subsection{Pośrednia metoda Lapunowa}
    Dla każdego z podanych systemów opisanych równaniami (\ref{eq:posrednia_1}), (\ref{eq:posrednia_2}), (\ref{eq:posrednia_3}),  (\ref{eq:posrednia_4}),  (\ref{eq:posrednia_5}):
    \begin{itemize}
        \item zamodelować układ równań w Simulinku (lub użyć skryptu do jego rozwiązania),
        \item wyznaczyć portrety fazowe,
        \item znaleźć punkty równowagi,
        \item sprawdzić, czy można zastosować pośrednią metodę Lapunowa do badania stabilności punktów równowagi,
        \item zbadać stabilność wszystkich punktów równowagi,
        \item zbadać doświadczalnie obszar atrakcji asymptotycznie stabilnych punktów równowagi - przedstawić go na tle portretu fazowego.
    \end{itemize}
        %---------------------------------------------------
        \subsubsection{System 1}
        Rozważamy układ mechaniczny masa-sprężyna ze sprężyną o nieliniowej charakterystyce.
        Układ ten opisany jest równaniem (\ref{eq:posrednia_1}).
        \begin{equation}\label{eq:posrednia_1}
            \ddot{x} + b\dot{x} + cx + dx^{3} = 0, \, b,c > 0,\, |c|<|d|.
        \end{equation}
        Gdzie:
        \begin{itemize}
            \item $x$ - odchylenie od punktu równowagi,
            \item $b$ - współczynnik tarcia,
            \item $c$ i $d$ - parametry sprężyny.
        \end{itemize}
        Proszę rozważyć przypadek dla $d>0$ i $d<0$.

        %---------------------------------------------------
        \subsubsection{System 2}
        Rozważamy wahadło z tłumieniem, opisywane równaniem:
        \begin{equation}\label{eq:posrednia_2}
            \ddot{x} + \frac{c}{lm}\dot{x} + \frac{g}{l}\sin{x} = 0
        \end{equation}
        gdzie:
        \begin{itemize}
            \item $x$ - kąt wychylenia wahadła z punktu równowagi trwałej,
            \item $g$ - przyspieszenie ziemskie,
            \item $m$ - masa wahadła,
            \item $l$ - długość wahadła.
        \end{itemize}

        %---------------------------------------------------
        \subsubsection{System 3}
        Rozważamy układ równań Van der Pola:
        \begin{equation}
            \label{eq:posrednia_3}
            \left\{
            \begin{array}{l}
                 \dot{x}_{1} = x_{2} - x_{1}^{3} - ax_{1}\\
                 \dot{x}_{2} = -x_{1}
            \end{array}
            \right\}.
        \end{equation}
        Proszę zbadać zachowanie układu dla różnych wartości $a$.

        %---------------------------------------------------
        \subsubsection{System 4}
        Rozważamy układ równań:
        \begin{equation}
            \label{eq:posrednia_4}
            \left\{
            \begin{array}{l}
                 \dot{x}_{1} = -x_{1} + \sin{x_{2}}\\
                 \dot{x}_{2} = -x_{2}
            \end{array}
            \right\}.
        \end{equation}

        %---------------------------------------------------
        \subsubsection{System 5}
        Rozważamy układ równań:
        \begin{equation}
            \label{eq:posrednia_5}
            \left\{
            \begin{array}{l}
                 \dot{x}_{1} = 2x_{1} - x_{2} - x_{1}^{3}\\
                 \dot{x}_{2} = 4x_{1} - 2x_{2}
            \end{array}
            \right\}.
        \end{equation}

    %=======================================================
    \subsection{Bezpośrednia metoda Lapunowa i twierdzenie La Salle'a}
    Dla poniższych układów (\ref{eq:zadania_przyklad_1}), (\ref{eq:zadania_przyklad_2}), (\ref{eq:zadania_przyklad_3}), (\ref{eq:zadania_przyklad_4}):
    \begin{itemize}
        \item zamodelować układ równań w Simulinku (lub użyć skryptu do jego rozwiązania),
        \item wyznaczyć portrety fazowe,
        \item znaleźć punkty równowagi,
        \item zbadać stabilność punktów równowagi, stosując podane funkcje Lapunowa,
        \item dla każdej z podanych funkcji Lapunowa znaleźć analitycznie estymatę obszaru atrakcji,
        \item symulacyjnie określić obszar atrakcji,
        \item porównać obszary odnalezione analitycznie i symulacyjnie - przedstawić obydwa na tle portretu fazowego,
        \item porównać estymaty obszaru atrakcji uzyskane z użyciem różnych funkcji Lapunowa - przedstawić obydwa na jednym wykresie na tle portretu fazowego.
    \end{itemize}
        %---------------------------------------------------
        \subsubsection{System 1}
        Rozważamy układ:
        \begin{equation}
            \label{eq:zadania_przyklad_1}
            \left\{
            \begin{array}{l}
                 \dot{x}_{1} = -x_{1} + 2x_{1}^{2}x_{2}\\
                 \dot{x}_{2} = -x_{2}
            \end{array}
            \right\}.
        \end{equation}
        Dla tego układu rozpatrujemy dwóch kandydatów na funkcję Lapunowa:
        \begin{itemize}[(a)]
            \item $V_{1}^{1}(x) = \frac{1}{2}x_{1}^{2} + x_{2}^{2}$,
            \item $V_{1}^{2}(x) = \frac{x_{1}^{2}}{1 - x_{1}x_{2}} + x_{2}^{2}$.
        \end{itemize}

        %---------------------------------------------------
        \subsubsection{System 2}
        Rozważamy układ:
        \begin{equation}
            \label{eq:zadania_przyklad_2}
            \left\{
            \begin{array}{l}
                 \dot{x}_{1} = x_{2} - x_{1} + x_{1}^{3}\\
                 \dot{x}_{2} = -x_{1}
            \end{array}
            \right\}.
        \end{equation}
        Dla tego układu rozpatrujemy dwóch kandydatów na funkcję Lapunowa:
        \begin{itemize}[(a)]
            \item $V_{2}^{1}(x) = \frac{1}{2} \left( x_{1}^{2} + x_{2}^{2} \right)$,
            \item $V_{2}^{2}(x) = \frac{1}{2} \left( 2x_{1}^{2} + 2x_{1}x_{2} + 3x_{2}^{2} \right)$.
        \end{itemize}

        %---------------------------------------------------
        \subsubsection{System 3}
        Rozważamy układ (zauważmy, że nie rozważamy $x\in \mathbb{R}^{2}$, ale $x_{1} > -1$):
        \begin{equation}
            \label{eq:zadania_przyklad_3}
            \left\{
            \begin{array}{l}
                 \dot{x}_{1} = x_{2}\\
                 \dot{x}_{2} = -\frac{x_{2}}{1 + x_{1}} - \frac{x_{1}}{1 + x_{1}}
            \end{array}
            \right\}.
        \end{equation}
        Dla tego układu rozpatrujemy dwóch kandydatów na funkcję Lapunowa:
        \begin{itemize}[(a)]
            \item $V_{3}^{1}(x) = \frac{1}{2}x_{2}^{2} + \int_{0}^{x_{1}}{\frac{z}{1+z}dz}$,
            \item $V_{3}^{2}(x) = x_{1}^{2} + x_{1}x_{2} + x_{2}^{2}$.
        \end{itemize}

        \subsubsection{System 4}
        Rozważamy układ (szukamy cyklu granicznego):
        \begin{equation}
            \label{eq:zadania_przyklad_4}
            \left\{
            \begin{array}{l}
                 \dot{x}_{1} = x_{2} - x_{1}^7 \left( x_{1}^{4} + 2x_{2}^{2} - 10 \right)\\
                 \dot{x}_{2} = -x_{1}^{3} - 3x_{2}^{5} \left( x_{1}^{4} + 2x_{2}^{2} - 10 \right)
            \end{array}
            \right\}.
        \end{equation}
        Dla tego układu rozpatrujemy następującego kandydata na funkcję Lapunowa:
        \begin{itemize}[(a)]
            \item $V_{4}^{1}(x) = \left( x_{1}^{4} + 2x_{2}^{2} - 10 \right)^{2}$.
        \end{itemize}


%%%%%%%%%%%%%%%%%%%%%%%%%%%%%%%%%%%%%%%%%%%%%%%%%%%%%%%%%%%%
\newpage
\begin{thebibliography}{9}

\bibitem{Mitkowski2007}
  Mitkowski W., Baranowski J., Hajduk K., Korytowski A., Tutaj A.,
  \emph{Teoria Sterowania: Materiały Pomocnicze do Ćwiczeń Laboratoryjnych},
  AGH Uczelniane wydawnictwo Naukowo-Dydaktyczne,
  2007.

\bibitem{Mitkowski2019}
    Mitkowski W.,
    \emph{Zarys Teorii Sterowania},
    Wydawnictwa AGH,
    2019.

\bibitem{Amborski1978}
  Amborski K., Marusak A.,
  \emph{Teoria Sterowania w Ćwiczeniach},
  Państwowe Wydawnictwo Naukowe,
  1978.

\bibitem{Gessing1981}
  Gessing R., Latarnik M., Skrzywan-Kossek A.,
  \emph{Zbiór Zadań z Teorii Nieliniowych Układów Regulacji i Sterowania},
  Wydawnictwo Naukowo-Techniczne,
  1981.

\bibitem{Gibson1968}
  Gibson J. E.,
  \emph{Nieliniowe Układy Sterowania Automatycznego},
  Wydawnictwo Naukowo-Techniczne,
  1968.

\bibitem{Kabzinski2018}
  Kabziński J., Mosiołek P.,
  \emph{Projektowanie Nieliniowych Układów Sterowania},
  Wydawnictwo Naukowo PWN,
  2018.

\bibitem{LaSalle1966}
  La Salle J., Lefschetz S.,
  \emph{Zarys Teorii Stabilności Lapunowa i Jego Metody Bezpośredniej},
  PWN,
  1966.

\bibitem{Shinners1992}
  Shinners S. M.,
  \emph{Modern Control System Theory and Design},
  Wiley \& Sons,
  1992.

\bibitem{Kaczorek2005}
  Kaczorek T.,
  \emph{Podstawy Teorii Sterowania},
  Wydawnictwo Naukowo-Techniczne,
  2005.

\bibitem{Chang2019}
  Chang Y-C., Roohi N., Gao S., 
  \emph{Neural Lyapunov Control},
  33rd Conference on Neural Information Processing Systems, Vancouver, Canada.,
  2019.

\bibitem{Dai2021}
  Dai H., Landry B., Yang L., Pavone M. i Tedrake R.,
  \emph{Lyapunov-stable neural-network control},
  Robotics: Science and Systems ,
  2021.

\end{thebibliography}

\end{document}
