\documentclass{article}

\usepackage{polski}
\usepackage[utf8]{inputenc}

\usepackage[margin=0.75in]{geometry}

\usepackage{graphicx} % Required for inserting images
\usepackage{amsmath}
\usepackage{amssymb}
\usepackage{float}
\usepackage[shortlabels]{enumitem}
\usepackage{matlab-prettifier}
\usepackage{multirow}

\title{
    Teoria Sterowania\\
    \Large Optymalizacja Parametryczna
}
\author{Maciej Różewicz}
\date{2025}

\makeatletter         
\def\@maketitle{
\raggedright
\begin{center}
    \includegraphics[width=0.5\linewidth]{fig/agh_logo.PNG}\\[8ex]
\end{center}
\begin{center}
    {\huge \bfseries \sffamily \@title }\\[4ex] 
    {\large  \@author}\\[4ex] 
    \@date\\[8ex]
\end{center}}
\makeatother

\begin{document}
\maketitle
\newpage

%%%%%%%%%%%%%%%%%%%%%%%%%%%%%%%%%%%%%%%%%%%%%%%%%%%%%%%%%%%
\section{Cel ćwiczenia}
Celem ćwiczenia jest zapoznanie się z optymalizacją parametryczną układów sterowania.
Dotyczy ona problemów, gdzie struktura regulatora i model matematyczny obiektu są stałe, a optymalizowane są jedynie wartości niektórych parametrów.
Optymalność tę można rozumieć na różne sposoby, a dobór odpowiedniej funkcji celu, która jest optymalizowana, jest zagadnieniem zasadniczym.
Zwykle żąda się, aby układ regulacji możliwie dokładnie i szybko odtwarzał wartość zadaną, z drugiej zaś strony, by był możliwie odporny na sygnały zakłócające.
Często wymagania te są sprzeczne i poszukiwać trzeba pomiędzy nimi kompromisu.

%%%%%%%%%%%%%%%%%%%%%%%%%%%%%%%%%%%%%%%%%%%%%%%%%%%%%%%%%%%
\section{Wprowadzenie}
    %======================================================
    \subsection{Struktura układu regulacji automatycznej}
    W czasie realizacji ćwiczenia rozważać będziemy układ regulacji automatycznej (URA), w którym wyróżniamy dynamikę obiektu sterowania $G_{O}(s)$ i regulatora $G_{R}(s)$, a także sygnał wartości zadanej $R(s)$ i dwa nieznane sygnały zakłócające: zakłócenie obciążeniowe $D(s)$ i zakłócenie pomiarowe $N(s)$.
    Schemat takiego URA pokazano na rysunku (\ref{fig:uklad_regulacji}).
    \begin{figure}[H]
        \centering
        \includegraphics[width=0.75\linewidth]{fig/05_optymalizacja_parametryczna/uklad_regulacji.png}
        \caption{Schemat układu regulacji z zaznaczonymi sygnałami: $R(s)$ - wartość zadana, $E(s)$ - uchyb regulacji, $U(s)$ - sterowanie, $D(s)$ - zakłócenie obciążeniowe (obciążenie), $V(s)$ - wejście obiektu, $O(s)$ - wyjście obiektu, $Y(s)$ - wyjście układu regulacji.}
        \label{fig:uklad_regulacji}
    \end{figure}
    
    Układ taki opisuje się zwykle kilkoma transmitancjami przedstawionymi w tabeli (\ref{tab:uklad_regulacji_transmitancje}).
    
    \begin{table}[H]
        \centering
        \begin{tabular}{|c|p{10cm}|}
            \hline
            \textbf{Transmitancja} & \textbf{Opis} \\
            \hline
            $G(s) = \frac{Y(s)}{R(s)} = \frac{G_{R}(s)G_{O}(s)}{1 + G_{R}(s)G_{O}(s)}$ & Transmitancja wypadkowa - od wartości zadanej do wyjścia układu regulacji.\\
            \hline
            $S(s) = \frac{E(s)}{R(s)} = \frac{1}{1 + G_{R}(s)G_{O}(s)}$ & Transmitancja uchybowa - od wartości zadanej do uchybu regulacji. Nazywana również funkcją wrażliwości.\\
            \hline
            $Z_{D}(s) = \frac{Y(s)}{D(s)} = \frac{G_{O}(s)}{1 + G_{R}(s)G_{O}(s)}$ & Transmitancja zakłóceniowa - od obciążenia do wyjścia. Nazywana funkcją wrażliwości na obciążenie.\\
            \hline
            $Z_{N}(s) = \frac{U(s)}{N(s)} = \frac{G_{R}(s)}{1 + G_{R}(s)G_{O}(s)}$ & Transmitancja od szumu pomiarowego do sterowania. Nazywana funkcją wrażliwości sterowania na szum pomiarowy.\\
            \hline
            $G_{U}(s) = \frac{U(s)}{R(s)} = \frac{G_{R}(s)}{1 + G_{R}(s)G_{O}(s)}$ & Transmitancja od wartości zadanej do sterowania.\\
            \hline
            $G_{otwarty}(s) = G_{R}(s)G_{O}(s)$ & Transmitancja układu otwartego.\\
            \hline
        \end{tabular}
        \caption{Transmitancje opisujące układ regulacji automatycznej z rysunku (\ref{fig:uklad_regulacji}).}
        \label{tab:uklad_regulacji_transmitancje}
    \end{table}

    Aby URA był "całościowo" stabilny, każda z tych transmitancji musi być co najmniej stabilna.

    Ponadto można zauważyć, że $G(s)$ i $S(s)$ są komplementarne, tzn.:
    $$
        G(s) + S(s) = \frac{G_{R}(s)G_{O}(s)}{1 + G_{R}(s)G_{O}(s)} + \frac{1}{1 + G_{R}(s)G_{O}(s)} = 1,
    $$
    zatem optymalizując wskaźnik jakości związany z jedną z nich, optymalizujemy automatycznie analogiczny wskaźnik jakości związany z drugą.
    %======================================================
    \subsection{Wskaźniki jakości URA}
    Do oceny działania URA stosuje się różne wskaźniki jakości, zwane również funkcją celu.
    W zależności od doboru wskaźnika jakości, parametry układu będą dobierane w inny sposób.

        %--------------------------------------------------
        \subsubsection{Całkowe wskaźniki jakości}
        Jednymi z najczęściej używanych wskaźników jakości są tzw. całkowe wskaźniki jakości:
        \begin{enumerate}[(a)]
            \item całka z kwadratu uchybu sterowania:
                \begin{equation}
                    \label{eq:j1}
                    J_{1} = \int_{0}^{\infty}{e(t)^{2}dt},
                \end{equation}
    
            \item całka z $p$-tej potęgi uchybu regulacji (aby wskaźnik taki miał sens $p$ musi być parzyste):
                \begin{equation}
                    \label{eq:j2}
                    J_{2} = \int_{0}^{\infty}{e(t)^{p}dt},
                \end{equation}
    
            \item całka z modułu uchybu sterowania:
                \begin{equation}
                    \label{eq:j3}
                    J_{3} = \int_{0}^{\infty}{|e(t)|dt},
                \end{equation}
    
            \item całka ważona względem czasu sterowania (aby wskaźnik taki miał sens $p$ musi być parzyste):
                \begin{equation}
                    \label{eq:j4}
                    J_{4} = \int_{0}^{\infty}{e(t)^{p}t^{q}dt},
                \end{equation}
    
            \item całka z kwadratu uchybu sterowania i jej pochodnej:
                \begin{equation}
                    \label{eq:j5}
                    J_{5} = \int_{0}^{\infty}{\left(e(t)^{2} + \dot{e}(t)^{2}\right)dt},
                \end{equation}
    
            \item w przypadku rozpatrywania przestrzeni stanu można rozpatrywać całkę z dodatnio określonej formy kwadratowej stanu:
                \begin{equation}
                    \label{eq:j6}
                    J_{6} = \int_{0}^{\infty}{x^{T}(t)Hx(t)dt},\,H=H^{T}>0,
                \end{equation}
    
            \item wskaźnik mieszany uwzględniający koszt sterowania:
                \begin{equation}
                    \label{eq:j7}
                    J_{7} = \int_{0}^{\infty}{(\alpha_{1}e(t)^{2} + \alpha_{1}u(t)^{2})dt}, \alpha_{1}, \alpha_{2} \in \mathbf{R}_{+}.
                \end{equation}
        \end{enumerate}

        %--------------------------------------------------
        \subsubsection{Niecałkowe wskaźniki jakości}
        \begin{enumerate}[(a)]
            \item czas regulacji $\tau_{r}$ - jest to najmniejszy czas, począwszy od którego wartość sygnału wyjściowego obiektu nie różni się od wartości w stanie ustalonej $y_{\infty} = \lim_{t \rightarrow \infty}y(t)$ nie bardziej niż zadany procent $\gamma$:
            $$
                \tau_{1} = \sup\{ t \in [0, \infty) : |y(t) - y_{\infty}| > \gamma|y_{\infty}| \},
            $$
            \begin{figure}[H]
                \centering
                \includegraphics[width=0.4\linewidth]{fig/05_optymalizacja_parametryczna/czas_regulacji.png}
                \caption{Czas regulacji.}
                \label{fig:czas_regulaacji}
            \end{figure}
            \item czas narastania $\tau_{n}$ - jest to czas, w którym sygnał wyjściowy $y(t)$ narasta od $10\%$ do $90\%$ wartości w stanie ustalonym $y_{infty}$:
            $$
                \tau_{2} = \min\{ t \in [0, \infty) : y(t) = 0.9y_{\infty}\} - \min\{ t \in [0, \infty) : y(t) = 0.1y_{\infty}\},
            $$
            \begin{figure}[H]
                \centering
                \includegraphics[width=0.4\linewidth]{fig/05_optymalizacja_parametryczna/czas_narastania.png}
                \caption{Czas narastania.}
                \label{fig:czas_narastania}
            \end{figure}
            \item przeregulowanie $\kappa$ - stosunek maksymalnej wartości sygnału wyjściowego $y_{max}$ do wartości w stanie ustalonym:
            $$
                \kappa = \frac{y_{max} - y_{\infty}}{y_{\infty}}
            $$
            \begin{figure}[H]
                \centering
                \includegraphics[width=0.4\linewidth]{fig/05_optymalizacja_parametryczna/przeregulowanie.png}
                \caption{Przeregulowanie.}
                \label{fig:przeregulowanie}
            \end{figure}
            \item uchyb ustalony $e_{\infty}$ - jest to wartość uchybu regulacji w stanie ustalonym:
            $$
                e_{\infty} = \lim_{t \rightarrow \infty}{e(t)}
            $$
            \item zapas modułu $\Delta_{L}$ - różnica między poziomem modułu równego jeden, a poziomem modułu $L(\omega_{1})$ transmitancji widmowej $G_{otwarty}(i\omega)$, gdzie $\omega_{1}$ jest pulsacją, dla której faza transmitancji widmowej $\varphi(\omega_{1}) = -\pi$.
            \item zapas fazy $\Delta_{\varphi}$ - jest różnicą między argumentem $\varphi(\omega_{2})$ transmitancji widmowej $G_{otwarty}(i\omega)$, dla której moduł $|G_{otwarty}(i\omega)|$ wynosi $1$, a wartością $-\pi$.
            \begin{figure}[H]
                \centering
                \includegraphics[width=0.4\linewidth]{fig/05_optymalizacja_parametryczna/Screenshot from 2024-04-28 21-09-41.png}
                \caption{Zapas fazy i modułu.}
                \label{fig:zapas}
            \end{figure}
        \end{enumerate}
    
    %======================================================
    \subsection{Wyznaczanie całkowych wskaźników jakości na podstawie transmitancji}
    \begin{enumerate}[(a)]
        \item $J_{1} = \int_{0}^{\infty}{e(t)^{2}dt}$ : do wyznaczenia całki z kwadratu uchybu sterowania $e(t)$ na podstawie znajomości jego transformaty $E(s) = \frac{B(s)}{A(s)}$, gdzie stopień wielomianu $A(s)$ jest wyższy niż wielomianu $B(s)$, można wykorzystać równość, wynikającą z twierdzeń Reileigha i Persevala:
        $$
            J_{1}
            =
            \int_{0}^{\infty}{e(t)^{2}dt}
            =
            \frac{1}{2\pi}\int_{-i\infty}^{+i\infty}{E(s)E(-s)ds}.
        $$
        Dokładne wyprowadzenie zależności na analityczne wyznaczenie tej całki można znaleźć w pracach \cite{Mitkowski2007} i \cite{Gorecki1993}.
        W tej instrukcji ograniczymy się do zaznaczenia, że wyznaczana jest ona poprzez poszukiwanie wielomianu:
        $$
            C(s) = c_{n-1}s^{n-1} + \cdots + c_{1}s + c_{0},
        $$
        który spełnia zależność:
        $$
            B(s)B(-s) = A(-s)C(s) + A(s)C(-s).
        $$
        Na podstawie tej równości dostajemy zestaw równań na wartości współczynników $c_i$, wynikających z równości odpowiednich współczynników wielomianów z lewej i prawej strony znaku równości.
        A ostateczną formułą na wskaźnik jakości $J_{1}$ jest (\ref{eq:j2}):
        \begin{equation}
        \label{eq:j2}
            J_{1} = \frac{c_{n-1}}{a_{n}}.
        \end{equation}

        \item $J_{4} = \int_{0}^{\infty}{e(t)^{p}t^{q}dt}$: stosunkowo prostą analityczną formę na wskaźnik jakości ważony czasem sterowania można przedstawić dla $p =2$ oraz dla $q = 2$.
        Można ją wyznaczyć poprzez podstawienie:
        $$
            \gamma(t) = te(t),
        $$
        wówczas wskaźnik jakości ma postać:
        $$
            J_{4} = \int_{0}^{\infty}{\gamma(t)^{2}dt}.
        $$
        A ponieważ transformatę $\gamma(t)$ można przedstawić jako:
        $$
            \Gamma(s) = -\frac{\partial E(s)}{ds},
        $$
        to można zastosować procedurę jak dla wskaźnika $J_{1}$ tylko z transformatą $\Gamma(s)$.

        \item $J_{5} = \int_{0}^{\infty}{\left(e(t)^{2} + \dot{e}(t)^{2}\right)dt}$: całkę tę można rozbić na sumę:
        $$
            J_{5} = \int_{0}^{\infty}{e(t)^{2}dt}  + \int_{0}^{\infty}{\dot{e}(t)^{2}dt} = J_{1} + \int_{0}^{\infty}{\dot{e}(t)^{2}dt}.
        $$
        Część $J_{1}$ została już objaśniona, natomiast co do drugiego składnika sumy, to można zauważyć, że jego transformata będzie miała postać:
        $$
            \dot{e}(t) \rightarrow E_{p}(s) = sE(s).
        $$
        Zatem powtarzamy postępowanie dla $J_{1}$, ale z transformatą $E_{p}(s)$.
    \end{enumerate}
    %======================================================
    \subsection{Wyznaczanie niecałkowych wskaźników jakości}
    W ogólności trudno jest wyznaczyć analityczne zależności na przeregulowanie czy czas narastania.
    Możliwe jest to jednak dla układów drugiego rzędu, i na tych układach się tutaj skupimy.
    W tym celu przyjmiemy następującą parametryzację obiektu drugiego rzędu:
    $$
        G_{II}(s) = \frac{\omega_{n}^{2}}{s^{2} + 2\xi\omega_{n}s + \omega_{n}^{2}} = \frac{k}{(s+\sigma)^{2} + \omega_{d}^{2}}
    $$
    gdzie:
    \begin{itemize}
        \item $\omega_{n}$ - pulsacja drgań własnych,
        \item $\xi$ - współczynnik tłumienia,
        \item $\sigma$ - tłumienie względne,
        \item $\omega_{d}$ - pulsacja drgań tłumionych.
    \end{itemize}

    \begin{enumerate}[(a)]
        \item \textit{Czas regulacji} - dla $\gamma = 0.02$ - dla takiego poziomu uchybu regulacji przybliżony czas regulacji można przedstawić jako:
        $$
            \tau_{r} = \frac{4}{\xi\omega_{n}},
        $$
        %\item \textit{Czas narastania}
        \item \textit{Przeregulowanie} - przeregulowanie można okreśłić na podstawie wzorU:
        $$
            \kappa = e^{-\frac{\pi\xi}{\sqrt{1-\xi^2}}},
        $$
        natomiast czas pierwszego maksimum to:
        $$
            t_{p} = \frac{\pi}{\omega_{n}\sqrt{1-\xi^2}}.
        $$
    \end{enumerate}
    
    %======================================================
    %\subsection{Podejście numeryczne}
    %======================================================
    %\subsection{Optymalizacja odporności}
    %======================================================
    \subsection{Optymalizacja wielokryterialna}
    W praktyce inżynierskiej bardzo często spotykane są problemy, które wymagają optymalizacji względem więcej niż jednego kryterium.
    Nierzadko nie są one wzajemnie przeliczalne, więc nie można ich sprowadzić do pojedynczego skalarnego przypadku.
    Ponadto bywa, iż kryteria, względem których wykonujemy optymalizację, są wzajemnie sprzeczne - jak np. żądanie jak najszybszej reakcji na zmiany wartości zadanej i minimalizacja energii sterowania.
    Formalnie można zapisać, że zadanie optymalizacji wielokryterialnej jest poszukiwaniem minimum wektora wskaźników jakości $F(x) = \begin{bmatrix} F_{1} & \cdots & F_{n}\end{bmatrix} \in R^{n}$.
    Jednak problematyczne jest określenie, czym właściwie jest minimum wektora.
    Gdyż minimum dla każdego z kryteriów osiągane jest zwykle w innym punkcie.
    Dlatego poszukuje się rozwiązań kompromisowych, na tzw. zbiorze (froncie) Pareto, będącym zbiorem rozwiązań, dla których co najmniej jedno kryterium osiąga lepszy wynik niż dla innych punktów - rysunek (\ref{fig:enter-label}).
    \begin{figure}[H]
        \centering
        \includegraphics[width=0.35\linewidth]{fig/05_optymalizacja_parametryczna/Screenshot from 2024-04-28 00-49-05.png}
        \caption{Zbiór (front) Pareto.}
        \label{fig:enter-label}
    \end{figure}
    Nierzadko jednak można zadanie to uprościć i zastosować tzw. metodę skalaryzacji, która polega na przekształceniu zadania optymalizacji wielokryterialnej w zadanie optymalizacji skalarnej poprzez nadanie każdemu ze wskaźników jakości wagi i zsumowanie ich:
    $$
        \mathbf{F} = \nu F = \sum_{i}^{n}\nu_{i}F_{i}, \nu_{i} > 0.
    $$
    Można wówczas pokazać, że jeśli $\hat{\textbf{F}}$ jest minimum lokalnym, to odpowiadający mu wektor $F$ leży na zbiorze Pareto.
    

%%%%%%%%%%%%%%%%%%%%%%%%%%%%%%%%%%%%%%%%%%%%%%%%%%%%%%%%%%%
\section{Przykłady}
    %======================================================
    \subsection{Przykład 1}
    Rozważmy obiekt o transmitancji:
    $$
        G(s) = \frac{s+1}{s(s^2 + s + 1)}
    $$
    w układzie regulacji automatycznej z rysunku (\ref{fig:uklad_regulacji}), z regulatorem:
    \begin{enumerate}[(a)]
        \item typu P: $G_{R,P}(s) = K_{P}$,
        \item typu PI: $G_{R,PI}(s) = K_{P}(1 + \frac{1}{T_{i}s})$ zakładając, że $T_{i}, K_{P} > 0$.
    \end{enumerate}
    Znaleźć optymalne nastawy względem wskaźnika jakości $J_{1}$ i $J_{7}$ (dla $\alpha_{1} = \alpha_{2} = 1$).


    \begin{enumerate}[(a)]
        %% ^^^^^^^^^^^^^^^^^^^^^^^^^^^^^^^^^^^^^^^^^^^^^^^
        \item W celu rozwiązania zadania w pierwszej kolejności wyznaczamy transmitancję zastępczą URA:
        $$
            G(s) = \frac{Ks + K}{s^{3} + s^{2} + s(K+1) + K}.
        $$
        Wzmocnienie $K$ musi stabilizować URA, zatem wyznaczamy zakres wartości $K$, dla których układ będzie stabilny, można to zrobić np. za pomocą kryterium Hurwitza:
        $$
            H
            =
            \begin{bmatrix}
                1 & 1 & 0\\
                K & K+1 & 1\\
                0 & 0 & K
            \end{bmatrix}
            \implies
            \Delta_{1} = 1 > 0,
            \Delta_{2} = 1 > 0,
            \Delta_{3} = K > 0
            \implies K > 0.
        $$
        Znając zakres stabilnych $K$ wyznaczamy transmitancję uchybową:
        $$
            S(s) = \frac{s^3 + s^2 + s}{s^{3} + s^{2} + s(K+1) + K}.
        $$
        Ponieważ $E(s) = S(s)R(s)$, to transformata odpowiedzi na skok jednostkowy wynosi:
        $$
            E(s) = \frac{1}{s} \cdot \frac{s^3 + s^2 + s}{s^{3} + s^{2} + s(K+1) + K} = \frac{s^2 + s + 1}{s^{3} + s^{2} + s(K+1) + K}.
        $$
        Aby wyznaczyć "produkowane" sterowanie wyznaczamy transmitancję:
        $$
            G_{U} = \frac{K(s^3 + s^2 + s)}{s^{3} + s^{2} + s(K+1) + K}.
        $$
        A więc transformata sterowania wynosi:
        $$
            U(s) = \frac{1}{s} \cdot \frac{K(s^3 + s^2 + s)}{s^{3} + s^{2} + s(K+1) + K} = \frac{K(s^2 + s + 1)}{s^{3} + s^{2} + s(K+1) + K}.
        $$
        Posiadając wyznaczone transformaty sygnałów sterowania i uchybu sterowania można wyznaczyć wskaźniki jakości:
        \begin{enumerate}[label=(\arabic*.)]
            \item $J_{2}(K) = \int_{0}^{\infty}{e(t)^2dt}$ wyznaczamy zgodnie z przedstawioną metodą analitycznego wyznaczania takiej całki na podstawie transformaty Laplace'a $E(s) = \frac{A(s)}{B(s)}$ funkcji $e(t)$.
            W tym celu konieczne jest wyznaczenie wielomianu $C(s) = c_{2}s^2 + c_{1}s + c_{0}$, który spełnia równanie:
            $$
                B(s)B(-s) = A(-s)C(s) + A(s)C(-s),
            $$
            które po rozpisaniu i pogrupowaniu wyrazów daje następujący układ równań liniowych:
            $$
                \begin{bmatrix}
                    2K & 0 & 0\\
                    2 & -2-2K & 2K\\
                    0 & -2 & 2
                \end{bmatrix}
                \begin{bmatrix}
                    c_{0}\\
                    c_{1}\\
                    c_{2}
                \end{bmatrix}
                =
                \begin{bmatrix}
                    1\\
                    1\\
                    1
                \end{bmatrix}
                \implies
                \begin{bmatrix}
                    c_{0}\\
                    c_{1}\\
                    c_{2}
                \end{bmatrix}
                =
                \begin{bmatrix}
                    \frac{1}{2K}\\
                    \frac{K}{2} + \frac{1}{2K} - \frac{1}{2}\\
                    \frac{K}{2} + \frac{1}{2K}
                \end{bmatrix}.
            $$
            Zatem wskaźnik jakości wynosi:
            $$
                J_{1}(K) = \frac{c_{2}}{a_{3}} = \frac{K}{2} + \frac{1}{2K}.
            $$
            Aby więc wyznaczyć optymalną wartość $K$ mamy do rozwiązania zadanie optymalizacji z ograniczeniem nierównościowym $-K < 0$, można je rozwiązać poprzez zastosowanie warunków Khuna-Karusha-Tuckera:
            $$
                \left\{
                \begin{array}{c}
                    \frac{\partial J_{2}(K)}{\partial K} + \lambda(-K) = 0\\
                    -K < 0\\
                    \lambda(-K) = 0
                \end{array}
                \right\}
                \implies
                \left\{
                \begin{array}{c}
                    \frac{1}{2} - \frac{1}{2K^2} + \lambda(-K) = 0\\
                    -K < 0\\
                    \lambda(-K) = 0
                \end{array}
                \right\}
                \implies
                \left\{
                \begin{array}{c}
                    K = 1\\
                    \lambda = 0
                \end{array}
                \right\}.
            $$
            Zatem optymalna wartość wskaźnika jakości wynosi:
            $$
                J_{1}(1) = \frac{1}{2} + \frac{1}{2} = 1.
            $$
            \item $J_{7}(K) = \int_{0}^{\infty}{\left( e(t)^2 + u(t)^2 \right)dt} = \int_{0}^{\infty}{e(t)^2dt} + \int_{0}^{\infty}{u(t)^2dt}$ - pierwsza część sumy została wyznaczona w poprzednim przypadku.
            Natomiast druga część, zależna tylko od sterowania, zostanie wyznaczona tutaj w analogiczny sposób, szukając odpowiedniego wielomianu $C(s) = c_{2}^{u}s^2 + c_{1}^{u}s + c_{0}^{u}$, co w tym przypadku prowadzi do układu równań:
            $$
                \begin{bmatrix}
                    2K & 0 & 0\\
                    2 & -2-2K & 2K\\
                    0 & -2 & 2
                \end{bmatrix}
                \begin{bmatrix}
                    c_{0}^{u}\\
                    c_{1}^{u}\\
                    c_{2}^{u}
                \end{bmatrix}
                =
                \begin{bmatrix}
                    K^2\\
                    K^2\\
                    K^2
                \end{bmatrix}
                \implies
                \begin{bmatrix}
                    c_{0}^{u}\\
                    c_{1}^{u}\\
                    c_{2}^{u}
                \end{bmatrix}
                =
                \begin{bmatrix}
                    \frac{K}{2}\\
                    \frac{K}{2} (K^2 - K + 1)\\
                    \frac{K}{2}(K^2 + 1)
                \end{bmatrix}.
            $$
            Zatem część wskaźnika jakości związana ze sterowaniem wynosi:
            $$
                J_{7}^{u}(K) = \frac{c_{2}^{u}}{a_{3}} = \frac{K}{2}(K^2 + 1).
            $$
            Co daje całościowo wskaźnik jakości w postaci:
            $$
                J_{7}(K) = K + \frac{1}{2K} + \frac{K^3}{2}.
            $$
            A optymalną wartość $K$ ponownie odnajdujemy poprzez rozwiązanie problemu KKT:
            $$
                \left\{
                \begin{array}{c}
                    \frac{\partial J_{7}(K)}{\partial K} + \lambda(-K) = 0\\
                    -K < 0\\
                    \lambda(-K) = 0
                \end{array}
                \right\}
                \implies
                \left\{
                \begin{array}{c}
                    \frac{3}{2}K^2 - \frac{1}{2K^2} + 1 + \lambda(-K) = 0\\
                    -K < 0\\
                    \lambda(-K) = 0
                \end{array}
                \right\}
                \implies
                \left\{
                \begin{array}{c}
                    K = \frac{\sqrt{3}}{3}\\
                    \lambda = 0
                \end{array}
                \right\}.
            $$
            Zatem optymalna wartość wskaźnika jakości wynosi:
            $$
                J_{7}\left( \frac{\sqrt{3}}{3} \right) \approx 1.5396.
            $$
        \end{enumerate}

        Przebiegi czasowe uzyskanych uchybów regulacji i sterowania pokazano na rysunkach (\ref{fig:przyklad_1_uchyb}) i (\ref{fig:model_symulacja}):
        \begin{figure}[H]
            \centering
            \includegraphics[width=0.5\linewidth]{fig/05_optymalizacja_parametryczna/przyklad_1_uchyb.png}
            \caption{Przebiegi czasowe uchybu sterowania dla optymalnych nastaw regulatora względem wskaźnika jakości $J_{2}$ i $J_{7}$.}
            \label{fig:przyklad_1_uchyb}
        \end{figure}
        \begin{figure}[H]
            \centering
            \includegraphics[width=0.5\linewidth]{fig/05_optymalizacja_parametryczna/przyklad_1_sterowanie.png}
            \caption{Przebiegi czasowe  sterowania dla optymalnych nastaw regulatora względem wskaźnika jakości $J_{2}$ i $J_{7}$.}
            \label{fig:eprzyklad_1_sterowanie}
        \end{figure}

        Obydwa przypadki, dla wskaźnika $J_{2}$, jak i $J_{7}$ można również rozwiązać numerycznie, korzystając z modelu Simulinka, jak na rysunku (\ref{fig:model_symulacja}):
        \begin{figure}[H]
            \centering
            \includegraphics[width=0.5\linewidth]{fig/05_optymalizacja_parametryczna/Screenshot from 2024-04-27 18-23-11.png}
            \caption{Model symulacyjny  używany do optymalizacji współczynników regulatora.}
            \label{fig:model_symulacja}
        \end{figure}
        Na jego podstawie uzyskano praktycznie identyczne wyniki jak w rozwiązaniu analitycznym:
        \begin{figure}[H]
            \centering
            \includegraphics[width=0.5\linewidth]{fig/05_optymalizacja_parametryczna/przyklad_1_j7.png}
            \caption{Wzrost wskaźnika jakości w czasie symulacji.}
            \label{fig:j7_symulacja}
        \end{figure}
        %% ^^^^^^^^^^^^^^^^^^^^^^^^^^^^^^^^^^^^^^^^^^^^^^^
        \item Podobnie jak w przypadku regulatora proporcjonalnego w celu rozwiązania zadania w pierwszej kolejności wyznaczamy transmitancję zastępczą URA z regulatorem proporcjonalno-całkującym:
        $$
                G(s) = \frac{K_{P}\,T_{I}\,s^2+\left(K_{P}+K_{P}\,T_{I}\right)\,s+K_{P}}{T_{I}\,s^4+T_{I}\,s^3+\left(T_{I}+K_{P}\,T_{I}\right)\,s^2+\left(K_{P}+K_{P}\,T_{I}\right)\,s+K_{P}}.
        $$
        Parametry $K_{P}$ i $T_{D}$  muszą stabilizować URA, zatem wyznaczamy zakres tych wartości, dla których układ będzie stabilny, można to zrobić np. z za pomocą kryterium Hurwitza:
        $$
            \begin{array}{c}
                H
                =
                \begin{bmatrix}
                \begin{array}{cccc} T_{I} & K_{P}+K_{P}\,T_{I} & 0 & 0\\ T_{I} & T_{I}+K_{P}\,T_{I} & K_{P} & 0\\ 0 & T_{I} & K_{P}+K_{P}\,T_{I} & 0\\ 0 & T_{I} & T_{I}+K_{P}\,T_{I} & K_{P} \end{array}
                \end{bmatrix}
                \implies \\
                \\
                \implies 
                \Delta_{1} = T_{I},
                \Delta_{2} = {T_{I}}^2-K_{P}\,T_{I},
                \Delta_{3} = -{K_{P}}^2\,{T_{I}}^2-{K_{P}}^2\,T_{I}+K_{P}\,{T_{I}}^3,
                \Delta_{4} = -K_{P}\,\left({K_{P}}^2\,{T_{I}}^2+{K_{P}}^2\,T_{I}-K_{P}\,{T_{I}}^3\right)
                \implies\\
                \\
                \implies
                T_{i} > 0, K_P < T_I, K_P < \frac{T_I^2}{T_I + 1}.
                \end{array}
        $$
        Znając ograniczenia na parametry regulatora wyznaczamy transformaty dla sygnałów sterowania i uchybu sterowania w odpowiedzi na skok jednostkowy - $R(s) = \frac{1}{s}$, będą to odpowiednio:
        \begin{itemize}
            \item transformata uchybu:
            $$
                E(s)
                =
                \frac{T_{I}\,s^3+T_{I}\,s^2+T_{I}\,s}{T_{I}\,s^4+T_{I}\,s^3+\left(T_{I}+K_{P}\,T_{I}\right)\,s^2+\left(K_{P}+K_{P}\,T_{I}\right)\,s+K_{P}},
            $$
            \item transformata sterowania:
            $$
                U(s)
                =
                \frac{K_{P}\,T_{I}\,s^3+\left(K_{P}+K_{P}\,T_{I}\right)\,s^2+\left(K_{P}+K_{P}\,T_{I}\right)\,s+K_{P}}{T_{I}\,s^4+T_{I}\,s^3+\left(T_{I}+K_{P}\,T_{I}\right)\,s^2+\left(K_{P}+K_{P}\,T_{I}\right)\,s+K_{P}}.
            $$
        \end{itemize}

        \begin{enumerate}[label=(\arabic*.)]
            \item $J_{1}(K) = \int_{0}^{\infty}{e(t)^2dt}$ wyznaczamy metodą analityczną, jak poprzednio, szukając odpowiedniego wielomianu $C(s) = c_{2}s^2 + c_{1}s + c_{0}$, spełniającego warunek:
            $$
                B(s)B(-s) = A(-s)C(s) + A(s)C(-s).
            $$
            Na podstawie tej równości otrzymuje się układ równań, którego rozwiązanie daje wynik:
            $$
                c_{3} = \frac{{T_{I}}^2\,\left({K_{P}}^2\,T_{I}+{K_{P}}^2-2\,K_{P}\,T_{I}+{T_{I}}^2\right)}{2\,{K_{P}}^2\,\left(-{T_{I}}^2+K_{P}\,T_{I}+K_{P}\right)}.
            $$
            A zatem wskaźnik jakości przybiera ostatecznie:
            $$
                J_{1}(K_P, T_I) = \frac{c_3}{a_{4}} = \frac{{T_{I}}^2\,\left({K_{P}}^2\,T_{I}+{K_{P}}^2-2\,K_{P}\,T_{I}+{T_{I}}^2\right)}{2T_{I}\,{K_{P}}^2\,\left(-{T_{I}}^2+K_{P}\,T_{I}+K_{P}\right)}.
            $$

            \item $J_{7}(K) = \int_{0}^{\infty}{\left( e(t)^2 + u(t)^2 \right)dt} = \int_{0}^{\infty}{e(t)^2dt} + \int_{0}^{\infty}{u(t)^2dt}$ - pierwsza część sumy została wyznaczona w podpunkcie wyżej, dlatego tutaj wystarczy wyznaczyć część związaną tylko ze sterowaniem:
            $$
                J_{7}^{u} = \frac{-{K_{P}}^2\,{T_{I}}^3-{K_{P}}^2\,{T_{I}}^2+{K_{P}}^2\,T_{I}+{K_{P}}^2+K_{P}\,{T_{I}}^3-2\,K_{P}\,T_{I}+{T_{I}}^2}{2T_I\,\left(-{T_{I}}^2+K_{P}\,T_{I}+K_{P}\right)}.
            $$
        \end{enumerate}
        Są to dość skomplikowane wyrażenia, dlatego analityczne ich wyznaczenie zostanie pominięte.
    \end{enumerate} 
    
    %======================================================
    \subsection{Przykład 2}
    Rozważamy obiekt sterowania o transmitancji:
    $$
        G(s) = \frac{1}{s + 1}, 
    $$
    w układzie automatycznej regulacji, jak na rysunku (\ref{fig:uklad_regulacji}).
    Z regulatorem proporcjonalno-całkującym:
    $$
        G_{R,I}(s) = 1 + \frac{1}{T_{i}s}.
    $$
    Odnaleźć nastawę regulatora, która zapewnia najlepszą wartość wskaźnika jakości $J_{2}$, przy jednoczesnym minimalizowaniu wpływu zakłócenia obciążeniowego $D(s)$.
    Wskaźnik jakości można zapisać zatem jako sumę:
    $$
        J(T_I) = J_{1,E} + J_{1,D}
    $$
    gdzie:
    \begin{itemize}
        \item $J_{1,E} = \int_{0}^{\infty}{e(t)^2dt}$ - całka z uchybu sterowania w odpowiedzi na skok jednostkowy na wartości zadanej,
        \item $J_{1,D} = \int_{0}^{\infty}{d(t)^2dt}$ - całka z wyjścia w odpowiedzi na skok jednostkowy na obciążeniu.
    \end{itemize}


    Podobnie jak poprzednio, zaczynamy od wyznaczenia transmitancji zastępczej i sprawdzenia zakresu wartości $T_{I}$, które stabilizują układ po zamknięciu sprzężenia zwrotnego:
    $$
        G(s)
        =
        \frac{T_{I}\,s+1}{T_{I}\,s^2+2\,T_{I}\,s+1}.
    $$
    Łatwo zauważyć, iż wystarczy aby $T_{I} > 0$, aby układ był stabilny.

    Kolejno wyznaczamy transmitancje:
    \begin{itemize}
        \item uchybową:
        $$
            S(s) = \frac{T_{I}\,s^2+T_{I}\,s}{T_{I}\,s^2+2\,T_{I}\,s+1},
        $$
        \item obciążeniową:
        $$
            G_{D}(s) = \frac{T_{I}\,s}{T_{I}\,s^2+2\,T_{I}\,s+1}.
        $$
    \end{itemize}
    A dalej na ich podstawie transformaty odpowiedzi na skok jednostkowy:
    \begin{itemize}
        \item na wartości zadanej:
        $$
            E(s) = \frac{T_{I}\,s+T_{I}}{T_{I}\,s^2+2\,T_{I}\,s+1},
        $$
        \item na obciążeniu:
        $$
            Y(s) = \frac{T_{I}}{T_{I}\,s^2+2\,T_{I}\,s+1}.
        $$
    \end{itemize}
    Znając te transformaty wyznaczamy składniki wskaźnika jakości:
    \begin{itemize}
        \item $J_{1,E} = \frac{T_{I}}{4}+\frac{1}{4}$,
        \item $J_{1,D} = \frac{1}{4}$.
    \end{itemize}
    Sumarycznie otrzymujemy:
    $$
        J(T_I) = \frac{T_{I}}{4}+\frac{1}{2}.
    $$
    Zatem nie ma minimum, ale współczynnik jakości poprawia się wraz ze zmniejszaniem $T_I$. 

    %======================================================
    %\subsection{Przykład 3}
    %Rozważamy obiekt sterowania o transmitancji:
    %$$
    %    G(s) = \frac{1}{s^2 +s}, 
    %$$
    %w układzie automatycznej regulacji jak na rysunku (\ref{fig:uklad_regulacji}).
    %Z regulatorem proporcjonalnym:
    %$$
    %    G_{R,P}(s) = K_{P}.
    %$$
    %Odnaleźć optymalne nastawy regulatora, minimalizujące czas narastania, z ograniczeniem na co najwyżej 15\% przeregulowanie.
    

    %======================================================
    %\subsection{Przykład 4}
   
%%%%%%%%%%%%%%%%%%%%%%%%%%%%%%%%%%%%%%%%%%%%%%%%%%%%%%%%%%%
\section{Jak to zrobić w MATLABie?}
    %======================================================
    \subsection{Optymalizacja bez ograniczeń}
    MATLAB Symbolic Toolbox dostarcza kilka funkcji służących do rozwiązywania problemów optymalizacji bez ograniczeń.
    Są to np.:
    \begin{itemize}
        \item \texttt{fminsearch} - funkcja ta rozwiązuje problem optymalizacji metodami bezgradientowymi,
        jej przykładowe użycie to:
        \begin{lstlisting}[style=Matlab-editor]
            fun = @(x)(x(1)^2 + x2(2));     % Uchwyt do minimalizowanej funkcji
            x_0 = [1; 1];                   % Warunek poczatkowy
            x_opt = fminsearch(fun, x_0);   % Uruchomienie procedury minimalizacyjnej
        \end{lstlisting}
        \item \texttt{fminunc} - funkcja ta rozwiązuje problem optymalizacji metodami gradientowymi, gdzie gradient funkcji celu może być podany analitycznie przez użytkownika, lub w przypadku, gdy jest trudnoosiągalny, może być aproksymowany numerycznie wewnątrz funkcji \texttt{fminunc}.
        Jej przykładowe użycie to:
        \begin{lstlisting}[style=Matlab-editor]
        % Definicja minimalizowanej funkcji - moze zwracac dodatkowo gradient
        function [f,g] = rosenbrockwithgrad(x)
            % Minimalizowana funkcja
            f = 100*(x(2) - x(1)^2)^2 + (1-x(1))^2;

            % Gradient
            if nargout > 1
                g = [-400*(x(2)-x(1)^2)*x(1)-2*(1-x(1));
                    200*(x(2)-x(1)^2)];
        end

        fun = @rosenbrockwithgrad;      % Uchwyt do minimalizowanej funkcji
        x_0 = [1; 1];                   % Warunek poczatkowy
        x_opt = fminsearch(fun, x_0);   % Uruchomienie procedury minimalizacyjnej
    \end{lstlisting}
    \end{itemize}

    %======================================================
    \subsection{Optymalizacja z ograniczeniami}
    Zadanie optymalizacji z ograniczeniami, z użyciem metod gradientowych, można rozwiązać w MATLABie za pomocą funkcji \texttt{fmincon}, z Optimization Toolbox.
    Uwzględnia ona następujące ograniczenia:
    $$
        \left\{
        \begin{array}{ll}
            c(x) \leq 0    & \textrm{nieliniowe ograniczenia nierównościowe}\\
            c_{eq}(x) = 0    & \textrm{nieliniowe ograniczenia równościowe}\\
            Ax \leq b   & \textrm{liniowe ograniczenia nierównościowe}\\
            A_{eq}x \leq b_{eq}   & \textrm{liniowe ograniczenia równościowe}\\
            x_{l} \leq x \leq x_{u} & \textrm{ograniczenie na dolną i górną wartość}
        \end{array}
        \right\}
    $$
    Jeśli minimalizowana funkcja znana jest w postaci analitycznej, to można podać jawnie wartość jej gradientu, jednak gdy nie możemy go jawnie wyznaczyć, to funkcja użyje przybliżenia numerycznego.

    Dla przykładu przedstawiona zostanie minimalizacja funkcji Rosenbrocka:
    \begin{lstlisting}[style=Matlab-editor]
        function [f,g] = rosenbrockwithgrad(x)
            % Minimalizowana funkcja
            f = 100*(x(2) - x(1)^2)^2 + (1-x(1))^2;

            % Gradient
            if nargout > 1
                g = [-400*(x(2)-x(1)^2)*x(1)-2*(1-x(1));
                    200*(x(2)-x(1)^2)];
        end
    \end{lstlisting}

    Ustawiamy opcje minimalizacji
    \begin{lstlisting}[style=Matlab-editor]
        options = optimoptions("fmincon", ...
            "SpecifyObjectiveGradient", true, ...
            "Display", "iter", ...
            "PlotFcn", @optimplotfval, ...
            "OptimalityTolerance", 1e-8 ...
        );
    \end{lstlisting}

    Minimalizacja
    \begin{lstlisting}[style=Matlab-editor]
        fun = @rosenbrockwithgrad;
        x0 = [-1,2];
        A = [];
        b = [];
        Aeq = [];
        beq = [];
        lb = [-2,-2];
        ub = [2,2];
        nonlcon = @(x)100*(x(2)-x(1)^2)^2 + (1-x(1))^2;
        x = fmincon(fun, x0, A, b, Aeq, beq, lb, ub, nonlcon, options);
    \end{lstlisting}

    %======================================================
    \subsection{Analityczne rozwiązywanie zadania optymalizacji}
    Analityczne rozwiązanie zadania optymalizacji może polegać na analitycznym wyznaczeniu wskaźnika jakości w postaci $J_{1} = \frac{c_{n-1}}{a_{n}}$, a następnie numerycznej optymalizacji tej funkcji.
    Może być to zrealizowane tak jak w poniższym skrypcie:
    \begin{lstlisting}[style=Matlab-editor]
        %% Definicja transmitancji - funkcja tf() nie obsluguje zmiennych symbolicznych
        syms K_P
        syms Gr(s) Go(s)
        Gr(s) = K_P;
        Go(s) = (s+1) / (s^3 + s^2 + s);
        
        Gopen(s) = Gr(s)*Go(s);
        G(s) = Gopen(s) / (1 + Gopen(s));
        S(s) = 1 / (1 + Gopen(s));
        Gu(s) = Gr(s) / (1 + Gopen(s));
        
        %% Badanie stabilnosci - sprawdzenie warunkow wynikajacych z kryterium Hurwitza
        [~, den_closed] = numden(G);
        [den_closed_coeffs, ~] = coeffs(den_closed, s);
        den_closed_coeffs_array = den_closed_coeffs(s);
        
        [H, delta] = hurwitz(den_closed_coeffs_array);
        
        solve(delta > 0, K_P)
        
        %% Transformacja uchybowa - wyznaczenie odpowiedzi transmitancji uchybowej na skok jednostkowy
        [num_e, den_e] = numden(S / s);

        % Wypisanie formul LateXa dla mianownika i licznika transformaty e(t)
        clc;
        [num_e_coeffs, ss] = coeffs(num_e, s);
        latex(sum(num_e_coeffs .* ss))
        
        [den_e_coeffs, ss] = coeffs(den_e, s);
        latex(sum(den_e_coeffs .* ss))

        % Wyznaczenie wielomianu C(s)
        syms c0 c1 c2
        C(s) = c2*s^2 + c1*s + c0;
        
        left = num_e(s)*num_e(-s);
        left_coeffs = coeffs(left, s);
        right = den_e(-s)*C(s) + den_e(s)*C(-s);
        right_coeffs = coeffs(right, s);
        
        conditions = [ ...
            left_coeffs(1) == right_coeffs(1); ...
            left_coeffs(2) == right_coeffs(2); ...
            left_coeffs(3) == right_coeffs(3) ...
        ];
        c_solved = solve(conditions, [c0,c1, c2]);

        % Wyznaczenie wskaznika jakosci J_1
        c_2 = c_solved.c2;
        latex(c_2)
        
        den_e_array = den_e_coeffs(s);
        a3 = den_e_array(1);
        
        J = c_2 / a3;
        
        % Optymalizacja numeryczna wskaznika jakosci
        j_numeric_e = @(x)(double(subs(J, {K_P}, x)));
        
        options = optimoptions("fmincon", ...
            "Display", "iter", ...
            "PlotFcn", @optimplotfval, ...
            "OptimalityTolerance", 1e-4 ...
        );
        
        x0 = [5];
        A = [];
        b = [];
        Aeq = [];
        beq = [];
        lb = [0];
        ub = [];
        nonlcon = [];
        x_opt_e = fmincon(j_numeric_e, x0, A, b, Aeq, beq, lb, ub, nonlcon, options);
        
        %% Transformacja obciazeniowa - wyznaczenie odpowiedzi transmitancji obciazeniowej na skok jednostkowy
        [num_u, den_u] = numden(Gu / s);

        % Wypisanie formul LateXa dla mianownika i licznika transformaty e(t)
        clc;
        [num_u_coeffs, ss] = coeffs(num_u, s);
        latex(sum(num_u_coeffs .* ss))
        
        [den_u_coeffs, ss] = coeffs(den_u, s);
        latex(sum(den_u_coeffs .* ss))

        % Wyznaczenie wielomianu C(s)
        syms c0 c1 c2
        C(s) = c2*s^2 + c1*s + c0;
        
        left = num_u(s)*num_u(-s);
        left_coeffs = coeffs(left, s);
        right = den_u(-s)*C(s) + den_u(s)*C(-s);
        right_coeffs = coeffs(right, s);
        
        conditions = [ ...
            left_coeffs(1) == right_coeffs(1); ...
            left_coeffs(2) == right_coeffs(2); ...
            left_coeffs(3) == right_coeffs(3) ...
        ];
        c_solved = solve(conditions, [c0,c1, c2]);

        % Wyznaczenie wskaznika jakosci J_1 dla u(t)
        c_2 = c_solved.c2;
        latex(c_2)
        
        den_e_array = den_e_coeffs(s);
        a3 = den_e_array(1);
        
        J = c_2 / a3;
        
        % Optymalizacja numeryczna dla sumy e(t)^2 + u(t)^2
        j_numeric_eu = @(x)(j_numeric_e(x) + double(subs(J, {K_P}, x)));
        
        options = optimoptions("fmincon", ...
            "Display", "iter", ...
            "PlotFcn", @optimplotfval, ...
            "OptimalityTolerance", 1e-4 ...
        );
        
        x0 = [5];
        A = [];
        b = [];
        Aeq = [];
        beq = [];
        lb = [0];
        ub = [];
        nonlcon = [];
        x_opt_eu = fmincon(j_numeric_eu, x0, A, b, Aeq, beq, lb, ub, nonlcon, options);
    \end{lstlisting}

    %======================================================
    \subsection{Sumylacyjne rozwiązywanie zadania optymalizacji}
    W sytuacji, gdy nie znamy analitycznej formuły na wskaźnik jakości, np. z powodu jego znacznego skomplikowania, można posiłkować się wynikami symulacyjnymi.
    Można w tym celu wykorzystać model jak na rysunku (\ref{fig:model_symulacja}), w którym wyznaczane są odpowiednie przebiegi i całki.
    Wówczas można użyć jednej z podanych wyżej funkcji optymalizacyjnych z funkcją celu, zdefiniowaną w MATLABie jako:
    \begin{lstlisting}[style=Matlab-editor]
        function j = wskaznik_jakosci(x)
            global KP TI
            KP = x(1);
            TI = x(2);
            output = sim("ura.slx");
            j = output.je.Data(end) + output.ju.Data(end);
        end
    \end{lstlisting}

%%%%%%%%%%%%%%%%%%%%%%%%%%%%%%%%%%%%%%%%%%%%%%%%%%%%%%%%%%%
\section{Przebieg ćwiczenia}
Dla każdego z podanych systemów w trakcie analizy należy dla układów 1 i 2:
\begin{itemize}
    \item utworzyć model symulacyjny układu sterowania w Simulinku (łącznie z wejściami zakłóceń),
    \item wyznaczyć analitycznie ograniczenia na zakres zmiennych projektowych (w zależności od zadanej struktury regulatora) - zaznaczyć ten obszar na płaszczyźnie,
    \item wykonać numerycznie optymalizacje ze względu na wskaźnik jakości $J_{1}$,
    \item wykonać numerycznie optymalizację ze względu na wskaźnik jakości $J_{7}$,
    \item wykonać numerycznie optymalizację ze względu na odporność na obciążenie,
    \item wykonać numerycznie optymalizację ze względu na wskaźniki $J_{1}$  i na obciążenie,
    \item dla każdego zestawu dobranych nastaw regulatora wykonać symulacyjnie:
        \begin{itemize}
            \item odpowiedź na skok jednostkowy bez zakłóceń (pokazać wartości sterowania),
            \item odpowiedź na skok jednostkowy z zakłóceniem obciążeniowym (skok jednostkowy, ale w innej chwili niż wartość zadana)(pokazać wartości sterowania).
        \end{itemize}
\end{itemize}

    %======================================================
    \subsection{Układ 1}
    Dla obiektu opisywanego transmitancją:
    \begin{equation}
        \label{eq:system_1}
        G_{O}(s) = \frac{2s+3}{s^3+2s^2+s},
    \end{equation}
    w układzie regulacji automatycznej z rysunku (\ref{fig:uklad_regulacji}), rozważyć do optymalizacji parametrycznej regulatory:
    \begin{itemize}
        \item P: $G_{R,P}(s) = K_{P}$,
        \item PD: $G_{R,PD}(s) = K_{P}(1 + T_{D}s)$.
    \end{itemize}

    %======================================================
    \subsection{Układ 2}
    Dla obiektu opisywanego transmitancją:
    \begin{equation}
        \label{eq:system_1}
        G_{1}(s) = \frac{1}{s^2+0.5s+1},
    \end{equation}
    w układzie regulacji automatycznej z rysunku (\ref{fig:uklad_regulacji}), rozważyć do optymalizacji parametrycznej regulatory:
    \begin{itemize}
        \item P: $G_{R,P}(s) = K_{P}$,
        \item PI: $G_{R,PD}(s) = K_{P}(1 + \frac{1}{T_{D}s})$.
    \end{itemize}

    %======================================================
    \subsection{Układ 3}
    Wyznaczyć optymalny regulator proporcjonalny $G_{R}(s) = K$, dla obiektu (w URA jak z rysunku (\ref{fig:uklad_regulacji})):
    $$
        G_{O}(s) = \frac{1}{s^2 + s + 1},
    $$
    minimalizujący czas narastania, przy przeregulowaniu nie przekraczającym $10\%$.

%%%%%%%%%%%%%%%%%%%%%%%%%%%%%%%%%%%%%%%%%%%%%%%%%%%%%%%%%%%
\newpage
\begin{thebibliography}{9}

\bibitem{Mitkowski2007}
  Mitkowski, W., Baranowski, J., Hajduk, K., Korytowski, A., Tutaj, A.,
  \emph{Teoria Sterowania: Materiały Pomocnicze do Ćwiczeń Laboratoryjnych},
  AGH Uczelniane wydawnictwo Naukowo-Dydaktyczne,
  2007.

\bibitem{Amborski1978}
  Amborski, K., Marusak, A.,
  \emph{Teoria Sterowania w Ćwiczeniach},
  Państwowe Wydawnictwo Naukowe,
  1978.

\bibitem{Gessing1981}
  Gessing, R., Latarnik, M., Skrzywan-Kossek, A.,
  \emph{Zbiór Zadań z Teorii Nieliniowych Układów Regulacji i Sterowania},
  Wydawnictwo Naukowo-Techniczne,
  1981.

\bibitem{Gibson1968}
  Gibson, J. E.,
  \emph{Nieliniowe Układy Sterowania Automatycznego},
  Wydawnictwo Naukowo-Techniczne,
  1968.

\bibitem{Grabowski1999}
  Grabowski P.,
  \emph{Stabilność Układów Lurie},
  AGH Uczelniane Wydawnictwo Naukowo-Dydaktyczne,
  1999.


\bibitem{Vukic2003}
  Vukic Z., Kuljaca L., Donlagic D., Tesnjak S.,
  \emph{Nonlinear Control Systems},
  Marcel Dekker,
  2003.

\bibitem{Kabzinski2020}
  Kabziński J.,
  \emph{Teoria Sterowania: Projektowanie Układów Regulacji},
  Państwowe Wydawnictwo Naukowe,
  2020.

\bibitem{Gorecki1993}
  Górecki H.,
  \emph{Optymalizacja Systemów Dynamicznych},
  Państwowe Wydawnictwo Naukowe,
  1993.

\end{thebibliography}

\end{document}
